\documentclass[a4paper,12pt,leqno]{article}

%%%%%%%%%%%%%%%%%%%%%%%%%%%%%%%%%%%%%%%%%%%%%%%%%%%%%%%%%%%%%%%%%%%%%%%%%%%%%%%%
%                                                                              %
%                                     TODO                                     %
%                                                                              %
%%%%%%%%%%%%%%%%%%%%%%%%%%%%%%%%%%%%%%%%%%%%%%%%%%%%%%%%%%%%%%%%%%%%%%%%%%%%%%%%
%                                                                              %
% - Double check gutter margin is included                                     %
% - Make sure title page wording and formatting is acceptable                  %
%   - Should supervisors be included on title page?                            %
% - Does GDPR need to be included in Declaration?                              %
% - Make sure ADAPT acknowledgement is correct                                 %
% - Check all references are accurate and not abbreviated                      %
% - Find out if "Summary" should be included versus Abstract                   %
% - Include lists of tables/figures(?) and double-check for correct labels     %
% - Ensure 100k word limit is not reached... lol                               %
% - Check that appendices and other sections not automatically included in the %
%   TOC are accurate and up to date.                                           %
% * Should Related Publications be included?                                   %
% - Is Times a preferred font/does it matter?                                  %
% * Don't have solitary sections (e.g. 4.1.1 without 4.1.2)                    %
% - Vinny Wade and Dave Lewis may be internal examiners, so try to address the %
%   issues they brought up during the transfer                                 %
% - Should FST discussion bother talking about FS automata/transducers?        %
% - Remind examiners regularly of core concepts, don't expect them to remember %
% - Should lit review cover smaller topics eg scheduling or just major themes? %
% - Sections should have conclusions as well as openings                       %
% - Table references should be capitalised                                     %
%                                                                              %
%%%%%%%%%%%%%%%%%%%%%%%%%%%%%%%%%%%%%%%%%%%%%%%%%%%%%%%%%%%%%%%%%%%%%%%%%%%%%%%%


% Use I_A as correction to below for intervals, not nominals
% Use N_A for first order nominals, as first order ensures each element occurs exactly once, rather than L_A which allows infinite strings for the gap calculation thing

\usepackage{natbib}
%\usepackage{times}
\usepackage{url}
\usepackage{array}
\usepackage{latexsym}
\usepackage{caption}
\usepackage{amssymb,amsmath,amscd}
\usepackage{stmaryrd}
\usepackage{xcolor}
\usepackage[hang,flushmargin]{footmisc}
\usepackage{graphicx}  %%% for including graphics
%\usepackage[bindingoffset=10mm,margin=25mm]{geometry} % I think this is the required "gutter margin"
\usepackage[margin=25mm]{geometry}
\usepackage{setspace}
\usepackage{diagbox}
\usepackage[nodayofweek]{datetime}
\usepackage{listings}

\captionsetup[figure]{font=small,labelfont=small,justification=centering}

\def\drs#1#2{
\renewcommand{\arraystretch}{0.8}
\begin{tabular}[c]{|>{\hspace{1pc}}c<{\hspace{1pc}}|}
	\hline #1 \\
	\hline #2 \\
	\hline
\end{tabular}
\renewcommand{\arraystretch}{1}
}

% Tim's custom commands
\newcommand{\bc}{{\rm b\!c}}
\newcommand{\unpad}{\mbox{{\rm unpad}}}
\newcommand{\vph}[1]{\vphantom{#1}}
\newcommand{\sta}[2]{\stackrel{#1}{#2}}


% David's custom commands
\newcommand{\ebox}[1]{\fbox{$\vph{'(),}#1$}}
\newcommand{\eboxl}[1]{\fbox{$\vph{'}#1$}}
\newcommand{\eboxh}[1]{\fbox{$\vph{,}#1$}}
\newcommand{\eboxb}[1]{\fbox{$\vph{@}#1$}}

\newcommand{\nbBefore}[2]{\ebox{#1}\ebox{}\ebox{#2}}
\newcommand{\nbMeets}[2]{\ebox{#1}\ebox{#2}}
\newcommand{\nbOverlaps}[2]{\ebox{#1}\ebox{#1,#2}\ebox{#2}}
\newcommand{\nbDuring}[2]{\ebox{#2}\ebox{#1,#2}\ebox{#2}}
\newcommand{\nbStarts}[2]{\ebox{#1,#2}\ebox{#2}}
\newcommand{\nbFinishes}[2]{\ebox{#2}\ebox{#1,#2}}
\newcommand{\nbEquals}[2]{\ebox{#1,#2}}

\newcommand{\nbAfter}[2]{\nbBefore{#2}{#1}}
\newcommand{\nbiMeets}[2]{\nbMeets{#2}{#1}}
\newcommand{\nbiOverlaps}[2]{\nbOverlaps{#2}{#1}}
\newcommand{\nbiDuring}[2]{\nbDuring{#2}{#1}}
\newcommand{\nbiStarts}[2]{\nbStarts{#2}{#1}}
\newcommand{\nbiFinishes}[2]{\nbFinishes{#2}{#1}}

\newcommand{\Before}[2]{\ebox{}\nbBefore{#1}{#2}\ebox{}}
\newcommand{\Meets}[2]{\ebox{}\nbMeets{#1}{#2}\ebox{}}
\newcommand{\Overlaps}[2]{\ebox{}\nbOverlaps{#1}{#2}\ebox{}}
\newcommand{\During}[2]{\ebox{}\nbDuring{#1}{#2}\ebox{}}
\newcommand{\Starts}[2]{\ebox{}\nbStarts{#1}{#2}\ebox{}}
\newcommand{\Finishes}[2]{\ebox{}\nbFinishes{#1}{#2}\ebox{}}
\newcommand{\Equals}[2]{\ebox{}\nbEquals{#1}{#2}\ebox{}}
\newcommand{\After}[2]{\ebox{}\nbAfter{#1}{#2}\ebox{}}
\newcommand{\iMeets}[2]{\ebox{}\nbiMeets{#1}{#2}\ebox{}}
\newcommand{\iOverlaps}[2]{\ebox{}\nbiOverlaps{#1}{#2}\ebox{}}
\newcommand{\iDuring}[2]{\ebox{}\nbiDuring{#1}{#2}\ebox{}}
\newcommand{\iStarts}[2]{\ebox{}\nbiStarts{#1}{#2}\ebox{}}
\newcommand{\iFinishes}[2]{\ebox{}\nbiFinishes{#1}{#2}\ebox{}}

\newcommand{\cBefore}[2]{`$#1$  before $#2$' -- \Before{#1}{#2}}
\newcommand{\cMeets}[2]{`$#1$ meets $#2$' -- \Meets{#1}{#2}}
\newcommand{\cOverlaps}[2]{`$#1$ overlaps $#2$' -- \Overlaps{#1}{#2}}
\newcommand{\cDuring}[2]{`$#1$ during $#2$' -- \During{#1}{#2}}
\newcommand{\cStarts}[2]{`$#1$ starts $#2$' -- \Starts{#1}{#2}}
\newcommand{\cFinishes}[2]{`$#1$ finishes $#2$' -- \Finishes{#1}{#2}}
\newcommand{\cEquals}[2]{`$#1$ equals $#2$' -- \Equals{#1}{#2}}
\newcommand{\cAfter}[2]{`$#1$ after $#2$' -- \After{#1}{#2}}
\newcommand{\ciMeets}[2]{`$#1$ imet by $#2$' -- \iMeets{#1}{#2}}
\newcommand{\ciOverlaps}[2]{`$#1$ overlapped by $#2$' -- \iOverlaps{#1}{#2}}
\newcommand{\ciDuring}[2]{`$#1$ contains $#2$' -- \iDuring{#1}{#2}}
\newcommand{\ciStarts}[2]{`$#1$ started by $#2$' -- \iStarts{#1}{#2}}
\newcommand{\ciFinishes}[2]{`$#1$ finished by $#2$' -- \iFinishes{#1}{#2}}


\newcommand{\siBefore}[2]{\ebox{\alpha(#1),\alpha(#2)}\ebox{\alpha(#2)}\ebox{\alpha(#2),\omega(#1)}\ebox{\omega(#1)}\ebox{\omega(#1),\omega(#2)}}
\newcommand{\siMeets}[2]{\ebox{\alpha(#1),\alpha(#2)}\ebox{\alpha(#2)}\ebox{\omega(#1)}\ebox{\omega(#1),\omega(#2)}}
\newcommand{\siOverlaps}[2]{\ebox{\alpha(#1),\alpha(#2)}\ebox{\alpha(#2)}\ebox{}\ebox{\omega(#1)}\ebox{\omega(#1),\omega(#2)}}
\newcommand{\siDuring}[2]{\ebox{\alpha(#1),\alpha(#2)}\ebox{\alpha(#1)}\ebox{}\ebox{\omega(#1)}\ebox{\omega(#1),\omega(#2)}}
\newcommand{\siStarts}[2]{\ebox{\alpha(#1),\alpha(#2)}\ebox{}\ebox{\omega(#1)}\ebox{\omega(#1),\omega(#2)}}
\newcommand{\siFinishes}[2]{\ebox{\alpha(#1),\alpha(#2)}\ebox{\alpha(#1)}\ebox{}\ebox{\omega(#1),\omega(#2)}}
\newcommand{\siEquals}[2]{\ebox{\alpha(#1),\alpha(#2)}\ebox{}\ebox{\omega(#1),\omega(#2)}}
\newcommand{\siAfter}[2]{\ebox{\alpha(#1),\alpha(#2)}\ebox{\alpha(#1)}\ebox{\alpha(#1),\omega(#2)}\ebox{\omega(#2)}\ebox{\omega(#1),\omega(#2)}}
\newcommand{\siiMeets}[2]{\ebox{\alpha(#1),\alpha(#2)}\ebox{\alpha(#1)}\ebox{\omega(#2)}\ebox{\omega(#1),\omega(#2)}}
\newcommand{\siiOverlaps}[2]{\ebox{\alpha(#1),\alpha(#2)}\ebox{\alpha(#1)}\ebox{}\ebox{\omega(#2)}\ebox{\omega(#1),\omega(#2)}}
\newcommand{\siiDuring}[2]{\ebox{\alpha(#1),\alpha(#2)}\ebox{\alpha(#2)}\ebox{}\ebox{\omega(#2)}\ebox{\omega(#1),\omega(#2)}}
\newcommand{\siiStarts}[2]{\ebox{\alpha(#1),\alpha(#2)}\ebox{}\ebox{\omega(#2)}\ebox{\omega(#1),\omega(#2)}}
\newcommand{\siiFinishes}[2]{\ebox{\alpha(#1),\alpha(#2)}\ebox{\alpha(#2)}\ebox{}\ebox{\omega(#1),\omega(#2)}}


\newcommand{\projects}[3]{\bc(\rho_{#3}(#1)) = #2}
\newcommand{\projectsVoc}[2]{\projects{#1}{#2}{voc(#2)}}

\renewcommand{\sp}{~\&~}
\newcommand{\spasync}{~\&_*~}
\newcommand{\spsigma}[1][\Sigma, \Sigma']{~\&_{#1}~}
\newcommand{\spvc}{~\&_{v\!c}~}
\newcommand{\V}{\mathcal{V}}

\renewcommand{\emptyset}{\varnothing}
\renewcommand{\phi}{\varphi}

\makeatletter
\newcommand*\bigcdot{\mathpalette\bigcdot@{0.75}}
\newcommand*\bigcdot@[2]{\mathbin{\vcenter{\hbox{\scalebox{#2}{$\m@th#1\bullet$}}}}}
\makeatother

\mathchardef\mhyphen="2D



% Allows entry of EventStrings as |a|{}|b,c|d|
% Use {} for empty box
\usepackage{etoolbox}
\DeclareListParser{\PipeParser}{|}
\newcommand{\EventString}[1]{%
	\renewcommand*{\do}[1]{\ebox{##1}}%
	\PipeParser{#1}%
}

\newenvironment{ctable}[3][]
{
	\def\myctabenvargumentII{#3}
	\def\myctabenvargumentLABEL{#1}
	\begin{center}
		\begin{tabular}[h!]{#2}
}{
		\end{tabular}
		\captionof{table}{\myctabenvargumentII}\myctabenvargumentLABEL
	\end{center}
}

\defcitealias{aristotlePhysicsIV}{Physics IV}

\newcommand{\refneeded}[1][]{{\color{red}[Figure reference needed!#1]}}
\newcommand{\citeneeded}[1][]{{\color{red}[Citation needed!#1]}}

\newcommand{\nb}[1]{{\color{red}[NB\footnote{{\color{red}#1}}]}}
\newcommand{\ipp}{(\refstepcounter{equation}\theequation)}

%\usepackage[nottoc,numbib]{tocbibind} %This includes the bibliography as a numbered section in the TOC
\usepackage[nottoc]{tocbibind}
% This includes paragraphs in TOC, but references to paras will be messed up
% \setcounter{tocdepth}{4}
% \setcounter{secnumdepth}{4}

\usepackage{cleveref}
\crefname{section}{\textbf{\S\!}}{\textbf{\S\!}}
\Crefname{section}{\textbf{\S\!}}{\textbf{\S\!}}
\crefname{equation}{}{}
\Crefname{equation}{}{}

\doublespacing
\linespread{2} %Not sure about this, may need to reset to 1(?)
\newdateformat{monthyeardate}{\monthname[\THEMONTH] \THEYEAR}

\title{\textbf{Strings for Temporal Annotation and\\Semantic Representation of Events}}

\author{by\\{\textbf{David Woods}}\bigskip\bigskip}

\date{\parbox{\linewidth}{\centering%
		{\large A dissertation submitted\\in fulfillment of the requirements\\for the Degree of\\\textbf{Doctor of Philosophy}}\\		
		\bigskip\bigskip\bigskip
		{\Large \textbf{University of Dublin, Trinity College}}\\\endgraf \monthyeardate\today}{\small \vspace{\fill} Supervised by: Dr Tim Fernando, Dr Carl Vogel}}

%%%%%%%%%%%%%%%%%%%%%%%%%%%%%%%%%%%%%%%%%%%%%%%%%%%%%%%%%%%%%%%%%%%%%%%%%%%%%%%%
%                                                                              %
%                                Preamble Begins                               %
%                                                                              %
%%%%%%%%%%%%%%%%%%%%%%%%%%%%%%%%%%%%%%%%%%%%%%%%%%%%%%%%%%%%%%%%%%%%%%%%%%%%%%%%

\begin{document}
\maketitle
\thispagestyle{empty}
%\linespread{1}

\newpage
\pagenumbering{roman}
\section*{Declaration}
\addcontentsline{toc}{section}{Declaration}
\noindent
I, the undersigned, declare that this thesis has not been submitted as an exercise for a degree at this or any other university and it is entirely my own work.\par

\vspace{2em}

\noindent
I, the undersigned, agree to deposit this thesis in the University's open access institutional repository or allow the Library to do so on my behalf, subject to Irish Copyright Legislation and Trinity College Library conditions of use and acknowledgement.\par

\vspace{2em}

\noindent
I, the undersigned, consent to the examiner retaining a copy of the thesis beyond the examining period, should they so wish (EU GDPR May 2018).

\vspace{\fill}

\begin{table*}[!htbp]
	\flushright
	\begin{tabular}{l}
		\makebox[10cm]{\hrulefill}\\[0.5cm]
		David Woods\\[0.25cm]
		{\monthyeardate\today}
	\end{tabular}
\end{table*}

\vspace{5em}

\newpage
\begin{abstract}
\addcontentsline{toc}{section}{Abstract}
\noindent
%This work describes the use of strings as models for the representation of temporal data -- i.e. events and times -- to form the basis of a framework for reasoning about that data. Some of the relevant motivating literature is examined, and a breakdown is given of the work done to develop and flesh out the framework so far, including discussion on superposition for collation of information into single, timeline-like strings, and projection which allows for the identification of temporal relations between arbitrary events and times from the strings. Possible ways of treating incomplete information are also looked at, including moving from intervals as primitives to semi-intervals. Some work done to implement this framework in code is described, with a discussion of potential applications in modern intelligent systems, including tooling for annotation software.
\end{abstract}

\newpage
\section*{Acknowledgements}
\addcontentsline{toc}{section}{Acknowledgements}
%My thanks to Tim for his patience and understanding, to Carl for guiding and reassuring comments, and to my friends and family for their continuous encouragement and support. In particular, Brian, who had the misfortune to be staying with me while I was working on this report, and the members of DU Trampoline club, who have had a more bad-tempered coach of late.
Although I wrote this thesis alone, a number of people supported me along the way, and made it possible for me to reach the end.

I would like to thank my parents, Margaret and Graham, who never doubted I would do my best; my brother, Fergus, who has often been a much-needed reminder of normal life; Conor and Katie, who inspired me to start; Caitr\'{i}ona and Se\'{a}n, who regularly found me comforting distractions; Adelais, who encouraged me to make it over the finish line and made sure I remembered life outside of writing; and all the members of DU Trampoline Club, who gave me a reason to stick around. I hope it was worth it!

My appreciation goes also to my supervisors: Tim, for being patient with me even when I was struggling, and Carl, who often managed to reassure me that I wasn't going down completely the wrong path. Thank you both for not giving up on me.

No thanks go to the COVID-19 global pandemic which disrupted my final year. Here's to survival, and here's to the memory of those that did not.

This research is supported by Science Foundation Ireland (SFI) through the CNGL 
Programme (Grant 12/CE/I2267) in the ADAPT Centre 
(\url{https://www.adaptcentre.ie}) at Trinity College Dublin. The
ADAPT Centre for Digital Content Technology is funded under the SFI Research 
Centres Programme (Grant 13/RC/2106) and is co-funded under the European 
Regional Development Fund.


\newpage
\section*{Related Publications}
\addcontentsline{toc}{section}{Related Publications}
% what they are and how they figure in, but don't go into it
During the course of my studies, I was an author on three papers that were accepted for publication, listed below.

\begin{itemize}
	% \citefullauthor not working??
	\item \textit{Towards Efficient String Processing of Annotated Events} (Woods, Fernando, and Vogel, \citeyear{woods2017towards}), describing the use of strings to model temporal data such as could be found in text annotated with ISO-TimeML. Presented at the 13th Joint ISO-ACL Workshop on Interoperable Semantic Annotation in Montpellier, France.
	\item \textit{Improving String Processing for Temporal Relations} (Woods and Fernando, \citeyear{woods2018improving}), discussing refinements to the previously described string-based model, such as varied granularity. Presented at the 14th Joint ISO-ACL Workshop on Interoperable Semantic Annotation, colocated with COLING 2018 in Santa F\'{e}, New Mexico, USA.
	\item \textit{MSO with tests and reducts} (Fernando, Woods, and Vogel, \citeyear{fernando2019mso}), discussing differing string granularities in the context of tests within Monadic Second Order logic. Presented at the 14th International Conference on Finite-State Methods and Natural Language Processing in Dresden, Germany.
\end{itemize}
\newpage
\addcontentsline{toc}{section}{Table of Contents}
\tableofcontents
\newpage
\listoffigures
\listoftables
\newpage
\pagenumbering{arabic}
%%%%%%%%%%%%%%%%%%%%%%%%%%%%%%%%%%%%%%%%%%%%%%%%%%%%%%%%%%%%%%%%%%%%%%%%%%%%%%%%
%                                                                              %
%                            Actual Document Begins                            %
%                                                                              %
%%%%%%%%%%%%%%%%%%%%%%%%%%%%%%%%%%%%%%%%%%%%%%%%%%%%%%%%%%%%%%%%%%%%%%%%%%%%%%%%
\section{Introduction}\label{sec:intro}
This thesis will explore and describe the use of strings as models to represent temporal information---data concerning times and events---for use in computational systems which deal with knowledge-based reasoning in some way. Such systems rely on temporal information in order to perform accurately. For example, a question-answering system that is asked ``Will it rain next week?'' must be able to locate itself temporally such that it knows what the current time is, the relation between that time and the queried time, and whether raining events have been forecast during the period that it defines using the phrase `next week'.
\newpage
\section{Relevant Literature}\label{sec:litreview}
In this chapter, the existing literature related to the major topics of the thesis are reviewed and analysed. \nb{Write more here later.} A gap is identified, which the remainder of this work seeks to fill.
\subsection{Times and Events}\label{sub:timesevents}
The concept of time has fascinated researchers for millenia, and as such, a great deal of work exists on the topic. What follows here focuses on the formal study of temporality in language.
\subsubsection{Allen's Interval Algebra}\label{ssub:allen}
% The seminal work of James F. Allen's \textit{Maintaining Knowledge about Temporal Intervals} has pushed the field since its publication in  \citeyear{allen1983maintaining}, and indeed the framework he described drives some of the design decisions in the present work.
Much of the present work draws its roots in James F. \citeauthor{allen1983maintaining}'s work \textit{Maintaining Knowledge about Temporal Intervals} \citeyearpar{allen1983maintaining}, both directly and indirectly via the many systems which have built upon it in the years since it was first published. In this seminal paper, Allen described a framework for the use of temporal intervals as primitives to represent events and time periods, as opposed to points on the real line. There are four key criteria given as being of primary importance in guiding the design of this framework \citeyearpar[p. 833]{allen1983maintaining}, all of which also feature in the system of string-based event representation described in the present work (see \cref{sub:strings}):
\begin{enumerate}\label{other:allen-motivations}
\onehalfspacing
\item The representation should allow for the fact that much temporal information is relative rather than precise.
\item Uncertainty of information should be allowed for, such as when the precise relation between two times is unknown (though constraints on the relation may exist).
\item The granularity of reasoning should be flexible---that is, capable of dealing with years and seconds in the same manner.
\item Reasoning should assume that states will persist unless there is some evidence to the contrary. In other words, \textit{change} is the marker of progression of time. 
\end{enumerate}
Part of the motivation for using intervals rather than points is stated as follows:

{
\onehalfspacing
\begin{quotation}
\noindent
``There seems to be a strong intuition that, given an event, we can always `turn up the magnification' and look at its structure. ... Since the only times we consider will be times of events, it appears that we can always decompose times into subparts. Thus the formal notion of a time point, which would not be decomposable, is not useful.'' \citep[p. 834]{allen1983maintaining}
\end{quotation}
}
\noindent
For example, taking just the event \textit{Going-Home} from the sentence ``I went home after work last night'', it could be conceptually broken down into sub-events that make up parts of the whole: \textit{Leaving-Work}, \textit{Commuting}, and \textit{Arriving-Home}. Each of these could be again further subdivided, repeatedly, as desired. However, if an event is given a specified time, such as ``I went home at midnight'', this seems to be a little trickier to deal with---`last night' is plainly referring to an interval of time during which the \textit{Going-Home} event occurs, while `midnight' appears to be an instantaneous moment of time. Nevertheless, the \textit{Going-Home} event can still be subdivided in the same manner, which shows that even times that appear to intuitively points may be thought of as intervals as well.

Allen makes further argument for the use of intervals as primitive, disallowing zero-width time points, with an illuminating example involving a lightbulb being switched on: there must be an interval of time when the light was off, followed by an interval when it was on, but whether these intervals are open or closed presents another issue. If both intervals are open, then there is some point of time between the two when the light is neither on nor off; however, if both are closed, then there is some time point when the light is both on and off. This fault can be resolved by having the intervals be open at one end and closed at the other, although Allen calls this an artificial solution which ``emphasizes that a model of time based on points on the real line does not correspond to our intuitive notion of time'' \citeyearpar[p. 834]{allen1983maintaining}.

However, this argument that temporal intervals are counter-intuitive if they are open on one end and closed at the other has not been universally accepted---for example, the event calculus \citep{Kowalski1986,Miller1999,Mueller2008} describes a predicate \textit{Initiates}$(e, f, t)$ which states that some event $e$ occurs at a timepoint $t$, and the temporal proposition $f$ is true after $t$. In the lightbulb example, this is equivalent to using intervals which are open on the lower end and closed at the higher, which can be interpreted as meaning the light is not on at the initial moment of switching it on, but is afterwards. Similarly, \citet{Fernando2018} describes using an open left border and closed right border as a means for event representation.

Allen goes on to formally present and label the thirteen possible relations which may exist between two temporal intervals, which can be thought of as six invertible relations (before, meets, overlaps, during, starts, finishes) and one symmetric (equals). \Cref{fig:allens-pictorial}, reproduced from \citet[p. 835, Figure 2]{allen1983maintaining} shows these relations along with the symbols typically used to abbreviate them, and examples of some events represented graphically as strings of `X' and `Y' demonstrating the relations. These are often referred to as Allen's interval relations, or simply the \textit{Allen relations}.
\begin{center}
	\begin{tabular}{| c c c c |}
		\hline
		\textbf{Relation} & \textbf{Symbol} & \textbf{Symbol for Inverse}  & \textbf{Pictoral Example} \\
		X \textit{before} Y & \textless & \textgreater & \verb|XXX YYY| \\
		X \textit{equal} Y & = & = & \verb|XXX| \\[-15pt]
		& & & \verb|YYY| \\
		X \textit{meets} Y & m & mi & \verb|XXXYYY| \\
		X \textit{overlaps} Y & o & oi & \verb|XXX | \\[-15pt]
		& & & \verb| YYY| \\
		X \textit{during} Y & d & di & \verb| XXX  | \\[-15pt]
		& & & \verb|YYYYYY| \\
		X \textit{starts} Y & s & si & \verb|XXX  | \\[-15pt]
		& & & \verb|YYYYY| \\
		X \textit{finishes} Y & f & fi & \verb|  XXX| \\[-15pt]
		& & & \verb|YYYYY| \\
		\hline
	\end{tabular}
	\captionof{figure}{Allen interval relations \citep[p. 835, Figure 2]{allen1983maintaining}.}
	\label{fig:allens-pictorial}
\end{center}
These relations form a cornerstone both within this work (see, for instance, \Cref{tab:allen-rels-strings}) and elsewhere---they are a fundamental part of the specification of ISO-TimeML \citep{pustejovsky2010iso} (see \cref{ssub:timeml}), the international standard markup language for temporal annotation, where they are used, as one might expect\footnote{Albeit with slightly different labels and omitting \textit{overlaps}---see \Cref{fig:tlink-allen-translation}, p. \pageref{fig:tlink-allen-translation}.}, as the possible relation types between a pair of events tagged elsewhere in a document. \citet{Freksa1992} also describes a larger set of \textit{semi-interval} relations in terms of disjunctions of the Allen relations---see also \cref{ssub:granularity} and \cref{ssub:incomplete}.

The framework proposed by Allen uses a directed graph as its basis, in which the nodes represent intervals, and the arcs are labelled according to the relation (or, in the case of uncertainty, relations---see \Cref{tab:allen-trans-table}) between the intervals. It is assumed that complete information about the relations in the network is maintained---that is, there are no nodes without an edge between them, as the transitivities between the various relations are to be computed as necessary---for instance, on the addition of a new interval into the network.

To illustrate, given a graph representing some pair of events $X$ and $Y$ such that $X$ is \textit{before} $Y$ (\Cref{fig:simple-transitivity} (a)), a third node representing the event $Z$ is added, with the additional information that $Y$ occurs \textit{before} $Z$ (\Cref{fig:simple-transitivity} (b)). The relation between $X$ and $Z$ can then be calculated due to the transitivity rules which apply to the Allen relations---in this case, $X$ \textless{} $Y$ and $Y$ \textless{} $Z$ results in $X$ \textless{} $Z$ (\Cref{fig:simple-transitivity} (c))---see also \Cref{tab:allen-trans-table}.
\begin{center}
	\begin{figure}[h!]
		\includegraphics[width=\textwidth]{images/simple-transitivity}
		\captionof{figure}{Simple graph network showing the computation of a transitivity.}
		\label{fig:simple-transitivity}
	\end{figure}
\end{center}
For $N$ nodes (intervals) there are $\frac{N^2-N}{2}$ edges if all new nodes are connected to all existant nodes when they are created. While in \Cref{fig:simple-transitivity} there is one label per edge, there may be as many as $13 \cdot \frac{N^2-N}{2}$ labels per edge if there is no knowledge about any relations. In \Cref{fig:complex-transitivity}, it can be seen that a pair of single-label edges may result in a multiple-label edge by transitivity: the relations $X$ o $Y$ and $Y$ o $Z$ results in the disjunction of $X$ \textless $Z$, $X$ o $Z$, and $X$ m $Z$. Without further data, there is no way to definitively constrain this disjunction to a single relation. If there is a single label for every edge in the graph, this is known as \textit{temporal closure}---as discourse often naturally features incomplete or vague temporal information, calculating the temporal closure is, in general, a difficult goal to achieve, though it is desirable in order to create a representation that is both complete and consistent \citep{Verhagen2005}.
\begin{center}
	\begin{figure}[h!]
		\includegraphics[width=\textwidth]{images/complex-transitivity}
		\captionof{figure}{Computation of a transitivity with a multiple-label result.}
		\label{fig:complex-transitivity}
	\end{figure}
\end{center}
\citet[p. 836]{allen1983maintaining} gives a $12 \times 12$\footnote{The equals relation is ommitted, since it is reflexive, symmetric, and transitive, meaning for any relation $\bigcdot$, $((A = B \land B \bigcdot C) \lor (A \bigcdot B \land B = C)) \Longrightarrow A \bigcdot C$.} transitivity table---a $6 \times 6$ fragment of which is reproduced below in \Cref{tab:allen-trans-table}---which provides the relation or relations which may exist between two intervals $A$ and $C$, given the relations $\bigcdot$ between $A$ and $B$, and $\bigcdot'$ between $B$ and $C$. Not shown in \Cref{tab:allen-trans-table} are relation pairs such as $A$ before $B$ and $B$ after $C$---the disjunction of relations between $A$ and $C$ in this case contains all 13 possibilities (in Allen's table, labelled as `no info'), indicating that there are no constraints on the relation between $A$ and $C$. A smaller fragment of this table which shows these transitivities interpreted as superpositions\footnote{See \cref{ssub:operations} for details of superposition.} of strings which represent the various Allen relations is given in \citet[p. 130]{woods2017towards}. One advantage of the string-based approach comes from the fact that a single string may be used to represent an arbitrary (finite) number of events, and all of the relations between them, at once---thus transitivities between, for example, four intervals can be represented as easily as three.% \nb{Table of this somewhere?}

\begin{center}
	\onehalfspacing
	\begin{tabular}{|c|c|c|c|c|c|c|}
		\hline
		\diagbox{$A \bigcdot B$}{$B \bigcdot' C$} & \textbf{\textless{}} & \textbf{d} & \textbf{o} & \textbf{m} & \textbf{s} & \textbf{f}\\
		\hline
		before \textbf{(\textless{})} & \textless{} & \textless{} o m d s & \textless{} & \textless{} & \textless{} & \textless{} o m d s\\
		\hline
		during \textbf{(d)} & \textless{} & d & \textless{} o m d s & \textless{} & d & d\\
		\hline
		overlaps \textbf{(o)} & \textless{} & o d s & \textless{} o m & \textless{} & o & o d s\\
		\hline
		meets \textbf{(m)} & \textless{} & o d s & \textless{} & \textless{} & m & o d s\\
		\hline
		starts \textbf{(s)} & \textless{} & d & \textless{} o m & \textless{} & s & d\\
		\hline
		finishes \textbf{(f)} & \textless{} & d & o d s & m & d & f\\
		\hline
	\end{tabular}
	\captionof{table}{Fragment of Allen's transitivity table \citeyearpar[p. 836, Figure 4]{allen1983maintaining}.}
	\label{tab:allen-trans-table}
\end{center}
\label{other:allen-algo}For the propagation of constraints upon the addition of new information to the network, \citet[p. 835]{allen1983maintaining} also details an algorithm using the transitivity table to update all arcs in the network as necessary. While Allen does concede that there are some issues with this algorithm---specifically mentioning the space requirement, which is somewhat high at $O(N^2)$ space for $N$ temporal intervals, and the fact that the algorithm doesn't guarantee consistency in larger than three-node networks\footnote{See \cref{sub:applications}, p. \pageref{ex:str-inconsistency} for how this limit may be improved upon by using strings and superposition to automatically reject inconsistencies.}, and \citet[p. 219]{Verhagen2005} also notes an $O(N^3)$ time complexity---it does give an upper bound of $13 \times \frac{(N - 1)(N - 2)}{2}$ to the number of modifications that can be made to the network, regardless of the number of constraints added. Additionally, ``the average amount of work for each addition is essentially linear (i.e., $N$ additions take $O(N^2)$ time; one addition on average takes $O(N)$ time)'' \citep[p. 837]{allen1983maintaining}. This is still quite high, given that it is not unusual for a given text, annotated documents of the TimeBank \citep{pustejovsky2003timebank,pustejovsky2006timebank} corpus for instance, to feature fifty intervals or more, and some are in the hundreds \citep[p. 213]{Verhagen2005}.

To tackle the space requirements while maintaining as much inferential power as possible, \citet[p. 838]{allen1983maintaining} also describes a method for grouping clusters of intervals which---in regards to the relations between them---are fully computed, termed ``reference intervals''. This is due to the fact that each interval $I_i$ in the cluster $\{I_1, I_2, ..., I_n\}$ will reference one or more new intervals $R_m$ (where $m$ is an index on the number of reference intervals), and these reference intervals will be connected in the graph, which can be used to find relations between nodes which are not directly connected. Since these reference intervals may be treated as normal intervals, they may themselves be grouped into a cluster, and thus hierarchies of intervals may arise.% These are not currently implemented in the present work, but they may be useful as a way to address documents with high numbers of intervals.

Allen's work was and remains highly influential in the field of temporal reasoning and annotation, thanks to an approach which is both intuitive and straight-forward, as well as easy to implement in various applications. One such proponent of interval relations has been TimeML, a markup language designed to ``capture the richness of temporal and event related information in language'' \citep[p. 123]{Pustejovsky2005}---see \cref{ssub:timeml}.


\subsubsection{Tense and Aspect}\label{ssub:tenseaspect}
Reichenbach's (\citeyear{reichenbach1947elements}) theory of verbal tense and aspect allows for the temporal placement of an event time in relation to a speech time and a reference time. The relative orderings of these three times gives rise to the categorisations of tense and aspect in English and other languages. This framework has been widely accepted and adopted, and has found empirical validation via TimeML \citep{timeml2005timeml} and the TimeBank corpus \citep{pustejovsky2003timebank} in \citet{Derczynski2013}, which finds that using \citet{Freksa1992}'s semi-interval relations leads to ``tense appropriately constrain[ing] the types of temporal relations that can hold between pairs of events described by verbs''.
%While verbs are not the only expression of events, these categorisations can convey enough information to temporally locate that event

% \nb{Bring up TEA here somewhere. Also worth mentioning Vendler and Dowty in relation to lexically telic/atelic verbs, which inform the stative-based approach of the strings and thus block compression. Tim mentioned that this is a difference from Schwer's S-Words.}
% Reichenbach, Vendler, TEA
% Aspect hypothesis of dowty - build from statives - this is behind block compression/destuttering
% as difference for Schwer s-words

\citet[p. 72]{reichenbach1947elements} initially presents six tense/aspect combinations of English as arrangements of three temporal points, the `point of the event', the `point of reference', and the `point of speech', which may coincide. These six are: the Simple Past, Past Perfect, Present, Present Perfect, Simple Future, and Future Perfect. However, it is noted that the `Simple Future' may have two possible interpretations in English, and the arrangement originally presented under that name---where the points of speech and reference are equal and precede the point of the event---is later (p. 77) renamed as the `Posterior Present', while the second interpretation becomes the arrangement associated with the term `Simple Future'---see \Cref{fig:reichenbach-arrangements}.



% This is reproduced in \Cref{fig:reichenbach-arrangements} below, with $E$ for the event time, $R$ for the reference time, and $S$ for the speech time:

% \citet[p. 77]{Derczynski2013}: ``PERFECTIVE PROGRESSIVE aspect was converted to PERFECTIVE; this feature value accounts for 20 of 5974 verb events, or 0.34%''
\begin{center}
	\begin{tabular}{|l l c|}
		\hline
		\textbf{Tense} & \textbf{Example} & \textbf{E, R, S}\\
		\hline
		Simple Past & ``I bought the book'' & $(E = R) < S$\\
		Past Perfect & ``I had bought the book'' & $E < R < S$\\
		Simple Present & ``I buy the book'' & $E = R = S$\\
		Present Perfect & ``I have bought the book'' & $E < (R = S)$\\
		Simple Future & ``I will buy the book'' & $S < (R = E)$\\
		Future Perfect & ``I will have bought the book'' & $S < E < R$\\
		\hline
	\end{tabular}
	\captionof{figure}{Arrangements of the points $E$, $R$, and $S$, from \citet{reichenbach1947elements}.}
	\label{fig:reichenbach-arrangements}
\end{center}
As can be seen from these arrangements, the tense of a sentence being in the Past, Present, or Future tense correlates to the ordering of the reference time $R$ and speech time $S$: whether the reference time $R$ appears before, equal to, or after the speech time $S$, respectively. The perfective aspect---called the `Anterior' tenses in \citet[p. 77]{reichenbach1947elements}---is then created by placing the event time $E$ before $R$, where it is equal to $R$ in the non-perfective aspect.

While \citet{reichenbach1947elements} (mostly) treats the time of event, speech and reference as temporal points, there is so far no issue with assuming that they may instead be treated as temporal intervals, \`{a} la \citet{allen1983maintaining} and TimeML \citep{timeml2005timeml,sauri2006timeml}. However, in order to treat the Progressive aspect which English may use, the so-called `point of the event' must in fact be taken as an interval rather than an instantaneous point, indicating the durative nature of the Progressive. If the event time was already considered to be an interval, then instead the set of relations must change so as to include the `during' relation from Allen's set---see \Cref{fig:allens-pictorial}, p. \pageref{fig:allens-pictorial}. Thus the Progressive aspect can be conceptualised by the reference time $R$ occurring during the event time $E$, as in \Cref{fig:reichenbach-progressive}.
\begin{center}
	\begin{tabular}{|l l c|}
		\hline
		\textbf{Tense} & \textbf{Example} & \textbf{E, R, S}\\
		\hline
		Past Progressive & ``I was buying the book'' & $R ~d~ E,~ R < S$\\
		Present Progressive & ``I am buying the book'' & $R ~d~ E,~ R = S$\\
		Future Progressive & ``I will be buying the book'' & $R ~d~ E,~ S < R$\\
		\hline
	\end{tabular}
	\captionof{figure}{Relations of $E$, $R$, and $S$ for the Progressive aspect.}
	\label{fig:reichenbach-progressive}
\end{center}
However, if all of $E$, $R$, and $S$ are intervals, then Allen's interval relations apply to all pairings. Where before there was an equality relation between two of the points, Allen's equality will apply, but if one point preceded another, then in fact either of the `before' or `meets' relations may hold. Additionally, taking the Past Progressive as an example gives $R ~d~ E$ and $R < S$---assuming for the moment that $R$ is before $S$ rather than meeting it, then by the rules of transitivity for Allen relations, the relation between $E$ and $S$ may be any of `before', `meets', `contains', `finished by', or `overlaps'. The fact that these disjunctions arise leads \citet{Derczynski2013} to use \citet{Freksa1992}'s set of 31 semi-interval relations---a superset of Allen's interval relations---in validating the Reichenbachian framework within the TimeBank \citep{pustejovsky2006timebank} corpus. Semi-intervals arise by considering an interval in terms of its beginning and ending, but treating those as intervals in and of themselves, and in this way a set of relations are defined between the beginnings and endings of events. For instance, if the Freksa relation `older' holds between some pair of events $a$ and $b$, then the beginning of $a$, $\alpha(a)$, occurs before the beginning of $b$, $\alpha(b)$, while the order of the endings of the events are left unspecified. Each Freksa relation may also be described in terms of a disjunction of Allen relations---in fact, the `older' relation corresponds to the same disjunction of `before', `meets', `contains', `finished by', or `overlaps' which appear between a verb's event time $E$ and its speech time $S$ in the Past Progressive. The Freksa relations are discussed in more detail in \cref{ssub:incomplete}, but they are relevant here due to the disjunctions which come out of treating event, reference, and speech times as events, and leveraging shared reference times may lead to more of these disjunctions arriving when attempting to derive temporal relations between verbal events---see \Cref{fig:tense-aspect-freksa}.

By using what \citet[p. 74]{reichenbach1947elements} refers to as the ``permanence of the reference point'', events in an utterance may be temporally ordered relative to each other. This states that, though events may be in different clauses, the fact that a speaker adjusts the use of tense and aspect for the events relative to each other shows that they must share a reference time. In the following example \cref{ex:permanent-ref-pt}, the two events are in boldface.
\begin{itemize}
	\item[\ipp\label{ex:permanent-ref-pt}] ``I had \textbf{left} her when I \textbf{bought} the book.''
\end{itemize}
Since the two events are in the same utterance, they have the same speech time, and by permanence of reference point, their reference times are also identical. Now, given that `left' is in the Past Perfect, while `bought' is in the Simple Past, the constraints from \Cref{fig:reichenbach-arrangements} can be used to order the two events---using points rather than intervals for the sake of keeping the example simple---as follows in \cref{ex:permanent-ref}:

\begin{subequations}\label{ex:permanent-ref}
	\onehalfspacing
	\begin{align}
		S_{leave} = S_{buy},~ R_{leave} = R_{buy}\\
		E_{leave} ~<~ R_{leave} ~<~ S_{leave}\\
		E_{buy} = R_{buy} ~<~ S_{buy}\\
		\therefore~~~ E_{leave} ~<~ E_{buy}
	\end{align}
\end{subequations}
This gives that the event time of `left' $E_{leave}$ occurs before the event time of `bought' $E_{buy}$, and thus for any pair of verbal events $a$ and $b$ which share a speech and reference time and the verb denoting $a$ is in the Past Perfect while the verb denoting $b$ is in the Simple Past, we can deduce that $a$ precedes $b$, as temporal points. While this principle still holds when $E$, $R$, and $S$ are treated as intervals instead of points, it does become a little more complex as the size of the possible relation set between a pair of intervals is over four times larger than the possible relation set for points, and as mentioned above, it is possible for disjunctions of relations to arise. \citet{Derczynski2013} give a table with all of the combinations of tense \{\texttt{PAST}, \texttt{PRESENT}, \texttt{FUTURE}\} and aspect\footnote{The \texttt{PERFECTIVE\_PROGRESSIVE} aspect is also possible---for example in ``I had been walking''---however, it is omitted for simplicity's sake due to featuring in less than 1\% of verb events in the TimeBank corpus \citep[p. 77]{Derczynski2013}.} \{\texttt{NONE}, \texttt{PROGRESSIVE}, \texttt{PERFECTIVE}\}---the aspect value of \texttt{NONE} denotes the Simple form of that tense---where each disjunction has been labelled with its associated Freksa relation, displayed in \Cref{fig:tense-aspect-freksa}.
\begin{center}
	\begin{figure}[h!]
		\includegraphics[width=\textwidth]{images/tense-aspect-freksa}
		\captionof{figure}{Tense-aspect pairs and the \citet{Freksa1992} relations they may suggest\\\citep[taken from][p. 80, Table 5]{Derczynski2013}.}\label{fig:tense-aspect-freksa}
	\end{figure}
\end{center}
Care must be given, though, to note that the rule of a permanent reference time only applies to verbs within the same \textit{temporal context} \citep{hornstein1990time,Derczynski2013}---verbs appearing in reported speech, for instance, are in a separate context to the verbs which introduce it. For example, in \cref{ex:reported-speech}, the events pointed to by `put' and `said' share their speech and reference times, and thus their temporal context is the same. However, `going' is part of a separate statement, enclosed by the quotation marks, and having a speech time that is equal to the event time of `said'.

\begin{itemize}
	\item[\ipp\label{ex:reported-speech}] He \textbf{put} his head in and \textbf{said} ``I'm \textbf{going} to the shop''.
\end{itemize}
In general, if a pair of events that are to be temporally related are in separate temporal contexts ``Reichenbach's framework may not directly apply, and the pair should not be further analysed'' \citep[p. 75]{Derczynski2013}. However, the temporal context can be difficult to determine automatically from a text, and the TimeML schema doesn't provide an explicit way to annotate it. One way to model a shared context between some pair of events is to look at the proximity of their appearance in the text, assuming events which are more textually distant than an adjacent sentence are unlikely to share the same context, and additionally checking if the events share the same tense---that is, the same speech and reference time. Using this model on the TimeBank 1.2 \citep{pustejovsky2006timebank} corpus, \citet[p. 80]{Derczynski2013} validated the Reichenbachian framework of tense and aspect by extracting pairs of verbal events from the \verb|<TLINK>| tags and using the tense and aspect of the verb pairs to derive one of the disjunctions of relations given in \Cref{fig:tense-aspect-freksa}. This derived disjunction contained as an element the relation that was actually marked up in the \verb|<TLINK>| tag in 67.8\% of cases\footnote{This excludes the prediction of the `all' or `unknown' relation, which is a disjunction of all possible relations. Including this relation is not useful, but when doing so the accuracy was found to be 91.9\%.}. It is noted that, taking into consideration that the temporal context model is somewhat crude, the amount of inaccuracy using this model (32.2\%) is comparable to the inter-annotator disagreement for \verb|<TLINK>| relation type labels (0.23) in the corpus: ``The fact that temporal context is derived from models and not explicit gold-standard annotation is also likely a significant source of noise in agreement.'' \citep[p. 80]{Derczynski2013}.

The next section takes a deeper look at the TimeBank corpus and the annotation schema with which it is marked up, TimeML.

\subsection{Temporal Annotation}\label{sub:annotation}
Ideally, humans who use artificially intelligent systems for question-answering, scheduling, or other applications which require some level of reasoning about temporal data won't ever have to consider the annotation which drives it---except, of course, in the case when annotation is itself the goal of using the system. Computer systems have no inherent understanding of natural language, and rely on the samples they have been provided with in order to give a facsimile of the communication abilities that humans have. The better the quality of those samples, and the more information embedded within them, the better the system can interpret human language, and in turn generate interpretable output.

The annotation of a document allows for information about the text to be expressed in such a way as to make the implicit explicit---revealing a layer of semantics which a human may be able to infer, but an artificial agent would not. There are many ways to mark up a text, and many concepts which may warrant marking up, but in terms of temporal data, the most prudent schema to follow is that of TimeML \citep{timeml2005timeml} and its successor, which has become the ISO (International Organization for Standardization) standard for temporal annotation: ISO-TimeML \citep{ISO24617-1}, a part of the ISO Semantic Annotation Framework---see \citet{Bunt2020}.

% Derczinski
TimeML has found considerable success in terms of widespread adoption as a means of marking up text with temporal information, at least in some part thanks to the release of TimeBank \citep{pustejovsky2003timebank}, a corpus of newswire articles\footnote{The original version of TimeBank contained 300 articles, while the latest version, TimeBank 1.2, contains 183.} which were manually annotated using the TimeML schema. TimeBank frequently appears in discussion about TimeML since, in terms of quality, manual annotation is still seen as the `gold standard' and superior to automatically generated markup, although machine-created annotation has seen much effort and improvement in recent years \citep{mani2006machine,uzzaman2013semeval,reimers2016temporal}.

The availability of appropriate tooling for the creation of human-driven annotation is crucial in ensuring that accurate and consistent results are produced. The first edition of the TimeBank corpus was annotated using the Alembic Workbench tool \citep{day1997mixed}, which was useful for the non-relational aspects, but impractical for the highly relational nature of the temporal data featured in TimeML annotations \citep{verhagen2005TBOX}. A number of other tools have been developed since then which aim to provide visual feedback and assistance to an annotator, including using directed graphs---as in \citet{allen1983maintaining}---and timeline-like depictions, of which the tooling described in \cref{sec:implementation} falls under the second category.

\subsubsection{TimeML and TimeBank}\label{ssub:timeml}
The initial goal for creating the TimeML language came from a desire to improve applications\allowbreak{}---such as question-answering systems---by means of event recognition, and giving each recognised event an explicit temporal location. This latter point is motivated due to the fact that a large proportion of temporal information in discourse rely on implicit or vague temporal expressions. This relates back to the first principle which \citet{allen1983maintaining} mentioned\footnote{See p. \pageref{other:allen-motivations}.} as influencing the design of the interval algebra framework: that it is not generally intuitive to always think of or refer to time in terms of explicit or precise time points. Instead we tend to use relative expressions, such as the boldface text in the following utterances:
\begin{itemize}
	\item[\ipp] ``I didn't go to work \textbf{last Monday}.''
	\item[\ipp] ``I was sick \textbf{the week before}.''
\end{itemize}
Speakers will rely on their listeners being able to use context to interpret what these temporal expressions refer to.

TimeML aimed, in part, to give implicit and relative temporal expressions an explicit anchoring, in order to assist with intelligent systems' temporal awareness. \citet[p. 125, (1a.)]{Pustejovsky2005} wanted to enable question-answering systems to be able to answer questions such as
\begin{itemize}
	\item[\ipp] ``Is Schr\"{o}der currently German chancellor?''
\end{itemize}
as capably as a human could after reading an appropriately relevant news article. A number of issues are raised that ought to be addressed within a system that can understand and answer questions similar to this one. Potentially problematic example queries can range from simple questions about the date of a specific event:
\begin{itemize}
	\item[\ipp] ``When did the USA first declare independence from the UK?''
\end{itemize}
to questions about non-unique events:
\begin{itemize}
	\item[\ipp] ``How long does it take to drive from Dublin to Cork?''
\end{itemize}
and questions where the system must perform some level of inference in order to derive the answer, possibly returning information that is not temporal in and of itself, but requires such data to find the correct solution:
\begin{itemize}
	\item[\ipp] ``Who was the last president of France?''
\end{itemize}
\citet[p. 132]{Pustejovsky2005} further discusses the kinds of temporal information that might be needed, and how to go about representing it in a useful way. Two tasks are deemed essential: the ability to place events on some timeline, and the ability to determine the relative order of any pair of events. These tasks are termed event \textit{anchoring} and \textit{ordering}, respectively, and these also form a core motivation for the framework described in \cref{sec:fst}, which uses strings to simultaneously represent the anchoring and ordering of events.  %a core part of the present work. Beyond these two tasks, the ability to identify and extract events and times from texts is an important task.

Events in TimeML are ``referred to by finite clauses, nonfinite clauses, nominalizations, event-referring nouns, adjectives, and even some kinds of adverbial clauses'' \citep[p. 133]{Pustejovsky2005}. Systems must also be aware of the possibility of negated and modal events, such as:
\begin{itemize}
	\item[\ipp] ``Ireland did \textbf{not} \textit{make it} to the World Cup this year.''
	\item[\ipp] ``The exhibition \textbf{might} \textit{create} new opportunities for the museum.''
\end{itemize}
The events (italicised) in these sentences should not be treated as if they actually occurred. Additionally, care should be taken to distinguish separate events in the representation which are referred to together in the text \citep[p. 134, (32a.)]{Pustejovsky2005}:
\begin{itemize}
	\item[\ipp\label{other:pustejovsky-type-instance}] ``James \textit{taught 3 times} on Tuesday.''
\end{itemize}
This leads to a distinction between types of events and instances of events, where an instance may appear in the representation in place of a type, \textit{cf.} \cref{ex:judder}, p. \pageref{ex:judder}.

Times, on the other hand, generally take the form of adverbial or prepositional phrases in English, such as `next week', `yesterday', or `10th of August'. For TimeML, these expressions must be normalised in order to anchor events to times on a timeline---converted to some machine-readable form, possibly an integer or real number, depending on the context. It is also important to know the time of utterance (or document creation time, for a text) in order to correctly normalise expressions such as `today' or `last Monday', as these terms refer to a time which is relative to some other point, often the time of utterance or document creation time. Some expressions, such as `recently' cannot be determinately linked to the timeline, due to their inherently vague nature, yet may still be ordered relative to other time points, and thus should still undergo normalisation. The last two kinds of time expressions mentioned in \citet{Pustejovsky2005} are durations and sets of times. Durations, such as `five hours', may or may not be anchored to times or events (boldface text):
\begin{itemize}
	\item[\ipp] ``You'll have it for \textit{a month} from \textbf{Monday}.''
	\item[\ipp] ``She submitted \textit{two weeks} before \textbf{the deadline}.''
\end{itemize}
Sets of times are generally used to place recurring events on the timeline:
\begin{itemize}
	\item[\ipp] ``He visited \textit{every week}.''
	\item[\ipp] ``They took the meds \textit{twice a day}.''
\end{itemize}
% 
% While the current work deals directly with data that is already annotated with markers for events and times, it is worth keeping this instructive material on extraction of temporal data in mind. The TimeBank corpus \citep{pustejovsky2006timebank} is a large collection of documents (in the domain of news articles) manually annotated with TimeML, and as such contains a number of human errors (e.g. inconsistencies in relation-labelling). It is also not as completely labelled as it could be, and has also not been updated in several years. It is conceivable, then, that an alternative source for data may at some point become desirable, either due to the release of a more attractive corpus, or by creating an entirely new corpus, following the same general principles as in TimeBank.
% 
In order for TimeML to represent the desired ordering and anchoring, a set of relations for times and events is required, and Allen's interval relations are selected as a strong basis for these, as reasoning over them is ``well-understood'' \citep[p. 138]{Pustejovsky2005}. However, it is noted that not all of Allen's relations are equally well represented in texts of the English language. In particular, the \textit{overlaps} relation is ``difficult to find instantiated in natural language text'', and thus may be considered unnecessary. In fact, this relation and its inverse are omitted from the final set of temporal relations used in TimeML---see \Cref{fig:tlink-allen-translation}. Nevertheless, since these relations may arise through transitivity and it is not immediately problematic to do so, they are reincluded in \cref{sec:implementation}.

The syntax of TimeML\footnote{For the full specification of TimeML (version 1.2.1), see \url{http://www.timeml.org} \citep{timeml2005timeml}, and annotation guidelines in \citet{sauri2006timeml}.} % \url{https://www.cs.brandeis.edu/~cs112/cs112-2004/annPS/TimeML12wp.htm} \citeneeded{}
uses the following types of tags to capture the various kinds of information: \verb|<TIMEX3>| for marking up time expressions, \verb|<EVENT>| for events and \verb|<MAKEINSTANCE>| for event instances, \verb|<SIGNAL>| for functional words (such as `at', `from', `when', etc.), and three types of linking tags for relations between the other tags: \verb|<SLINK>| captures subordinating relations involving evidentiality, modality, and factuality; \verb|<ALINK>| captures aspectual links; and \verb|<TLINK>| captures temporal relationships between events and times.

\Cref{ssub:tlinks,sec:implementation} focus primarily on this last tag type, as it is here that Allen's interval relations are represented, albeit under slightly different nomenclature, with some duplication, and omitting the overlapping relations as mentioned above---see \Cref{fig:tlink-allen-translation}. Each of TimeML's tags take a number of attributes which help to flesh out the representation of the information, and for a \verb|<TLINK>| these are: either a \texttt{timeID} or \texttt{event\allowbreak{}InstanceID}, which refers to some temporal expression in the text; either a \texttt{relatedTo\allowbreak{}Time} or \texttt{relatedTo\allowbreak{}Event\allowbreak{}Instance}, which will refer to another temporal expression; and a \texttt{relType}, which will specify the relation from the first attribute to the second.
\begin{center}
	\setstretch{1.15}
	\begin{tabular}[h!]{|c c|}
		\hline
		\textbf{TLINK} & \textbf{Allen}\\
		\hline
		\texttt{SIMULTANEOUS} & equal \textbf{(=)}\\
		\texttt{IDENTITY} & equal \textbf{(=)}\\
		\texttt{BEFORE} & before \textbf{(\textless{})}\\
		\texttt{AFTER} & after \textbf{(\textgreater{})}\\
		\texttt{IBEFORE} & meets \textbf{(m)}\\
		\texttt{IAFTER} & met by \textbf{(mi)}\\
		\texttt{INCLUDES} & contains \textbf{(di)}\\
		\texttt{IS\_INCLUDED} & during \textbf{(d)}\\
		\texttt{DURING} & during \textbf{(d)}\\
		\texttt{DURING\_INV} & contains \textbf{(di)}\\
		\texttt{BEGINS} & starts \textbf{(s)}\\
		\texttt{BEGUN\_BY} & started by \textbf{(si)}\\
		\texttt{ENDS} & finshes \textbf{(f)}\\
		\texttt{ENDED\_BY} & finished by \textbf{(fi)}\\
		\hline
	\end{tabular}
	\captionof{figure}{Possible values of a TLINK's relType attribute and their Allen relation counterparts.}\label{fig:tlink-allen-translation}
\end{center}
The rationale behind distinguishing \texttt{INCLUDES}/\texttt{IS\_INCLUDED} and \texttt{DURING\_INV}/\texttt{DURING} is that the former should be used for cases where an event or time is included in another, like in:
\begin{itemize}
	\item[\ipp] ``He \textbf{arrived} there \textbf{last week}.''
\end{itemize}
while the latter should be used specifically for states or events that persist throughout a duration \citep[p. 158]{Pustejovsky2005}, such as in:
\begin{itemize}
	\item[\ipp] ``They \textbf{were Captain} for \textbf{two seasons}.''
\end{itemize}
Below in \Cref{fig:example-timeml-annotation} is a simple example of a completed annotation---derived from the annotation guidelines \citep{sauri2006timeml}---of the sentence:
\begin{itemize}
	\item[\ipp] ``He panicked on Wednesday.''
\end{itemize}
\begin{center}
	\onehalfspacing
	\begin{tabular}[h!]{|l|}
		\hline
		\begin{minipage}{0.75\textwidth}
		\begin{verbatim}

		He
		<EVENT eid="e1" class="OCCURRENCE">
		 panicked
		</EVENT>
		<SIGNAL sid="s1">
		 on
		</SIGNAL>
		<TIMEX3 tid="t1" type="DATE">
		 Wednesday
		</TIMEX3>
		<MAKEINSTANCE eiid="ei1" eventID="e1" pos="VERB"
			tense="PAST" aspect="NONE" polarity="POS" />
		<TLINK eventInstanceID="ei1" relatedToTime="t1"
			signalID="s1" relType="IS_INCLUDED" />

		\end{verbatim}
		\end{minipage}\\
		\hline
	\end{tabular}
	\captionof{figure}{Example TimeML annotation for ``He panicked on Wednesday.''}\label{fig:example-timeml-annotation}
\end{center}
Since the initial publication of \citet{pustejovsky2003timeml}\footnote{In this work, the authors do refer to the TimeML language as being ``developed in the context of a six-month workshop, TERQAS'' during 2002. However, the URL link cited for \textit{Annotation Guideline to TimeML 1.0} (\url{http://time2002.org}) no longer available points to an available resource as of March 2021. The TERQAS (Time and Event Recognition for Question Answering Systems) workshop was held at MITRE Bedford and Brandeis University \citep[p. 647]{pustejovsky2003timebank}.}, TimeML has received numerous improvements, updates, and tweaks. Although a version of the language has been adopted as an ISO standard (ISO 24617-1:2012) for temporal annotation \citep{pustejovsky2010iso}, the most recent publically available version---see \url{http://www.timeml.org}---is TimeML 1.2.1 \citep{timeml2005timeml,sauri2006timeml}, and this is the version which the present work will focus on. The primary reason for this is due to the most recently available edition\footnote{This refers to the English language corpus. There do exist versions of other TimeBank corpora in languages such as French \citep{bittar2011french} and Hindi \citep{goel2020hindi} which use the ISO-TimeML schema.} of the TimeBank corpus---one of the largest available corpora for documents annotated with TimeML---being TimeBank 1.2 \citep{pustejovsky2006timebank}, wherein the annotation was updated in order to bring the corpus in line with the updated specification, which at the time was TimeML 1.2.1.

The TimeBank 1.2 corpus contains 183 texts, featuring news articles from a variety of sources, including broadcast news from ABC, CNN, PRI, and VOA, as well as articles from the Wall Street Journal. The counts for how frequently each tag appears in the corpus are given below in \Cref{fig:timebank-stats}, reproduced from \citet{pustejovsky2006timebank}:
\begin{center}
	\onehalfspacing
	\begin{tabular}[h!]{|l l|}
		\hline
		\textbf{Tag} & \textbf{Count}\\
		\hline
		EVENT & 7,935\\
		MAKEINSTANCE & 7,940\\
		TIMEX3 & 1,414\\
		SIGNAL & 688\\
		ALINK & 265\\
		SLINK & 2,932\\
		TLINK & 6,418\\
		% \hline
		\textit{Total} & 27,592\\
		\hline
	\end{tabular}
	\captionof{figure}{Tag counts for TimeBank 1.2.}\label{fig:timebank-stats}
\end{center}
The TempEval-3 \citep{uzzaman2013semeval} shared task, which aimed to ``advance research on temporal information processing'', also produced a slightly updated version of the TimeBank 1.2 corpus, which is described as having been ``cleaned''. This included adjusting all of the files to be have consistent formatting, to be XML and TimeML schema compatible, as well as adding some missing events, times, and relations \citep[p. 2]{uzzaman2013semeval}.

While this work will focus on TimeML 1.2.1, it is worth outlining some of the most notable alterations which were introduced by ISO-TimeML. One of the largest changes, from a structural point of view, is moving away from in-line annotation, whereby the markup tags are inserted directly into the body of the text. Instead, ISO-TimeML separates the annotation from the main text, using ``stand-off'' annotation, which increases the interoperability of annotation languages by conforming to the general practice of not modifying the text which is being annotated \citep[p. 395]{pustejovsky2010iso}. For example, a sentence like in \cref{ex:inline-tag-sentence} which features an event that appears partway through will be marked up as something similar to \cref{ex:inline-tag-markup} (omitting some attributes) using the older versions of TimeML:
\begin{itemize}
	\item[\ipp\label{ex:inline-tag-sentence}] ``He walked to the shop.''
	\item[\ipp\label{ex:inline-tag-markup}] \verb|He <EVENT eid="e1" ... >walked</EVENT> to the shop.|
\end{itemize}
while ISO-TimeML would be something more like in \cref{ex:standoff-event-seg,ex:standoff-event-tml}, which are each in separate files:
\begin{itemize}
	\item[\ipp\label{ex:standoff-event-seg}] \verb|<seg type="token" xml:id="token1">He</seg>|\\\verb|<seg type="token" xml:id="token2">walked</seg>|\\\verb|<seg type="token" xml:id="token3">to</seg>|\\\verb|<seg type="token" xml:id="token4">the</seg>|\\\verb|<seg type="token" xml:id="token5">shop</seg>|
	\item[\ipp\label{ex:standoff-event-tml}] \verb|<EVENT xml:id="e1" target="#token2" ... />|
\end{itemize}
Additionally, the \verb|<EVENT>| tag is now explicitly used to denote an event instance, and as a result, the \verb|<MAKEINSTANCE>| tag no longer exists. Some of its attributes are shifted to the \verb|<EVENT>| tag, and \verb|<TLINK>| tags in ISO-TimeML use \texttt{eventID} and \texttt{relatedToEvent} instead of \texttt{eventInstanceID} and \texttt{relatedToEventInstance}, respectively.

\label{other:iso-mlink}Finally, ISO-TimeML also takes a proposal from \citep{bunt2010annotating} and provides a new type of link tag, \verb|<MLINK>|, which has an ``inherent relation type of \verb|MEASURE|'' \citep[p. 396]{pustejovsky2010iso}, and is used to better reflect the relationships between durative events and stretches of time---for instance, where an event may have been interrupted and resumed during some period, but is referred to as having taken the entire span of time. The new tag uses a measuring function \citep{bunt1985mass} to interpret the relation, so that period of time that is measured is equal to the sum of all of the spans of time that make it up, whether contiguous or not. For example, in the sentence:
\begin{itemize}
	\item[\ipp] ``It rained for an hour.''
\end{itemize}
it may not have been raining consistently for the full hour---there may have been a period of, say, twenty minutes when it abated---but it is valid to interpret it either as a full, non-stop hour of rain, or as a span of an hour during which it rained. This is what the \verb|<MLINK>| is designed to treat, as in the example \cref{ex:mlink} below (where \textit{P1H} refers to a period of 1 hour):
\begin{itemize}
	\item[\ipp\label{ex:mlink}] \verb|<EVENT xml:id="e1" pred="RAIN" />|\\\verb|<TIMEX3 xml:id="t2" type="DURATION" value="P1H" />|\\\verb|<MLINK eventID="e1" relatedToTime="t2" />|
\end{itemize}
ISO-TimeML introduces a number of other changes over TimeML, such as the alteration and addition of several tag attributes, with the general aim of increasing interoperability, making it easier for other systems or software that might make use of the represented information. However, neither version of the language is intended to stand truly alone, but rather annotations are created for use in other applications. As such, there is no built-in way to visualise the depicted timeline of times and events, which is an intuitive way\footnote{See \cref{ssub:timelines}.} to assist in understanding the anchoring and ordering of the temporal entities. As such---like the string-based framework described in \cref{sec:fst}---other tools have been created which aim to aid in this arena, either simply for visualising a completed annotation, or as means of assisting an annotator in creating the markup.

\subsubsection{Tango and T-BOX}\label{ssub:tango-tbox}
Manual annotation of temporal data in text is not a simple task, requiring a solid understanding of the annotation schema, a strong ability to identify and keep track of multiple times and events, as well as being skilled in interpreting the often vague temporal data that exists in language \citep[pp. 213--214]{Verhagen2005}. Accordingly, a number of tools have been designed with the aim of assisting an annotator in making more correct decisions, either by helping to visualise the temporal structure of the document, or by automatically computing relations or marking inconsistencies as the annotator works. For example, Tango \citep{pustejovsky2003tango} was developed with the aim of improving the annotation of documents with TimeML by allowing users to---quite literally---draw connections between times and events which were displayed graphically, as in \Cref{fig:tango}.
\begin{center}
	\begin{figure}[h]
		\centering
		\includegraphics[width=0.9\textwidth]{images/tango}
	\end{figure}
	\captionof{figure}{Tango's interface \citep[taken from][p. 2, Figure 1]{verhagen2005TBOX}.}
	\label{fig:tango}
\end{center}
The graph here works on the same principle as those described in \cref{ssub:allen}, in that the nodes represent the events and times which are marked up in the TimeML document, while the arcs are labelled with the relation between the intervals. According to \citet[p. 2]{verhagen2005TBOX}, using Tango improved the quality and reliability of the annotations being produced, with a higher number of links being found between the nodes, though he also notes that Tango's main flaw is that it does not allow a user to ``quickly capture the temporal structure of the document'' \citep[p. 2]{verhagen2005TBOX}---that is, interpreting the overall chronology of a text by means of a directed graph is not straightforward for a human user of the technology. Part of the problem, according to \citet{verhagen2005TBOX}, is that the graph labels---representing the \verb|<TLINK>| tags of TimeML---quickly become difficult to read, especially when the document contains a large number of nodes. More troublesome, however, is the fact that there is no inherent semantics to the placement of the nodes; there is no graphical depiction of the temporal ordering, left-to-right or otherwise, that is enforced by the system in a meaningful way.

As a means of addressing some of these issues with the Tango tooling and its predecessor, the Alembic Workbench \citep{day1997mixed}, \citet{verhagen2005TBOX} presents the T-BOX framework, intended to be used in a complementary fashion alongside the table-based and graph-based depictions in the existant tools. T-BOX is based around the core idea that ``relative placement of two events or times is completely determined by the temporal relations between them'' (p. 2). The image shown in \Cref{fig:tbox} represents the same data as in \Cref{fig:tango}, placing each event and time into boxes (called T-BOXes), and using arrows, stacking, and box inclusion to represent the various relations between the temporal expressions that may be derived from \verb|<TLINK>| tags, under the assumption that events and times are intervalic, as described in \cref{ssub:allen}.

The rules governing placement of boxes are \citep[pp. 3--5]{verhagen2005TBOX}:
\begin{itemize}
	\item An event which occurs \textit{before} another is placed to the left of it, with an arrow leading from the one to the other, or a line ending in a dot if the relation is \textit{meets}.
	\item Simultaneous events are stacked, one box atop the other. Identical events are placed in the same box, rather than being treated the same as simultaneous ones.
	\item An event which \textit{contains} or \textit{includes} another gains an extended box with thinner walls, and the included event's box is placed inside this box.
	\item If an event \textit{starts} or \textit{finishes} another, it is placed inside that event's extended box, touching the left or right edge, respectively.
	\item Otherwise, none of these configurations may occur.
\end{itemize}

\begin{center}
	\begin{figure}[h!]
		\centering
		\includegraphics[width=0.8\textwidth]{images/tbox}
		\captionof{figure}{\Cref{fig:tango}'s data now drawn using T-BOX \citep[taken from][p. 3, Figure 2]{verhagen2005TBOX}.}\label{fig:tbox}
	\end{figure}
\end{center}
It is stressed that the vertical and horizontal positioning of boxes doesn't mean anything in and of itself, and thus despite a focus on semantically arranging and depicting the temporal relations in a TimeML annotation, T-BOX abandons the ``timeline metaphor'' \citep{verhagen2005TBOX}. However, using the string-based framework described in \cref{sub:strings}, many of the principles which guide the T-BOX architecture can be maintained whilst also preserving the ideology of timelines, being an intuitive way to conceptualise sequences of events and times---see \cref{ssub:timelines}.

Another point of interest with T-BOX is that it requires as input a TimeML document which is `complete', and has already had temporal closure constraints applied to its \verb|<TLINK>| tags, providing all of the inferrable relations ahead of time. It then reduces this to a minimal graph \citep[p. 6]{verhagen2005TBOX}. This is as opposed to the more typical scenario of trying to compute the temporal closure of a document using a constraint propogation algorithm---such as \citet{allen1983maintaining}'s in \cref{other:allen-algo}, p. \pageref{other:allen-algo}---or, as in the present work, using the superposition of strings to calculate relations at the same time as depicting them.

One further aspect of the T-BOX framework worth noting is that it can be used to possibly detect some inconsistencies in input data if there is some part of the data which cannot be drawn using the rules described above. For example, if the \verb|<TLINK>| tags give that some event X is \textit{before} some other event Y, that Y is \textit{before} a third event Z, and also that Z is \textit{before} X---an impossibility due to the circular transitivity---the inconsistency should be discovered when attempting to draw X to the left of Y, which is to the left of Z, which should then somehow be drawn to the left of X. However, while `non-drawability' implies an inconsistency, drawability does not imply consistency \citep[p. 12]{verhagen2005TBOX}, since similarly to the constraint propogation algorithm in \citet{allen1983maintaining}, inconsistencies may appear in the graph which appear consistent when looking at three intervals, but become problematic in a larger context. This issue can be circumvented using a string which may represent far more than three intervals and their relations at once, and strings which represent information inconsistent with the knowledge base are ejected through superposition---see \cref{def:vc-superposition}, p. \pageref{def:vc-superposition}---and thus, `non-superposability' can be seen as similar to the `non-drawability' of T-BOX.

The next section discusses some of the existing approaches to semantic representation of times and events aside from the semantics associated with the annotation schemas of TimeML and ISO-TimeMLs, in particular the structures of Discourse Representation Theory.

\subsection{Temporal Semantics}\label{sub:semantics}
The following sections describe some of the existing ways that the semantics of times and events can be represented.
\nb{This section needs more---Prior's TL, FOL, SOL, Event Calculus?, Frames?, FST? Durand and Schwer?}
% \nb{Bunt (2020) mentions that Abziande \& Bos (2019) mention that almost everyone uses Neo-Davidsonian event semantics...}
% \nb{Harry Bunt's slides from his DCLRS talk would be useful here, else ref ``A semantic annotation scheme for quantification" (Bunt, 2019)}

\subsubsection{Discourse Representation Theory}\label{ssub:drt}
In what have become robust and well-known works since their publication, \citet{kamp1981theory,kamp1988discourse,Kamp1993} presented Discourse Representation Theory (DRT) as a formal framework for semantically representing information derived from discourse---that is, coherent series of sentences or propositions, which may appear in speech or text. The development of DRT aimed to allow model-theoretic semantic representations that go beyond single sentences, which was the standard approach in First-Order Logic (FOL), and to treat phenomena such as anaphora, where an entity named earlier in a discourse may be referred to again later without naming it again. The classic example sentences \citep{kamp1988discourse} involve an indefinite, `a donkey', appearing as an antecedent, and acting as a quantifier which binds the pronoun `it' in the second sentence, as in \cref{ex:donkey-1}.
\begin{subequations}
	\begin{align}
		&\text{``John owns a donkey. He feeds it.''}\label{ex:donkey-1}\\
		&\exists x ~(donkey(x) \land own(John, x) \land feed(John, x))\label{ex:donkey-1-fol}
	\end{align}
\end{subequations}
The sentence in \cref{ex:donkey-2} is problematic in Montague-style approaches, which assume that any indefinite in a sentence should introduce an existential quantifier, but this comes into conflict with the scope of the universal quantifier introduced by `Every' \citep[p, 91]{kamp1988discourse}. In \cref{ex:donkey-2-fol}, the expected existential has acquired a universal force to capture the natural interpretation of \cref{ex:donkey-2} as claiming that every farmer feeds every donkey they own.
\begin{subequations}
	\begin{align}
		&\text{``Every farmer who owns a donkey feeds it.''}\label{ex:donkey-2}\\
		&\forall x \forall y ~((farmer(x) \land donkey(y) \land own(x,y)) \Longrightarrow feed(x,y))\label{ex:donkey-2-fol}
	\end{align}
\end{subequations}
DRT solves these issues with the notion of discourse referents, which are the set of entities under discussion in a given discourse, corresponding to the individual variables of FOL \citep[p. 397]{Bird2009}. Under DRT, indefinites do not introduce existential quantifiers, instead introducing new discourse referents, which are added to a mental representation known as a Discourse Representation Structure (DRS) \citep{geurts2007discourse}. DRSs contain separately both the discourse referents and the DRS conditions, which describe what is known about the referents.

For the example in \cref{ex:donkey-1}, the first sentence is processed and the two entities create two discourse referents, $x$ and $y$, which appear at the top of the DRS in \cref{ex:donkey-1-drs-a}, with the DRS conditions over these referents appearing in the lower box of the DRS.
\begin{align}
	\drs{$x~~~y$}{John($x$)\\donkey($y$)\\owns($x, y$)}\label{ex:donkey-1-drs-a}
\end{align}
As the second sentence of \cref{ex:donkey-1} is processed, the DRS is augmented with two further discourse referents, $u$ and $v$, corresponding with the two pronouns `He' and `it'. Resolving the anaphora\footnote{The component which performs the anaphora resolution is assumed separate to DRT, which merely sets constraints on which discourse referents are available as candidates for selection \citep[p. 399]{Bird2009}.} adds two new conditions, identifying the two pronoun referents with the two existing noun referents in \cref{ex:donkey-1-drs-a}. Finally the condition that $u$ feeds $v$ is added, giving \cref{ex:donkey-1-drs-b}.
\begin{align}
	\drs{$x~~y~~u~~v$}{John($x$)\\donkey($y$)\\owns($x, y$)\\$u = x$\\$v = y$\\feeds($u, v$)}\label{ex:donkey-1-drs-b}
\end{align}
There is a direct translation from a DRS to a formula of FOL, where the discourse referents which appear at the outermost level are interpreted as existential quantifiers, and the DRS conditions are interpreted as being conjoined. The translated formula for \cref{ex:donkey-1-drs-b} is in \cref{ex:donkey-1-drs-b-fol}, which has the same interpretation as \cref{ex:donkey-1-fol} despite its differences.
\begin{align}
	\exists x \exists y \exists u \exists v ~(john(x) \land donkey(y) \land owns(x,y) \land u = x \land v = y \land feeds(u, v))\label{ex:donkey-1-drs-b-fol}
\end{align}
Universal quantification is usually handled by embedding two DRSs connected by an implication within another DRS \citep[p. 98]{kamp1988discourse}, or alternatively by embedding a negated DRS within another embedded negated DRS\footnote{Due to the equivalence $\forall x~(P(x)) \Longleftrightarrow \lnot \exists x ~(\lnot P(x))$, stating that if some property $P(x)$ holds for all values of $x$, then there is no value for $x$ where $P(x)$ does not hold.}. Below, \cref{ex:donkey-2-drs} represents \cref{ex:donkey-2}.
\begin{align}
	\drs{$x$}{farmer($x$)\\\drs{$y$}{donkey($y$)\\owns($x,y$)} $\Longrightarrow$ \drs{$u$}{$u = y$\\feeds($x,y$)}}\label{ex:donkey-2-drs}
\end{align}
DRT is capable of representing events and times as temporal entities which introduce discourse referents and may appear in DRS conditions. For example, including a special `indexical' discourse refererent $now$ for the time of the utterance, as in \citet[p. 104]{kamp1988discourse} and \citet{abzianidze2017parallel}, \cref{ex:drs-temporal-b} represents the sentence in \cref{ex:drs-temporal-a}.
\begin{subequations}
	\begin{align}
		\text{``The doctor cured the patient.''}\label{ex:drs-temporal-a}\\
		\drs{$x~~y~~e~~now$}{doctor($x$)\\patient($y$)\\cure($e,x,y$)\\$e \prec now$}\label{ex:drs-temporal-b}
	\end{align}
\end{subequations}
The condition $e \prec now$ denotes that the event $e$ precedes the time of the utterance, indicating the past tense of the verbal event `cure'. Other available temporal relations available in DRS conditions usually include $=$ equality, $\subset$ during, and $\bigcirc$ overlaps, though other relations occasionally are used, and in principle any binary relation could be used, including those in \citet{allen1983maintaining}'s set. 

The DRS in \cref{ex:drs-temporal-b} represents a Davidsonian approach to event semantics, where an event variable $e$ was added as an argument to the predicate corresponding to the verb \citep{davidson1967logical,Kamp1993}, and DRSs can also feature discourse referents representing statives as well as eventives \citep[p. 103]{kamp1988discourse}. However, by introducing an event's semantic roles as conditions to the DRS, a neo-Davidsonian style \citep{dowty1989semantic} can be implemented instead, reducing the verbal predicate's arguments to just the event variable $e$. This is an approach which \citet{abzianidze2019thirty,Bunt2020} point out is the \textit{de facto} standard approach taken by most (if not all) semantically annotated corpora. Indeed, the Parallel Meaning Bank\footnote{Available at \url{https://pmb.let.rug.nl/}.} (PMB) corpus \citep{abzianidze2017parallel}---which aims to ``provide fine-grained meaning representations for words, sentences and texts'' and contains over 8000 DRSs from English sentences---uses this approach in their DRS representations. The DRS in \cref{ex:drs-temporal-c} below represents the sentence in \cref{ex:drs-temporal-a} again, this time including such semantic roles as might appear in the PMB, which uses an inventory of roles from the VerbNet resource \citep{schuler2005verbnet}.
\begin{align}
	\drs{$x~~y~~e~~now$}{doctor($x$)\\patient($y$)\\cure($e$)\\Agent($x$)\\Patient($y$)\\$e \prec now$}\label{ex:drs-temporal-c}
\end{align}
Automatic parsing of discourse text to DRSs is a non-trivial, multi-faceted task, involving tokenisation, part-of-speech recognition, and syntactic parsing before DRSs can even be constructed. A wide-coverage parser called Boxer was released by \citet{Bos2008}, which claimed 95\% converage for semantic analysis of newswire texts, although significantly for the present work, it was noted in that release that performance for temporal information was not as strong as its other areas \citep[pp. 283--285]{Bos2008}. Boxer takes as input a Combinatory Categorial Grammar (CCG) \citep{steedman2000syntactic,steedman2011combinatory} which has itself been parsed following named-entity recognition, and part-of-speech tagging prior to that, using the C\&C tools \citep{curran2007linguistically}, and outputs one or more DRSs as formally interpretable semantic representations \citep[p. 285]{Bos2008}. An updated version of Boxer \citep{van-noord-etal-2018-exploring} is implemented as part of the Parallel Meaning Bank toolchain \citep{abzianidze2017parallel}---although, this version had unfortunately not been made available for general use at the time of writing. However, the recent shared task created by \citep{abzianidze2019first} shows that there is an interest in developing systems which can perform automatic DRS parsing. The state-of-the-art was improved considerably in this task, with the winning system of \citet{liu2019discourse} achieving an F1 score of 84.8\%, an improvement from the baseline score of 54.3\%.

% One issue with DRSs in terms of representation for times and events

This chapter has attempted to give a general overview of some of the literature which informs the current work. In particular, the interval algebra put forward in \citet{allen1983maintaining}, which finds application in the temporal relations of the TimeML \citep{timeml2005timeml} and its successor ISO-TimeML \citep{pustejovsky2010iso}, which has become the international standard for semantic annotation of temporal information, and also the empirical validation of \citet{reichenbach1947elements}'s framework of tense and aspect in \citet{Derczynski2013}, which also brings out \citet{Freksa1992}'s semi-interval relations, and finally the semantic structures of the formalism of Discourse Representation Theory \citep{kamp1981theory} which aims to reflect the context of an utterance in its model-theoretic interpretation. The next chapter goes into some depth for the approach to temporal semantics known as finite-state temporality, and details the string-based framework it licences which forms the backbone of the work.

\newpage
\section{Finite-State Temporality}\label{sec:fst}
Following the intuition that sequences of (possibly overlapping) events and time periods may be conceptualised in a manner akin to a strip of film, or a timeline, finite-state techniques may be applied to temporal semantics in an approach known as \textit{finite-state temporality} \citep{fernando2005entailments}. In this chapter, strings are demonstrated as a tool of choice in modelling sequences of times and events for use within this approach, justified by their interpretation as finite models of Monadic Second-Order Logic, which leads to an equivalence with regular languages (see \cref{ssub:mso}). The advantage of this is that strings can be accepted and parsed by finite-state automata (FSA), which are a well-known formalism and have found significant usage across many fields, including disciplines of Mathematics, Linguistics, and Computer Science (see, for example, \citet{buchner1993finite,veanes2012symbolic}), as well as in modern software development (for instance, \citet{khourshid_2015} provides a tool for using FSA in the creation of online web applications). Such widespread application of FSA is due to their flexibility and efficiency as a technology, with benefits such as deterministic recognition being linear according to the length of the input, the associated closure properties of regular lanugages, and the ability to compose several automata \citep{wintner2007finite}.
% \url{https://www.elprocus.com/finite-state-machine-mealy-state-machine-and-moore-state-machine/}
% \utl{https://www.quora.com/What-are-the-advantages-of-using-finite-state-machines}

The mechanics of these temporal strings are shown in detail, along with an explanation on the methods of their creation, and a discussion on the granularity of temporal information that should be included within a string---that is, what level of detail should be considered when representing events in this manner. A number of operations are subsequently described for working with these strings, in particular \textit{superposition} for the composition of multiple strings and combining the data found within them. The availability of these manipulations will prove useful when reasoning about event relations, and also in maximising data density, as will be seen in \cref{sub:extracting} and \cref{sub:reasoning}.

A description follows of how strings may be applied in the creation of timelines, representations which contain the temporal information---events, times, and the relations between them---as extracted from an annotated piece of text, which has uses in the areas of, for instance, automatic summary-generation, fact-checking, and question-answering systems. Additionally, the techniques which are laid out in \cref{ssub:operations} may be augmented with additional constraints so that strings can be used to model scheduling restrictions---for instance, that in a given system some event $a$ must occur before some other event $b$, but may not be occurring while $c$ is occurring---and the superposition of these amounts to constraint satisfaction. A variation of the well-known Zebra Puzzle (also referred to as `Einstein's Riddle', as it is occasionally attributed to Albert Einstein---see \citet[p. 10]{stangroom2009einstein}) which uses temporal properties in place of spatial constraints is provided, exemplifying this particular usage of strings, as well as a string-based treatment of the train scheduling problem from \citet{durand2008tool}.

\subsection{Strings for Times and Events}\label{sub:strings}
A string is a basic computational entity, defined as a finite sequence of symbols selected from some finite alphabet. They are amenable to manipulation using finite-state methods, something lacking in the infinite models of predicate logic, thanks to fixing finite sets of symbols to serve as the alphabets which make up the strings.

Strings as described and used throughout this work are used to represent sequences of events and time periods such that the linear order and inter-relations of these events and times are clearly apparent, while unnecessary repetition of information is avoided. For example, where $js =$ ``John sleeps'', $fa =$ ``The fire alarm sounds'', and $lt =$ ``Last Tuesday'', the sentence ``John slept through the fire alarm last Tuesday'' might be represented by the string \EventString{{}|lt|js,lt|fa,js,lt|js,lt|lt|{}}\footnote{It is worth noting that, although tense and aspect are often abstracted away from the string models, this is not a necessity, and speech and reference times may be represented in the same way as event times \citep{fernando2016regular,Derczynski2013,reichenbach1947elements}.} (a detailed explanation follows in \cref{ssub:creating}).

These strings model the concept of inertial worlds, wherein a state will persist unless and until it is altered. The intuition for viewing inertia as a default state seems to go at least as far back as Aristotle (``But neither does time exist without change'' in \textit{\citetalias{aristotlePhysicsIV}}), and is known by the term \textit{commonsense law of inertia} \citep[p. 19]{shanahan1997solving}. This notion is also present in the Event Calculus, which represents the effects of actions on fluents in order to reason about change \citep{Kowalski1986,Miller1999,Mueller2008}, and in the Fluent Calculus \citep{thielscher1999situation}, which asserts that a state will be unaltered after an event, except for just those conditions which the event changes.
Building this inertial world view into strings allows for certain flexibilities, as real duration does not need to be accounted for\footnote{Real-time durations may still be represented using strings, but the depiction will not (necessarily) align with the assumption that one string should be longer than another if it features events that have a longer duration.} when, for example, superposing strings in order to determine the relations between the events they mention (see \cref{para:str-op-sp}, p. \pageref{para:str-op-sp}; \cref{ssub:superposition}). \citet[p. 44]{Fernando2018} also directly connects the notion of ``No change unless forced'' with strings, using it to motivate the concept of actions creating change. Additionally, the built-in inertiality---coupled with the fact that strings are explicit as to whether a particular \textit{fluent} holds or does not hold at any particular moment in time---means that, for string-based representations of events and times, the classic issue of the frame problem\footnote{The frame problem is an issue that can arise in first-order logic representations of the world, whereby specifying the conditions which change as the result of an action is not sufficient to entail that no other conditions have changed.} is avoided (see \citet[pp. 30-31]{Mccarthy69somephilosophical}).

% \nb{Would be useful to discuss logical circumscription and show/discuss how strings are an application since fluents not in the vocabulary are not relevant to the string, and fluents not mentioned in a box are implicitly not true at that time point (but can be made true through superposition) -- is this still circumscription?}

% \nb{Should also probably go into a small bit of detail about event calculus in lit. section}

% \nb{Perhaps talk about forces, a la Tim's (2018, superseded 2020) paper? since it couples with the inertia idea, i.e. nothing changes without a force to change it. It's probably too much to mention Newton's 1st law...?}

A fluent is a condition which may change over time---thus a predicate such as $\allowbreak{}Sleeps(john)$ (``John sleeps'') becomes the fluent $Sleeps(john, t)$, where $t$ is the time at which John is sleeping. Following the convention set out and used by \citet{Mccarthy69somephilosophical}, \citet{van2008proper}, \citet{fernando2016prior} (among others), a fluent may be understood here as naming a temporal proposition---some event, time period, or state which may change (which hereafter will also be referred to as an event, as in \citet{Pustejovsky2005}). Sets of fluents will be encoded as symbols so that any number of them may hold at once, and these symbols will make up the alphabet from which strings are created.

% How should events be labelled? Some say an event is defined in part by its participants \citeneeded{}. Many of the procedures defined in this work which use these strings operate primarily on a syntactic level rather than depending on the lexical semantics of individual words, in order to produce an approach that works broadly. As such, i
In most cases, simple identifiers suffice for the purposes of labelling fluents as they appear in a string---so, for example, $Sleeps(john, t)$ may be labelled as \texttt{"js"}, where the time $t$ being represented by the fluent's position in the string, explained in depth in the following section. This follows TimeML's standard \citep{timeml2005timeml,pustejovsky2010iso} of using identification labels such as \texttt{"e1"} or \texttt{"t2"} for events and times. However, in some instances, particularly when drawing inferences (see %\cref{ssub:inferring} and 
\cref{ssub:ontology}) using semantic roles or lexical semantics, it will be useful to use a fuller labelling system, and so a string \EventString{{}|js|{}} would be rendered instead as \EventString{{}|sleeps(john)|{}}.

\subsubsection{Creating Strings}\label{ssub:creating}
In order to create a string, first it is necessary to fix a finite set $\V$ of symbols which represent the times and events under discussion, such that each $v \in \V$ will be understood as naming a \textit{fluent} as a unary predicate which holds at a particular time. A string $s = \sigma_1\sigma_2\cdots\sigma_n$ of subsets $\sigma_i$ of $\V$ is interpreted as a finite model of $n$ discrete, contiguous moments of time, with $i \in \{1, 2, \ldots, n\}$.

The set $\V$ will be known as a \textit{vocabulary}, and the powerset of $\V$ will serve as a finite alphabet $\Sigma = 2^{\V}$ of a string $s \in \Sigma^*$. Accordingly, at each position $i$ in the string $s$, the \textit{component} $\sigma_i$ will be a (possibly empty) set of the fluents which hold at that position.

The chronology of a string is read from left to right, and thus, each component of the string depicts one of the $n$ moments similar to a snapshot, or a frame of a film reel, and specifies the set of exactly those fluents which hold simultaneously at the $i^{th}$ moment\footnote{Conversely, the set $\V-\sigma_i$ specifies exactly those fluents which \textit{do not} hold at the $i^{th}$ moment. This relates to the notion of logical circumscription \citep{mccarthy1980circumscription}, wherein if a formula is not known to be true, then it must be false.}. A string does not (necessarily) give any indication of real-time duration\footnote{Although the durations may still be used in certain applications---see \cref{ssub:zebra}, p. \pageref{tab:rel-durations}.}, due to the fact that it models an inertial world, and thus if a symbol occurs in several string components, this is not implicative of the fluent associated with that symbol occurring multiple times, nor of the fluent's real-time duration being necessarily different to another fluent whose associated symbol only occurs in a single string component (see also \cref{para:block-compression}, p. \pageref{para:block-compression}). A fluent $a \in \sigma_i$ is understood to be occurring before another fluent $b \in \sigma_j$ if $i < j$ and $b \notin \sigma_i$; if $a \in \sigma_i$ and $b \in \sigma_i$, then $a$ and $b$ are understood as occurring at the same time.%\nb{this is more or less the same as in ISA-13 -- should cite?}.

For convenience of notation, boxes \ebox{\cdot} are used instead of curly braces $\{\cdot\}$ to denotate the sets which make up each component $\sigma_i$ of a string $s = \sigma_1\sigma_2\cdots\sigma_n$ (as in \citet{Fernando2004,fernando2015semantics,fernando2016prior,woods2017towards})---for example, the string $\{\}\{a\}\{a,b\}\{b\}\{\}$ is written as \EventString{{}|a|a,b|b|{}}. This lends to the intuition that the strings may be read as strips of film, or as panels of a comic, with the same narrative-style layout as a timeline\footnote{It is worth pointing out here that, as each box is a set, the order of fluents appearing in it is immaterial---for instance, \EventString{a,b} is exactly equivalent to \EventString{b,a}, where in both cases, $a$ is co-occurring with $b$. This contrasts with \EventString{a|b} vs \EventString{b|a}, where in the first case $a$ is occurring before $b$, and in the second, the converse is true.}. An empty box \ebox{} is drawn for the empty set $\emptyset$: this is a string of length 1, a moment of time during which no fluent $v \in \V$ holds. This should not be confused with the empty string $\epsilon$, which has length 0, and contains no temporal information.

It will be said that for any event $a$ occurring in a string $s = \sigma_1\sigma_2\cdots\sigma_n$, $a$ may not \textit{judder}---that is, if $a$ appears in multiple positions of the string, then those positions are contiguous; there are no gaps between appearances of $a$ in $s$:
\begin{align}\label{impl:contiguous-events}
	a \in \sigma_i \wedge a \in \sigma_j \wedge i < j \Longrightarrow \forall k \in [i~..~j]~(a \in \sigma_k)
\end{align}
If some string, such as in \cref{ex:judder-invalid}, features judder in this way, it is said to be invalid, similarly to \citet[p. 134]{Pustejovsky2005} distinguishing separate instances of an event which are referred to together---see \cref{other:pustejovsky-type-instance}, p. \pageref{other:pustejovsky-type-instance}---and only allowing instances to appear in a string. Judder can, if necessary, be treated by subindexing an event---for example, a juddering event $a$ becomes separate sub-events $\{a_1, a_2, a_3\}$, as in \cref{ex:judder-valid}. Alternatively, similarly to \citet{bunt2010annotating} and ISO-TimeML's treatment of non-contiguous events over a duration\footnote{See \cref{other:iso-mlink}, p. \pageref{other:iso-mlink}.}, it may be possible to interpret judder as one whole event, which includes the pauses, as in \cref{ex:judder-valid-2}, which would then be \textit{block compressed}---see \cref{def:bc}, p. \pageref{def:bc}.
\begin{subequations}\label{ex:judder}
	\begin{align}
		\EventString{a|{}|a|{}|a}\label{ex:judder-invalid} ~-~& \text{invalid}\\
		\EventString{a_1|{}|a_2|{}|a_3}\label{ex:judder-valid} ~-~& \text{valid}\\
		\EventString{a|a|a|a|a}\label{ex:judder-valid-2} ~-~& \text{valid}
	\end{align}
\end{subequations}
Sets of strings are \textit{languages}, which allow for grouping of strings based on some property---such as their source---and which introduces a method for handling such cases of ambiguity and non-determinism as will be shown in more detail in \cref{ssub:superposition} (see also \cref{ex:lang-superposition}, p. \pageref{ex:lang-superposition}). For example,
\begin{align}
	L = \{\EventString{a|b|c}, \EventString{a|c|b}\}
\end{align}
where the language $L$ contains two strings which represent different---albeit related---sequences of events.

Note that a language containing only a single string may be conflated with its sole member, and vice versa. For instance:
\begin{align}
	\EventString{a|b|c} \approx \{\EventString{a|b|c}\}
\end{align}
This is a useful admittance, particularly in regards to superposition and other string operations (see \cref{ssub:operations}), when it may be necessary to, for example, superpose a string and a language, in which case the string is conflated with a language containing just that string. This allows for superposition to be extended beyond just two strings to any arbitrary number: $s \spvc s' \spvc s'' \spvc \ldots$ (see p. \pageref{def:vc-superposition}).\label{pt:conflation}

The next section details how strings are expressions of finite models of Monadic Second-Order Logic, and how they are thus manipulable by finite-state automata.


\subsubsection{Strings as MSO Models}\label{ssub:mso}

Strings of symbols representing events as described in \cref{ssub:creating} may be interpreted as finite models of Monadic Second-Order Logic (MSO)\footnote{See \citet[ch. 7]{Libkin2004} for an in-depth introduction to the facets of MSO.}. MSO is a fragment of Second-Order Logic that restricts quantification so as to be permitted solely for unary predicates, which is equivalent to quantification over sets---this is due to the fact that a unary predicate may be effectively described by the set of terms for which that predicate is true. That is, in a given model of MSO, if there exists some property $P$, then $\llbracket P \rrbracket$ is the set of individuals for which $P$ holds---known as the \textit{interpretation} of $P$ relative to the given model---such that:
\begin{align}\label{impl:prop-interpretation}
	P(x) \Longleftrightarrow x \in \llbracket P \rrbracket
\end{align}
This may be construed for temporal representation by considering each predicate of a model as describing an event, and the terms which make that predicate true are the moments (relative to the model) during which the associated event is occurring.

This is most clearly illustrated by means of an example. The string below in \cref{ex:mso-string} will serve for this purpose, with the positional indices shown underneath the string position with which it corresponds:
% align/alignat does column alignment as alternating left/right, so double ampersand
% gets consistent left alignment
\begin{alignat}{5}\label{ex:mso-string}
	&\EventString{{}} && \EventString{a} && \EventString{a,b} && \EventString{a} && \EventString{{}}\\[-1.2em]
	&{\scriptstyle 1} && {\scriptstyle ~2} && {\scriptstyle ~~3} && {\scriptstyle ~4} && {\scriptstyle 5}\notag
\end{alignat}
This is a string of length $5$, which contains two events, $a$ and $b$, where $b$ occurs \textit{during} $a$ (see \cref{tab:allen-rels-strings}). The linear ordering of $a$ and $b$ can be identified by the string positions in which each occurs \citep{fernando2016regular,Fernando2018}:
\begin{align}
	\llbracket P_a \rrbracket = \{2,3,4\} \text{ and } \llbracket P_b \rrbracket = \{3\}
\end{align}
where $P_a$ and $P_b$ are unary predicates, and their interpretations are subsets of $\{1,2,3,4,5\}$, which is the set of all string positions for the string in \cref{ex:mso-string}.

Generally, for any string $s = \sigma_1\sigma_2\cdots\sigma_n$ with length $n \ge 0 \in \mathbb{N}$, the set $[n]$ of string positions\footnote{If $n = 0$, i.e. $s = \epsilon$, the empty string, then $[n] = \emptyset$, allowing a model to have an empty domain, as in \citet{Libkin2004}.} is defined as:
\begin{align}
	[n] := \{1,2,\ldots,n\}
\end{align}
Since $s$ is restricted to being a finitely bounded string, $[n]$ is also finite, and thus the MSO model described by $s$ must also be finite.

The vocabulary $\V$ is the set of fluents which appear in $s$, and for each $v \in \V$, the set of positions during which $v$ (as a fluent) occurs $\llbracket P_v \rrbracket$ is a subset of $[n]$, and if there exists some $x \in [n]$ such that $x \in \llbracket P_v \rrbracket$, then $P_v$ holds at $x$, as in \cref{impl:prop-interpretation}:
\begin{align}
	\llbracket P_v \rrbracket &\subseteq [n]\\
	P_v(x) \Longleftrightarrow& ~x \in \llbracket P_v \rrbracket
\end{align}
The successor relation which links each string position to the next is also defined:
\begin{align}
	S_n := \{(i, i+1) ~|~ i \in [n - 1]\}
\end{align}
Now, for each $v \in \V$, the predicate $P_v$ specifies all the string positions in which $v$ occurs:
\begin{align}\label{def:str-positions}
	\llbracket P_v \rrbracket := \{i \in [n] ~|~ v \in \sigma_i\}
\end{align}
If MSO$_{\V}$ is the set of sentences (closed formulas, with no free variables) of MSO whose vocabulary is limited to subsets of $\V$, then an MSO$_{\V}$ model $mod(s)$---which is described by the string $s$---is defined by the tuple \citep{fernando2016regular}:
\begin{align}
	mod(s) := \langle [n], S_n, \{\llbracket P_v \rrbracket ~|~ v \in \V\} \rangle
\end{align}
Given an arbitrary MSO$_{\V}$ model $M$, the string $str(M)$ which describes it can be obtained by inverting \cref{def:str-positions} to get each set $\sigma_i \in str(M)$, for $i \in [n]$, where $\llbracket P_v \rrbracket_M$ is the interpretation of the predicate $P_v$ relative to the model $M$\footnote{In general, for a given string $s$, it will be convenient to write the interpretation of a property $P$ relative to the MSO$_{\V}$ model described by $s$, $\llbracket P \rrbracket_{mod(s)}$, as simply $\llbracket P \rrbracket_s$.}:
\begin{align}
	\sigma_i := \{v \in \V ~|~ i \in \llbracket P_v \rrbracket_M \}
\end{align}
There is a fundamental theorem which was independently put forward by B\"{u}chi, Elgot, and Trakhtenbrot \cite[p. 30]{fernando2016regular}\footnote{A proof is given in \citet[p.124, Theorem 7.21]{Libkin2004}.} which states that sentences $\phi$ of MSO capture regular languages---that is, \textit{the MSO-definability of a language is equivalent to its regularity}. This leads to the following definition in \citet[p. 35]{Fernando2018} of the set of regular languages over the powerset $2^{\V}$ of the vocabulary $\V$\footnote{The powerset $2^{\V}$ is used here in place of the usual $\V$, due to the fact that strings may allow any number of fluents $v \in \V$ to hold at once.}, given by the sentences $\phi$ of MSO$_{\V}$:
\begin{align}\label{def:mso-regularity}
	\{s \in (2^{\V})^* ~|~ mod(s) \models \phi \}
\end{align}
Thus, languages of temporal strings are regular, and as such are open to manipulation and reasoning using finite-state techniques, due to the equivalence between regular languages and finite automata according to Kleene's theorem (see, for example, \citet[p. 41]{yu1997regular}). For instance, the entailment of one regular language by another is decidable, which is not the case for First-Order Logic \citep{trakhtenbrot1953recursive,elgot1966decidability}. This translates to being able to determine whether one string entails another, which is a powerful feature for ascertaining the relations which appear in a particular string, and for reasoning with them (see \cref{para:str-op-projection}, p. \pageref{para:str-op-projection}; \cref{ssub:superposition}).
% \nb{more here, would be good to have some examples of an MSO-sentence eg fernando2016a p.33}

The fact that strings can be construed as models of MSO gives rise to a convenient method of comparison between strings, whereby the set of temporal relations (see \Cref{tab:allen-rels-strings}, p. \pageref{tab:allen-rels-strings}) found in one string $s$ may be said to be equal to the set of relations in another string $s'$ if $s$ and $s'$ are analogous.

\paragraph{Analogous Strings}\label{para:analogous-strings}
A string $s$ will be said to be \textit{analogous} to some other string $s'$ if the MSO models corresponding to each string can be said to be, in a sense, isomorphic%
% \nb{wiki: bijection, isomorphism;\\\url{https://www.britannica.com/science/isomorphism-mathematics}\\\url{https://mathworld.wolfram.com/Isomorphism.html}\\\url{https://plato.stanford.edu/entries/modeltheory-fo/}}
---that is, $s$ and $s'$ are of equal length, there exists a bijective function $f$ mapping between the vocabulary of $s$ and the vocabulary of $s'$, and for every fluent $v \in \V_s$, its image $f(v) \in \V_{s'}$ appears in only those same strings positions in $s'$ as $v$ appears in $s$:
\begin{align}
	mod(s) &\cong mod(s') \Longleftrightarrow \\\notag &\text{length}(s) = \text{length}(s') \land \exists f ~(f: \V_s \leftrightarrow \V_{s'} \land \forall v \in \V_s ~(\llbracket P_v \rrbracket = \llbracket P_{f(v)} \rrbracket)\\
	mod(s) &\cong mod(s') \Longleftrightarrow s \sim s'
\end{align}
For example, in \cref{ex:analog}, \Before{a}{b} is analogous to \Before{c}{d}, while in \cref{ex:not-analog} \Before{a}{b} is not analogous to \Overlaps{c}{d}
\begin{align}
	\Before{a}{b} &\sim \Before{c}{d}\label{ex:analog}\\
	\Before{a}{b} &\nsim \Overlaps{c}{d}\label{ex:not-analog}
\end{align}
This definition of analogy can be lifted to languages $L$ and $L'$ simply by testing if all strings in $L$ are analogous with all strings in $L'$:
\begin{align}
	L \sim L' := \forall (s \in L) \forall (s' \in L')~ (s \sim s')
\end{align}
Determining whether strings are analogous is useful when ascertaining the relations between events appearing in a string. If the relations between events in a string $s$ are known, and another string $s'$ can be shown to be analogous to $s$, then the relations between the events that appear in $s'$ are also known\footnote{For example, the string on the left hand side of \cref{ex:analog} says that event $a$ is \textit{before} event $b$, and since the string on the right hand side is analogous, THEN event $c$ must be \textit{before} event $d$.}. By employing this concept in conjunction with that of projection (see p. \pageref{para:str-op-projection}) and a set of reference strings---such as those modelling Allen's relations (see \cref{tab:allen-rels-strings})---it becomes simple to determine which relations appear in a string.

Further, if a pair of strings are shown to be analogous, this can make any calculations or processing involving these strings to be more efficient: any set of operations applied to one of the strings would produce the same result if applied to the other, so there is no need to apply them to both. This is expanded upon further in \cref{ssub:timelines}, see p. \pageref{ex:sp-analogy}.
%%%%%%%%%%%%
% Lead into granularity?

\subsubsection{Granularity: Points and Intervals}\label{ssub:granularity}
The vocabulary from which a string's alphabet is constructed is a set of fluents, or temporal propositions---for example ``John sleeps'' $Sleeps(john, t)$, where $Sleeps$ is a predicate and $john$ is an individual for which $Sleeps$ holds true for some time $t$. However, it has not yet been made explicit whether $t$ should be an instantaneous point in time, or some temporal interval which has a non-zero duration. The need for this distinction is presented below.

In order for a string such as in \cref{ex:subintervals} to be valid, where some event $a$ occurs in multiple positions in the string, $a$ must be an interval: a period of time with a beginning and an ending, and $a$ is occurring between these.
\begin{align}\label{ex:subintervals}
	\EventString{{}|a|a,b|{}}
\end{align}
This allows for $a$ to be represented via subdivisions\footnote{An interval may be subdivided theoretically \textit{ad infinitum}, though the constraint of a finite vocabulary prevents this from occurring within a string.} across the components within the string, such that the event $a$ is understood as beginning at its first (leftmost) occurrence, and ending after its last (rightmost)---as long as it appears in a component, the event is occurring at that moment. While a point is instantaneous and therefore indivisible, an interval may be split into an arbitrary number of contiguous subintervals in this way, which allows for the representation of other events co-occurring with a specific subdivision of an interval, as in \cref{ex:subintervals}, where $a$ is occurring during two string positions $\llbracket P_a \rrbracket = \{2, 3\}$, while $b$ is only occurring during a single string position, $\llbracket P_b \rrbracket = \{3\}$. In this example, the real-time interval duration $\tau$ of position 3 is equal to that of the event $b$, which is also the span of time during which $a$ and $b$ are co-occurring. The summed duration $\tau'$ of positions 2 and 3 is equal to that of the event $a$, and so the duration of position 2 (the span of time in which $a$ is occurring, but $b$ has not begun) is equal to $\tau' - \tau$.

If $a$ and $b$ represent points of time, then the string in \cref{ex:subintervals} is invalid---since a point is instantaneous, it may not occur at multiple string positions, which are discrete moments of time. Additionally, it is not possible for one point to have a shorter duration than another, and thus the fact that $b$ occurs at the end of $a$ is lost. \citet{allen1983maintaining} also discusses the logical and physical issues with allowing events to be representable as instantaneous points, but there is also a cognitive issue, discussed in \citet{Freksa1992}, where events must have some non-zero duration in order to be perceivable \citep{hamblin1972instants}.

That is not to say that this information cannot be represented in a string using points, but instead, a translation from events to \textit{event borders} must occur, where event borders are points representing the beginnings and endings of events. \citet{Fernando2018,fernando2018prior} uses\footnote{In a fashion similar to the S-words of \citet{durand2008reasoning,durand2008tool}, which also use the beginnings and endings of events as basic.} $l(a)$ as the (open) left border and $r(a)$ as the (closed) right border of some event $a$---the \textit{granularity} is altered to focus on the event borders (as points) rather than on the events (as intervals). In keeping with the analogy of strings being similar to strips of film, the change of granularity is akin to altering the level of `camera zoom' to focus on details which were previously considered simply as parts of the whole.

This change does introduce additional complexity to the vocabulary, with the new set $\breve{\V}$ constructed from the old $\V$ \cite[p. 37]{Fernando2018}:
\begin{align}
	L &= \{ ~l(v) ~|~ v \in \V~ \}\notag\\
	R &= \{ ~r(v) ~|~ v \in \V~ \}\notag\\
	\breve{\V} &:= L \cup R ~~~~~~~~\text{if } L, R, \text{ and } \V \text{ are pairwise disjoint.}
\end{align}
The cardinality of $\breve{\V}$ is thus twice that of $\V$, with two symbols required to represent the same information of a single interval. The translation from a string $s = \sigma_1\sigma_2\cdots\sigma_n$ whose symbols are events (intervals) to a string $\breve{s} = \breve{\sigma_1}\breve{\sigma_2}\cdots\breve{\sigma_n}$ whose symbols are event borders (points) is given as \cite[p. 38]{Fernando2018}:
\begin{align}
	\breve{\sigma_i} := \begin{cases}
		\{ r(a) ~|~ a \in \sigma_i \} & \text{if}~ i = n\\
		\{ l(a) ~|~ a \in \sigma_{i+1} - \sigma_i \} \cup \{ r(a) ~|~ a \in \sigma_{i} - \sigma_{i+1} \} & \text{if}~ i < n
	\end{cases}
\end{align}
The symbol $l(a)$ appearing in $\breve{\sigma_i}$ says that the event $a$ is not occurring at position $i$, but is occurring at position $i+1$. The symbol $r(a)$ appearing in $\breve{\sigma_i}$ says that $a$ is occurring at position $i$, but is not occurring at position $i+1$. Using event borders in this way, the string in \cref{ex:subintervals} would be translated to be drawn as:
\begin{align}
	\EventString{l(a)|l(b)|r(a),r(b)|{}}
\end{align}
Points are, in some ways, simpler than intervals---for instance, two points may be related in just three ways (\textless, =, and \textgreater), while a pair of intervals have thirteen possible relations between them (precisely, the set of relations given by Allen's interval algebra \citep{allen1983maintaining}, see \Cref{tab:allen-rels-strings}, p. \pageref{tab:allen-rels-strings}). Additionally, focusing on the borders of events opens up a pathway to representing incomplete information, by omitting one of the beginning or ending of an event from the vocabulary of a string, it is left unspecified, which is often the case in natural language. For example, in \cref{ex:points-incomplete} below, it is known that the events $a$ and $b$ begin simultaneously, but it is not known when $a$ ends:
\begin{align}\label{ex:points-incomplete}
	\EventString{l(a),l(b)|r(b)|{}}
\end{align}
This situation cannot be simply represented using ordinary intervals, since for any symbol in a string's vocabulary, its appearance or non-appearance in a string component indicates explicitly whether or not the event is occurring at that moment, with no option for `possibly occurring'.

This work, however, will primarily consider intervals as the basic unit of representation for fluents. When looking at a particular string component, it is more straightforward to tell whether some event $a$ is occurring at that moment if the symbol $a$ appears within that component, than to have to either check backwards and forwards through the string to see if the event has begun ($l(a)$ appears somewhere to the left of the component of interest) and not yet ended ($r(a)$ does not appear to the left of the component of interest), or to have to perform a translation. This also more closely mirrors the intuition of the analogy with panels of a comic strip or frames of a reel of film, such that all events which are occurring at any particular moment are `visible' within that panel or frame (string component). There is further a higher level of descriptiveness that can be achieved with thirteen relations between a pair of intervals, as opposed to requiring a set of four points to achieve the same.% Allowing for a combination of points and intervals to appear in a string only complicates the vocabulary and operations further.

Chiefly, however, among reasons to generally consider intervals as primitive is the fact that ISO-TimeML, the international standard for temporal annotation \cite{pustejovsky2010iso}, considers events and event-like entities to be intervals, and uses relations which are based in \citet{allen1983maintaining}'s set of interval relations. Further, \citet[p. 201]{Freksa1992} agrees with Allen from a cognitive perspective that events should not be ``represented by points on the real line'', and additionally describes the concept of \textit{semi-intervals}---using intervals to represent the beginnings and endings of events---which allows for treating incomplete information similar to using points (as in \cref{ex:points-incomplete} above), and expanding the number of possible relations to a set of 31 options. Nonetheless, semi-intervals do, again, introduce an additional level of complexity over plain intervals, and are discussed more completely in \cref{ssub:incomplete}.

% It's worth noting that there are translations available between the different granularities.

It is worth noting here that, in general, in this work it is also assumed that any text which will be used as a source for the creation of strings will feature only \textit{finite events}. The fluents which represent these events will therefore hold for a finite amount of time, and thus intervals appearing in strings will be \textit{bounded} \citep{allen1994actions}. That is, for some event $a$, there is some time \textit{immediately before} $a$ holds during which $a$ does not hold, and similarly, there is some time \textit{immediately after} $a$ holds during which $a$ does not hold. This fact is represented through the use of bounding empty sets, and so any given string will both begin and end with an empty set, drawn as an empty box \ebox{}\footnote{Equivalently, as a formula of MSO: $\forall v \in \V~ (\lnot P_v(1) \land \lnot P_v(n))$.}. The framework does not, in fact, require this assumption to be true---compare the bounded (finite) interval $a$, drawn as \EventString{{}|a|{}}, and the non-bounded (infinite) interval $b$, drawn as \EventString{b}---and indeed, it is convenient to go beyond it when discussing the handling of incomplete data, although this moves away from plain intervals to semi-intervals (see \cref{ssub:incomplete}).

The following section describes a number of operations which may be used with interval-based strings, though it should be pointed out that for each operation there is an equivalent which may be used for a string which treats event borders as basic instead of events.

\subsubsection{String Operations}\label{ssub:operations}
% Many many operations. The most important is \textbf{superposition} (and it's more advanced forms)! Also we have block compression, reduct, vocabulary, projection, border translations. Note also that there are equivalent operations for strings which use points instead of intervals.
A number of operations are available for the manipulation and description of strings which represent sequences of times and events, and these are detailed below. Key among these is \textit{superposition} (basic, asynchronous, and vocabulary-constrained), which may be used to combine the information from multiple strings, but additionally defined are: an operation to find the vocabulary of arbitrary strings; block compression, which removes duplicate string components; reduct, which filters a string to only contain a specified set of events; and projection, whereby a string can be said to contain the same temporal data as another. Many of these operations are also available at the language level, in general by performing the particular desired operation on each string within the language.

\paragraph{Vocabulary:}
The \textit{vocabulary} $\V$ of a string or language is the set of fluents which appear in it. For an arbitrary string $s = \sigma_1\sigma_2\cdots\sigma_n$, the vocabulary $\V_s$ of $s$ may be determined by taking the union of the string's components:
\begin{align}
	\V_s := \sigma_1 \cup \sigma_2 \cup \cdots \cup \sigma_n
\end{align}
and the vocabulary of a language is just the union of the vocabularies of the strings it contains:
\begin{align}
	\V_L := \bigcup \{\V_s ~|~ s \in L \}
\end{align}
The vocabulary of a string is a key factor in the calculation of many other operations, notably projection (p. \pageref{para:str-op-projection}) and vocabulary-constrained superposition (p. \pageref{para:str-op-vc-sp}).
% Determining a string's vocabulary permits some operations to be chained 
\paragraph{Block Compression:}\label{para:block-compression}
Since the length of a string $s = \sigma_1\sigma_2\cdots\sigma_n$ does not reflect its real duration (due to the understanding that strings model intertial worlds---see \cref{sub:strings} p. \pageref{sub:strings}), it is also not required that the length of time represented by any $\sigma_i$ is equal to that represented by any $\sigma_j$, for $i \neq j$. Similarly, if a fluent symbol $v \in \V$ from the vocabulary appears in both $\sigma_i$ and $\sigma_{i+1}$, this does not imply that the event represented by $v$ has a duration twice as long as if it had only appeared in $\sigma_i$. Indeed, the symbol $v$ may appear in any number of consecutive positions in $s$ without affecting the interpretation of the real length of time of the event it represents. Further, if the string features a repeating component, i.e. $\sigma_i = \sigma_{i+1}$ for any $1 \le i < n$, the interpretation of the string is not affected by the deletion of one of either $\sigma_i$ or $\sigma_{i+1}$. So for example, the interpretation of the string \EventString{a|a|a,b|b|b} is equal to the interpretation of the string \EventString{a|a,b|b}. A string featuring such repetitions is said to contain \textit{stutter}.

As a result, the \textit{block compression} $\bc(s)$ of a string $s$ may be introduced, which removes any stutter present in $s$. This is defined as \citep{fernando2015semantics, woods2017towards}:
\begin{align}\label{def:bc}
\bc(s) := 
\begin{cases}
	~~s & \text{if \textit{length}}(s) \leq 1\\
	~~\bc(\sigma s') & \text{if } s = \sigma \sigma s'\\
	~~\sigma \bc(\sigma' s') & \text{if } s = \sigma \sigma' s' \text{ with } \sigma \neq \sigma'
\end{cases}
\end{align}
Stutter may also be induced in a string which is \textit{stutterless} (it does not contain stutter) by using the inverse of block compression, which will generate infinitely many strings\footnote{If the string $s$ features stutter, then $\bc^{-1}(s)$ will not contain any strings with a length shorter than $s$, including $\bc(s)$. To capture all possible \bc-equivalent strings, $s$ is block compressed before the inverse is applied.}:
\begin{align}\label{def:inverse-bc}
\bc^{-1}(\bc(s)) := \sigma_1^+ \sigma_2^+ \cdots \sigma_n^+ ~~~ \mbox{ if } \bc(s) = \sigma_1 \sigma_2 \cdots \sigma_n
\end{align}
For example:
\begin{align}\label{ex:inverse-bc}
	\bc^{-1}(\EventString{a|c}) = \{\EventString{a|c}, \EventString{a|a|c}, \EventString{a|c|c}, \EventString{a|a|c|c}, \ldots\}
\end{align}
Since these strings all block compress to the same string, they can be said to be equivalent under block compression. Specifically, strings $s$ and $s'$ are \textit{\bc -equivalent} iff $\bc(s) = \bc(s')$. This ability to generate infinitely many strings which have an equivalent interpretation allows for varying the length of a string as will be required in order to form a useful notion of superposition (see p. \pageref{def:initial-async-superposition}).

\paragraph{Superposition:}\label{para:str-op-sp}
In its most basic form, the \textit{superposition} $s \sp s'$ of two strings $s = \sigma_1\sigma_2\cdots\sigma_n$ and $s' = \sigma'_1\sigma'_2\cdots\sigma'_n$ of equal length $n$ is simply their component-wise union\footnote{The vocabulary of the resulting string is, as might be expected, the union of the vocabularies of the original strings: $\V_{s \sp s'} = \V_s \cup \V_{s'}$.}:

\begin{align}\label{def:superposition}
	\sigma_1\sigma_2\cdots\sigma_n \sp \sigma'_1\sigma'_2\cdots\sigma'_n := (\sigma_1 \cup \sigma'_1)(\sigma_2 \cup \sigma'_2)\cdots(\sigma_n \cup \sigma'_n)
\end{align}
For example:
\begin{align}\label{ex:superposition}
	\EventString{a|b|c} \sp{} \EventString{a|c|d} = \EventString{a|b,c|c,d}
\end{align}
This is easily extended to pairs of languages $L \sp L'$ by collecting the superpositions of strings of equal lengths in each language:
\begin{align}\label{def:lang-superposition}
	L \sp L' := \bigcup_{n \ge 0}\{ s \sp s' | s \in L \cap \Sigma^n, s' \in L' \cap \Sigma^n\}
\end{align}
The result $L \sp L'$ of superposing two languages $L$ and $L'$ is also a language, and if $L$ and $L'$ are regular languages (due to strings being finite models of Monadic Second-Order Logic and the theorem due to B\"uchi, Elgot, and Trakhtenbrot---see \cref{ssub:mso}, p. \pageref{def:mso-regularity}), then $L \sp L'$ is also regular (\citealp{Fernando2004}; \citealp[p. 126]{woods2017towards}). If $L$ is accepted by the finite automaton $\langle Q, (2^{\V_{L}})^*, (q \sta{\sigma}{\to} r), q_0, F \rangle$ and $L'$ is accepted by the finite automaton $\langle Q', (2^{\V_{L'}})^*, (q' \sta{\sigma'}{\to} r'), q'_0, F' \rangle$ then $L \sp L'$ is computed by a finite automaton composed of the automata accepting each $L$ and $L'$: $\langle Q \times Q', (2^{\V_L \cup \V_{L'}})^*, ((q, q') \sta{(\sigma \cup \sigma')}{\to} (r, r')), (q_0, q'_0), F \times F' \rangle$.% \nb{expand on this and its importance}

Using languages provides more flexibility than strings alone, since non-determinism can be accounted for through variations between strings within a language. For example, in \cref{ex:lang-superposition} below, the result of the superposition accounts for the alternate event sequences in the strings of the first input language.
\begin{align}\label{ex:lang-superposition}
	\{\EventString{a|b|c}, \EventString{a|c|b}\} \sp \{\EventString{a|c|d}\} = \{\EventString{a|b,c|c,d}, \EventString{a|c|b,d}\}
\end{align}
This may reflect a situation where there is uncertainty as to the correct order of events---in this case a language is useful to collect all of the possible alternatives, which can then be still be superposed with other languages. 

\paragraph{Asynchronous Superposition:}\label{para:str-op-sp-async}
In order to extend this operation further, it is necessary to remove the restriction that only strings of equal length may be superposed together. This is desirable so as to allow arbitrary numbers of events to appear in strings, and to superpose strings which may be of unknown length. For example, the operation in \cref{ex:unequal-lengths} below cannot be calculated, and even if the strings were instead singleton members of languages and those languages were superposed, the result would just be the empty set.
\begin{align}\label{ex:unequal-lengths}
	(\EventString{a|b} \sp \EventString{c|d}) \sp (\EventString{a|b|c} \sp \EventString{a|c|d})
	= \EventString{a,c|b,d} \sp \EventString{a|b,c|c,d}
	= undefined
\end{align}
To achieve this, \bc-equivalence is exploited and the \textit{inverse block compression} operation (see \cref{def:inverse-bc}, p. \pageref{def:inverse-bc}) is leveraged. Since, by inducing stutter in a string, infinitely many new strings of greater or equal length can be generated which are \bc-equivalent to the starting string, it is effectively possible to force a pair of strings to be of equal length.

So, the \textit{asynchronous superposition} $s \spasync s'$ of two strings $s$ and $s'$ is initially defined as the language obtained by applying block compression to the results of superposition between the languages which are respectively \bc-equivalent to each of $s$ and $s'$\footnote{Note that $\bc(s) = \bc(s') \Longleftrightarrow s \in \bc^{-1}(\bc(s'))$}:
\begin{align}\label{def:initial-async-superposition}
	s \spasync s' := \{\bc(s'') ~|~ s'' \in \bc^{-1}(\bc(s)) \sp \bc^{-1}(\bc(s'))\}
\end{align}
Now the strings in \cref{ex:unequal-lengths} can be superposed using asynchronous superposition, as in \cref{ex:async-superposition} below:
\begin{align}\label{ex:async-superposition}
	\EventString{a,c|b,d} \spasync \EventString{a|b,c|c,d} = \{&\EventString{a,c|a,b,c|b,c,d}, \EventString{a,c|a,b,d|b,c,d},\\&\EventString{a,c|a,b,c|a,c,d|b,c,d}, \EventString{a,c|b,c,d}\}\notag
\end{align}
However, one slightly problematic aspect of this definition is the fact that $\bc^{-1}$ maps from a string to an infinite language. While this is not an issue from a theoretical standpoint, since $\spasync$ collects the set of block compressed strings from the superposition of these languages, from a practical and computational standpoint anything infinite is inconvenient.

In order to tackle this back to something finite and to avoid generation of large amounts of redundant information, in \citet[p. 127]{woods2017towards} an upper bound of $n + n' - 1$\label{def:sp-upper-bound-length} is established for the maximum length of any string produced via asynchronous superposition $s \spasync s'$, where $n$ and $n'$ are the (nonzero) lengths of the strings $s$ and $s'$, respectively. This work additionally introduces the operation $pad_k$, which will perform inverse block compression on a string, but will only produce strings of a given length $k > 0$:
\begin{align}\label{def:padk}
	pad_k(\bc(s)) &:= \sigma_1^+ \sigma_2^+ \cdots \sigma_n^+ \cap \Sigma^k ~~~ \mbox{ if } \bc(s) = \sigma_1 \sigma_2 \cdots \sigma_n\\
	&=\{\sigma_1^{k_1}\sigma_2^{k_2}\cdots\sigma_n^{k_n} ~|~ k_1,\ldots,k_n \ge 1, \sum_{i = 1}^{n}k_i = k  \}\notag
\end{align}
The language produced by padding a string is a proper subset of the language produced by performing inverse block compression on that same string. For example, using the same string as \cref{ex:inverse-bc}:
\begin{align}\label{ex:padk}
	pad_3(\EventString{a|c}) = \{\EventString{a|a|c}, \EventString{a|c|c}\}
\end{align}
By using this padding operation in place of the inverse block compression in an updated definition of asynchronous superposition, setting $k$ to be the upper bound derived from the lengths of the input strings, the issue of going beyond finite sets is avoided without losing any of the power of using \bc-equivalence:
\begin{align}\label{def:async-superposition}
	% \text{\textit{For any strings }}s\text{\textit{ and }}s'\text{\textit{ with nonzero lengths $n$ and $n'$, respectively}}\notag\\
	s \spasync s' := \{\bc(s'') ~|~ s'' \in pad_{n+n'-1}(s) \sp pad_{n+n'-1}(s')\}
\end{align}
It's worth noting here that neither basic superposition $\sp$ nor asynchronous superposition $\spasync$ place any importance on the semantic content contained within the strings over which they operate. That is to say, they are entirely syntactical operations, and any meaningful information represented by a given string is liable to be lost once it has been superposed with some other string.

For instance, in \cref{ex:superposition}, the second of the operand strings has the event $c$ appearing in a box to the left of (before) the event $d$, whereas the result has $c$ and $d$ occurring in the same box together, which states that they were occurring at the same moment.

Similarly, in \cref{ex:async-superposition}, the first input string has events $a$ and $c$ appearing in the same box, while the second input string has $a$ appearing in a box before $c$. While this should seem like a contradiction which should not have viable results, the operation instead produces a set of four strings, the second and third of which are invalid according to \cref{impl:contiguous-events}, which states that a given fluent may not appear in non-contiguous string positions---in \EventString{a,c|a,b,d|b,c,d} the event $c$ occurs in positions 1 and 3 but not position 2, and in \EventString{a,c|a,b,c|a,c,d|b,c,d} the event $b$ occurs in positions 2 and 4 but not in position 3.

This stands in contrast to the concept of \bc-equivalence, where strings have an equal interpretation regardless of how much or how little stutter is present. By using superposition as it is currently presented, information is often, in fact, lost rather than gained when strings are combined. The next two operations, \textit{reduct} and \textit{projection}, aim to assist in the resolution of this issue.

\paragraph{Reduct:}\label{para:str-op-reduct}
It will be useful to be able to alter the vocabulary of a string---in particular, to shrink it---so as to control which events are mentioned. If a string contains, for instance, five events, but only two of these are relevant to the application, there is sense in being able to focus on those two.

As such, for any set $A$, the $A$-\textit{reduct} $\rho_A$ of a string $s=\sigma_1\sigma_2\cdots\sigma_n$ is defined as the componentwise intersection of $s$ with $A$ \citep{fernando2016regular,woods2018improving}:
\begin{align}
	\rho_A(\sigma_1\sigma_2\cdots\sigma_n) := (\sigma_1 \cap A)(\sigma_2 \cap A)\cdots(\sigma_n \cap A)
\end{align}
The resulting new string has a vocabulary of $A$, but the remaining fluents are still in the same relative positions to each other as in the original:
\begin{align}
	\V_{\rho_A(s)} &= A\\
	\forall a \in A~ (\llbracket P_a \rrbracket_{mod(s)} &= \llbracket P_a \rrbracket_{mod(\rho_A(s))})
\end{align}
For example, with the string $s = \EventString{{}|a,b|a,b,c|a,c,d|a,e|e|{}}$ and $A=\{a,d\}$, the $A$-reduct of $s$ is:
\begin{align}\label{ex:reduct}
	\rho_{\{a,d\}}(\EventString{{}|a,b|a,b,c|a,c,d|a,e|e|{}}) = \EventString{{}|a|a|a,d|a|{}|{}}
\end{align}
The resulting string in \cref{ex:reduct} contains only the events of interest (those mentioned in $A$), without loss of information. That is, the relative ordering of the events in the result string is the same as that in the input. The result string can additionally be block compressed to derive the simplest representation of the information it contains\footnote{In the case of \cref{ex:reduct,ex:bc-reduct}, the event $a$ \textit{contains} the event $d$, according to Allen's relations \citep{allen1983maintaining}---see also \Cref{fig:allens-pictorial}, p. \pageref{fig:allens-pictorial}.}:
\begin{align}\label{ex:bc-reduct}
	\bc(\EventString{{}|a|a|a,d|a|{}|{}}) = \EventString{{}|a|a,d|a|{}}
\end{align}
It's worth noting that for any pair of strings $s$ and $s'$ with equal length and disjoint vocabularies, the reduct of the result of basic superposition $s \sp s'$ with respect to each of the strings' vocabularies is equal to the string itself:
\begin{align}\label{impl:disjoint-vocab-reduct}
	\V_{s} \cap \V_{s'} = \emptyset \Longrightarrow \rho_{\V_{s}}(s \sp s') = s \text{ and } \rho_{\V_{s'}}(s \sp s') = s'
\end{align}
For example, with $s = \EventString{a|b}$ and $s' = \EventString{c|d}$, $\V_s = \{a,b\}$ and $\V_{s'} = \{c,d\}$:
\begin{align}\label{ex:disjoint-vocab-reduct}
	s \sp s' &= \EventString{a,c|b,d}\\
	\rho_{\{a,b\}}(\EventString{a,c|b,d}) &= \EventString{a|b}\notag\\
	\rho_{\{c,d\}}(\EventString{a,c|b,d}) &= \EventString{c|d}\notag
\end{align}
This shows how the information contained within $s$ and $s'$ is not lost in the superposition $s \sp s'$. For asynchronous superposition---which is built on basic superposition---this also holds true for each string\footnote{True when the string $s'' \in s \spasync s'$ has been block compressed after the reduct. For example: $\bc(\rho_{\V_s}(s'')) = s$.} in the result language.
\paragraph{Projection:}\label{para:str-op-projection}
This process is streamlined through the use of \textit{projection}, where the $A$-projection $\pi_A$ of a string $s$ is simply the block compressed reduct of $s$ relative to $A$:
\begin{align}
	\pi_A(s) := \bc(\rho_A(s))
\end{align}
It will be said that a string $s$ \textit{projects to} another string $s'$, $s \sqsupseteq s'$, if the $\V_{s'}$-projection of $s$ is equal to $s'$:
\begin{align}
	s \sqsupseteq s' \Longleftrightarrow  \pi_{\V_{s'}}(s) = s'
\end{align}
If $s$ projects to $s'$, then all of the information represented within $s'$ is also represented in $s$---$s$ effectively `contains' $s'$. Trivially, any block compressed string $s = \bc(s)$ will project to itself, since the reduct of $s$ with respect to its own vocabulary is $s$.

Borrowing from the examples in \cref{ex:reduct,ex:bc-reduct}, the temporal data represented by the string on the right hand side of \cref{ex:projects-to} is also contained in the string on the left hand side:
\begin{align}\label{ex:projects-to}
	\EventString{{}|a,b|a,b,c|a,c,d|a,e|e|{}} \sqsupseteq \EventString{{}|a|a,d|a|{}}
\end{align}
It is worth noting that, if strings $s$ and $s'$ share the same vocabulary, but are not equal, then neither can $s$ project to $s'$ nor $s'$ project to $s$---this scenario suggests that $s$ and $s'$ are incompatible, that they describe contradictory sequences of the same events:
\begin{align}\label{impl:cannot-project}
	\V_s = \V_{s'} \wedge s \neq s' \Longrightarrow \lnot (s \sqsupseteq s' \lor s' \sqsupseteq s)
\end{align}
A language $L$ can be said to project to another language $L'$ if every string $s \in L$ projects to every string $s' \in L'$:
\begin{align}
	L \sqsupseteq L' \Longleftrightarrow  \forall (s \in L)\forall (s' \in L')~ (s \sqsupseteq s')
\end{align}
This notion of projection is particularly useful, allowing for temporal reasoning when used in conjunction with the concept of \textit{analogous} strings (see \cref{para:analogous-strings}, p. \pageref{para:analogous-strings}). Particular events can be simply extracted from larger, more complex strings, and compared against reference to determine the relations between and ordering of the events of interest.

Importantly, projection will also be used to enrich the asynchronous superposition (see p. \pageref{def:async-superposition}) operation, injecting the currently-lacking `semantic-ness' by ensuring that all results of superposing a pair of strings project to each of their input strings.

\paragraph{Generate and Test:}\label{para:gen-and-test}
The predominant issue with asynchronous superposition $\spasync$ as it stands is that it needs not \textit{preserve projections}---that is, data can become lost or `corrupted' when combining strings. If a string generated by superposition does not project back to both of the strings that were superposed to generate it, then information has effectively been lost, as it has become impossible to return to the original event relation data. Only those result strings in $s \spasync s'$ which do project back can be said to be a valid result of the superposition, in terms of preserving the temporal information.

For example, in \cref{ex:async-superposition} (p. \pageref{ex:async-superposition}), none of the result strings contain the information that is represented in the input strings. This can be verified by testing whether any string in the result set projects to either of the inputs, which they do not:
\begin{center}
	\begin{tabular}[h!]{r c l | r c l}
		\EventString{a,c|a,b,c|b,c,d}&$\not\sqsupseteq$&\EventString{a,c|b,d}&\EventString{a,c|a,b,c|b,c,d}&$\not\sqsupseteq$&\EventString{a|b,c|c,d}\\
		\EventString{a,c|a,b,d|b,c,d}&$\not\sqsupseteq$&\EventString{a,c|b,d}&\EventString{a,c|a,b,d|b,c,d}&$\not\sqsupseteq$&\EventString{a|b,c|c,d}\\
		\EventString{a,c|a,b,c|a,c,d|b,c,d}&$\not\sqsupseteq$&\EventString{a,c|b,d}&\EventString{a,c|a,b,c|a,c,d|b,c,d}&$\not\sqsupseteq$&\EventString{a|b,c|c,d}\\
		\EventString{a,c|b,c,d}&$\not\sqsupseteq$&\EventString{a,c|b,d}&\EventString{a,c|b,c,d}&$\not\sqsupseteq$&\EventString{a|b,c|c,d}\\
	\end{tabular}
	\captionof{table}{Failed projections for \cref{ex:async-superposition}.}\label{tab:failed-projections}
\end{center}
In fact, every string in \cref{ex:async-superposition} shares the same vocabulary, so by \cref{impl:cannot-project}, it is not possible for any to project to any other.% It is, in general, possible for asynchronous superposition $s \spasync s'$ to produce results such that they can project to the strings $s$ and $s'$ which generated them (see \cref{ex:overgen} below).

In the case where the vocabularies of the two strings are identical, then none of the results will project to the original strings\footnote{This assumes that $s \neq s'$. In the case where $s = s'$, there is exactly one string $s'' \in s \spasync s'$ such that $s = s' = s''$.}. This implication follows from \cref{impl:cannot-project}, since the vocabulary of every string $s'' \in s \spasync s'$ is $\V_{s \cup s'} = \V_s = \V_{s'}$:
\begin{align}
	\V_s \cap \V_{s'} = \V_s = \V_{s'} &~\Longrightarrow~ \forall (s'' \in s \spasync s')~\lnot (s'' \sqsupseteq s \lor s'' \sqsupseteq s')\label{impl:voc-intersectionA}
\end{align}
The implication of \cref{impl:voc-intersectionA} can be used to avoid unuseful superpositions: since the operation $\spasync$ has the potential to generate a large number of new strings, which can become costly from a computational perspective, it is prudent to test ahead of time whether every generated string will be spurious and unable to project back---as seen in \Cref{tab:failed-projections}.

Conversely, if the vocabularies of the input strings are disjoint, then every resulting string will project back, due to \cref{impl:disjoint-vocab-reduct}:
\begin{align}
	\V_s \cap \V_{s'} = \emptyset &~\Longrightarrow~ \forall (s'' \in s \spasync s')~(s'' \sqsupseteq s \land s'' \sqsupseteq s')\label{impl:voc-intersectionB}
\end{align}
%\\\text{else} &~\Longrightarrow~ (\exists s'' \in s \spasync s')~s'' \sqsupseteq s \lor s'' \sqsupseteq s'\label{impl:voc-intersectionC}
% counter-example: |a|b|c| &* |d|c|b|
This can be seen by returning to the example in \cref{ex:disjoint-vocab-reduct}:%, with $s = \EventString{a|b}$ and $s' = \EventString{c|d}$, $\V_s = \{a,b\}$ and $\V_{s'} = \{c,d\}$:
\begin{align}\label{ex:spasync-disjoint-vocab}
	\EventString{a|b} \spasync \EventString{c|d} = \{\EventString{a,c|a,d|b,d}, \EventString{a,c|b,d}, \EventString{a,c|b,c|b,d}\}
\end{align}
\begin{center}
	\begin{tabular}[h!]{r c l | r c l}
		\EventString{a,c|a,d|b,d}&$\sqsupseteq$&\EventString{a|b}&\EventString{a,c|a,d|b,d}&$\sqsupseteq$&\EventString{c|d}\\
		\EventString{a,c|b,d}&$\sqsupseteq$&\EventString{a|b}&\EventString{a,c|b,d}&$\sqsupseteq$&\EventString{c|d}\\
		\EventString{a,c|b,c|b,d}&$\sqsupseteq$&\EventString{a|b}&\EventString{a,c|b,c|b,d}&$\sqsupseteq$&\EventString{c|d}
	\end{tabular}
	\captionof{table}{Preserved projections of \cref{ex:spasync-disjoint-vocab}.}\label{tab:spasync-disjoint-vocab}
\end{center}
% ---as in \cref{impl:voc-intersectionA}---or uninformative---as in \cref{impl:voc-intersectionB}
However, in general, the vocabularies of the input strings to asynchronous superposition may overlap without being equal. It might be assumed that, in this case, some number greater than 0 but less than all of the resulting strings may project back---however, this assumption does not hold in fact. A counter-example is readily found:
\begin{align}\label{ex:spasync-overlap-vocab}
	\EventString{a|b} \spasync \EventString{b|c} = \{\EventString{a,b|a,c|b,c}, \EventString{a,b|b,c}, \EventString{a,b|b|b,c}\}
\end{align}
\begin{center}
	\begin{tabular}[h!]{r c l | r c l}
		\EventString{a,b|a,c|b,c}&$\not\sqsupseteq$&\EventString{a|b}&\EventString{a,b|a,c|b,c}&$\not\sqsupseteq$&\EventString{b|c}\\
		\EventString{a,b|b,c}&$\not\sqsupseteq$&\EventString{a|b}&\EventString{a,b|b,c}&$\not\sqsupseteq$&\EventString{b|c}\\
		\EventString{a,b|b|b,c}&$\not\sqsupseteq$&\EventString{a|b}&\EventString{a,b|b|b,c}&$\not\sqsupseteq$&\EventString{b|c}
	\end{tabular}
	\captionof{table}{Failed projections for \cref{ex:spasync-overlap-vocab}.}\label{tab:failed-projections-overlap}
\end{center}
Thus it is necessary to test whether each result of a superposition is valid when neither \cref{impl:voc-intersectionA} nor \cref{impl:voc-intersectionB} are true. This approach of generating then testing was initially taken in \citet{woods2017towards}\footnote{Albeit, the testing algorithm used there was based on matching string positions rather than projections.} to ensure that only valid strings were finally produced. Yet, this is not without issue either---consider the following examples:
\begin{align}
	\Before{a}{b} \spasync \Before{b}{c} = \{&\EventString{{}|a,b|{}|b,c|{}}, \EventString{{}|b|a,b|{}|b,c|{}}, \EventString{{}|b|a|{}|b,c|{}},\label{ex:overgen}\\
	&\EventString{{}|b|a|c|b,c|{}}, \EventString{{}|b|a|c|b|{}}, \ldots\}\notag\\
	\EventString{{}|a|b|c|{}} \spasync \EventString{{}|d|c|b|{}} = \{&\EventString{{}|a,d|a,c|b|c|{}}, \EventString{{}|d|a,d|b,d|b,c|{}}, \EventString{{}|d|c|b|a,b|b|c|{}},\label{ex:overgen-none}\\
	&\EventString{{}|a|b|c|{}|d|c|b|{}}, \EventString{{}|a|b|c,d|c|b,c|b|{}}, \ldots\}\notag
\end{align}
The language result of \cref{ex:overgen} contains 270 strings, and in fact, only one of these will project back to both of the input strings: namely, \EventString{{}|a|{}|b|{}|c|{}}. This makes an intuitive sense, since the inputs are \cBefore{a}{b} and \cBefore{b}{c}, and this result string is the only possibility where the linear ordering of the events $a$, $b$, and $c$ is retained.

Using projection to test each of the generated strings in \cref{ex:overgen} and rejecting those which fail to project back to the inputs will produce the singular correct result, though it is rather inefficient: 269 (over 99\%) of the generated results must be discarded in this example. The language result of \cref{ex:overgen-none} contains 257 strings, and this time 100\% of them must be discarded: not one of the results projects back to both of the inputs. Clearly, the computational effort required to produce so many non-viable strings is entirely wasted, and so a modified approach is required to avoid this issue.

In \citet{woods2018improving}, a new version of superposition is defined which integrates projection-based testing into the generation process. This prevents problematic strings from ever being produced, improving on the efficiency of asynchronous superposition.

\paragraph{Vocabulary-Constrained Superposition:}\label{para:str-op-vc-sp}
In order to define \textit{vocabulary-constrained superposition} $\spvc$\!, begin by fixing an infinite set of fluents $\Theta$. Then for any string $s$, the set of finite subsets of $\Theta$ is $Fin(\Theta)$ such that $s \in Fin(\Theta)^*$. Given a pair of finite subsets of $\Theta$, $\Sigma \in Fin(\Theta)$ and $\Sigma' \in Fin(\Theta)$, an operation $\spsigma: (Fin(\Theta)^* \times Fin(\Theta)^*) \rightarrow 2^{Fin(\Theta)^*}$ is defined, mapping a pair of strings $s$ and $s'$ to a language $s \spsigma s'$ as follows, where $\epsilon$ is the empty string (of length 0)\footnote{It follows from this definition that any string in $s \spsigma{} s'$ will have a length less than $n+n'$ where $n$ and $n'$ are the lengths of $s$ and $s'$, respectively, which is the same upper bound found in \citet{woods2017towards} (see p. \pageref{def:sp-upper-bound-length}).}:

\begin{subequations}\label{def:sp-sigma}
	\begin{align}
		\epsilon \spsigma \epsilon &:= \{\epsilon\}\\
		\epsilon \spsigma s &:= \emptyset ~~~~\text{ if } s \neq \epsilon\\
		s \spsigma \epsilon &:= \emptyset ~~~~\text{ if } s \neq \epsilon
	\end{align}
	and with $\sigma \in Fin(\Theta), \sigma' \in Fin(\Theta)$

	\begin{equation}
		\sigma s \spsigma \sigma' s' :=
		\begin{cases}\label{def:sp-sigmaD}
			~\{(\sigma \cup \sigma') s'' ~|~ s'' \in L(\sigma, s, \sigma', s', \Sigma, \Sigma')\} &\text{if } \Sigma \cap \sigma' \subseteq \sigma \mbox{ and } \Sigma' \cap \sigma \subseteq \sigma'\\
			~\emptyset &\text{otherwise}
		\end{cases}
	\end{equation}
	where
	\begin{equation}
		L(\sigma, s, \sigma', s', \Sigma, \Sigma') := (\sigma s \spsigma s') ~\cup~ (s \spsigma \sigma' s') ~\cup~ (s \spsigma s')
	\end{equation}
\end{subequations}
If $\Sigma = \Sigma' = \emptyset$, then the condition in the first case of \cref{def:sp-sigmaD} ($\Sigma \cap \sigma' \subseteq \sigma \mbox{ and } \Sigma' \cap \sigma \subseteq \sigma'$) holds vacuously, and $\spsigma$ becomes effectively identical to asynchronous superposition $\spasync$\!. Otherwise, this condition can be used to eject those strings which do not project back to both $s$ and $s'$, according to Proposition 1 and Corollary 2 in \citet[p. 81]{woods2018improving}, reproduced below.

\noindent
\textbf{Proposition 1.} {\sl For all $\Sigma \in Fin(\Theta), \Sigma' \in Fin(\Theta)$ and $s \in Fin(\Theta)^*, s' \in Fin(\Theta)^*$, $s \spsigma s'$ selects those strings from asynchronous superposition $s \spsigma[\emptyset, \emptyset] s'$ which project to both the $\Sigma$-projection of $s$ and the $\Sigma'$-projection of $s'$}:
\begin{align}
s \spsigma s' ~=~ \{s'' \in s \spsigma[\emptyset, \emptyset] s' ~|~
s'' \sqsupseteq \pi_{\Sigma}(s) \wedge s'' \sqsupseteq \pi_{\Sigma'}(s')\}
\end{align}

\noindent
\textbf{Corollary 2.} {\sl For all $s \in Fin(\Theta)^*, s' \in Fin(\Theta)^*$ such that $s$ and $s'$ are stutterless, if $\Sigma = \V_s$ and $\Sigma' =\V_{s'}$, then $s \spsigma s'$ selects those strings from asynchronous superposition $s \spsigma[\emptyset, \emptyset] s'$ which project to $s$ and $s'$}:
\begin{align}
s \spsigma s' = \{s''\in s \spsigma[\emptyset, \emptyset] s' ~|~ s'' \sqsupseteq s \wedge s'' \sqsupseteq s'\}
\end{align}
\doublespacing
According to Corollary 2, \textit{vocabulary-constrained superposition} $\spvc$ can be used to preserve temporal information under projection during superposition:
\begin{align}\label{def:vc-superposition}
s \spvc s' :=  s \spsigma[\V_s, \V_{s'}] s'
\end{align}
Now, this new form of superposition can be used for the same example as \cref{ex:overgen}, and only the one valid result will be produced:
\begin{align}
	\Before{a}{b} \spvc \Before{b}{c} = \{\EventString{{}|a|{}|b|{}|c|{}}\}
\end{align}
Note that this result is still a language (a set of strings), and that there may be more than one string in this language, depending on the input strings. Where each input's vocabulary has a cardinality of 2, and the intersection of their vocabularies has cardinality of 1, then the number of result strings from the vocabulary-constrained superposition of the inputs corresponds with the transitivity table in \citet[Fig. 4]{allen1983maintaining}. For example, \cref{ex:spvc-allen-transitivity} shows the string \cOverlaps{a}{b} superposed with \cDuring{b}{c}, and according to Allen's transitivity table there should be three results, corresponding to \cDuring{a}{c}, \cOverlaps{a}{c}, and \cStarts{a}{c}, and in fact this is the result shown by \Cref{tab:spvc-allen-transitivity-projections}.
\begin{align}\label{ex:spvc-allen-transitivity}
	\Overlaps{a}{b} \spvc \During{b}{c} = \{&\EventString{{}|c|a,c|a,b,c|b,c|c|{}}, \EventString{{}|a|a,c|a,b,c|b,c|c|{}},\\
	&\EventString{{}|a,c|a,b,c|b,c|c|{}}\}\notag
\end{align}
\begin{center}
	\begin{tabular}[h!]{r c l}
		\EventString{{}|c|a,c|a,b,c|b,c|c|{}}&$\sqsupseteq$&\During{a}{c}\\
		\EventString{{}|a|a,c|a,b,c|b,c|c|{}}&$\sqsupseteq$&\Overlaps{a}{c}\\
		\EventString{{}|a,c|a,b,c|b,c|c|{}}&$\sqsupseteq$&\Starts{a}{c}
	\end{tabular}
	\captionof{table}{Projections of \cref{ex:spvc-allen-transitivity} matching Allen's transitivities.}\label{tab:spvc-allen-transitivity-projections}
\end{center}
In \citet[p. 82]{woods2018improving} some simple benchmark tests were run, comparing the time (in milliseconds\footnote{The mean time of 1001 runs is given. The testing environment was Node.js v10.0.0 (64-bit) on Ubuntu 16.04 using an Intel i7-6700 CPU with 16GB of memory.}) taken to compute the superpositions of a number of pairs of strings, using each of asynchronous superposition (generating then testing) and vocabulary-constrained superposition (testing while generating). These figures indicate a notable increase in the efficiency of time to calculate the correct results in using vocabulary-constrained superposition. Each of the strings in \Cref{tab:allen-rels-strings} (p. \pageref{tab:allen-rels-strings}) is superposed with itself and each of the others (e.g. \textit{before} with \textit{before}, \textit{before} with \textit{after}, \textit{before} with \textit{meets}, ..., \textit{finished by} with \textit{started by}, \textit{finished by} with \textit{finished by}), while varying the vocabularies of the operand strings as follows: first, both strings had the same vocabulary $\{a, b\}$; second, the strings shared one fluent in common, $\{a, b\}$ and $\{b, c\}$; finally, the strings had disjoint vocabularies, $\{a, b\}$ and $\{c, d\}$. A fragment of these tests is shown in \Cref{tab:speedtest}.
\begin{center}
	\begin{tabular}{c|c|c|c}
	& $\triangle = *$ & $\triangle = v\!c$ & Decrease in time\\
	\hline
	\Before{a}{b} $\&_{\triangle}$ \After{a}{b} & 0.3207ms &  0.0180ms & 94.39\%\\
	$\vdots$ & $\vdots$ & $\vdots$ & $\vdots$\\
	\Before{a}{b} $\&_{\triangle}$ \After{c}{b} & 0.3207ms & 0.0659ms & 79.45\%\\
	$\vdots$ & $\vdots$ & $\vdots$ & $\vdots$\\
	\Before{a}{b} $\&_{\triangle}$ \After{d}{c} & 22.4016ms & 5.3616ms & 76.07\%\\
	$\vdots$ & $\vdots$ & $\vdots$ & $\vdots$
	\end{tabular}
	\captionof{table}{Fragment of speed comparison between $\spasync$ and $\spvc$.}
	\label{tab:speedtest}
\end{center}
The mean percentage decrease for all pairs with identical vocabularies using vocabulary-constrained superposition was 74.28\%. With one shared fluent, the mean decrease was 34.11\%, and with no fluents in common, it was 64.51\%.

The main operations for string-based finite-state temporality have now been defined: block compression, reduct, projection, and (vocabulary-constrained) superposition. These operations may be leveraged in a number of ways in order to create useful applications.

\subsection{Applications}\label{sub:applications}
% The following will outline some of the ways that the use of string-based FST can applied to problems in NLP.
Having discussed how to create and manipulate strings which represent temporal data---the linear order and inter-relations of a set of events---the following will explore some of the possible applications of this technology, including the ability to create timelines from annotated texts, to verify that the narrative structure of an annotated text is internally consistent with regards to the relations it depicts between its events, and to infer information not explicitly stated in an annotated text based on the event relations that can be extracted. These abilities can be used as part of tooling for the automatic aiding of creation of temporal annotation in new texts, for automated generation of summaries of a text, or fact-checking via corroboration of event sequences between sources.

Additionally, strings can be used as a tool for other, related applications which deal with sequential data through the use of external constraints. An example of such an application is given via a solution to a variant of the well-known Zebra Puzzle which models temporal constraints (that is, scheduling constraints) rather than spatial ones.
% Permitting external constraints to be placed on strings can allow for further uses, including in areas such as scheduling problems.\nb{flesh this out(?)}

\subsubsection{Timelines from Texts}\label{ssub:timelines}
Strings, as entities which are comprised of sequential components representing temporal data, have an intuitive comparison to a traditionally linear view of time. That is, that events which have not yet occurred are ahead of us, and events which occurred in the past are behind us---although not all languages or cultures perceive time in this way, it is common cross-linguistically to use some spatial reference points when discussing temporality
% \nb{wiki: timeline;\\\url{https://onlinelibrary.wiley.com/doi/full/10.1111/cogs.12804}\\\url{https://sites.williams.edu/thea228/files/2011/09/Cartographies-of-Time.pdf}\\\url{https://books.google.ie/books?hl=en&lr=&id=DqWqKVzipToC&oi=fnd&pg=PA8&dq=Rosenberg,+Daniel+(2010),+Cartographies+of+Time:+A+History+of+the+Timeline,+Princeton+Architectural+Press&ots=m2BG84e69h&sig=uykTBRs2rIPfmTf6rW0jtWlSBNo&redir_esc=y#v=snippet&q=spatial&f=false}\\\url{https://www.jstor.org/stable/pdf/1343108.pdf?casa_token=JdY5vEGKYgMAAAAA:ykEfao2S22Jekocs5fWyNixqQwM-YD6Os4r5wT-1cNFsLsk7NkAMcI9Y-7Z6GQQNGDgXJpGPpWXRAHrRrV_EQGW-rAR7RlTh9gkBUEdqsem31aBvyKvk}}
(even absent vision, see \citet{bottini2015space})---\citet[pp. 42--43]{lakoff2008metaphors}, for example, discusses the metaphor of ``Time is a moving object'', where the future is perceived as moving towards us, and \citet[p. 542]{mitchell1980spatial} argues that ``we literally cannot `tell time' without the mediation of space''. Often, time is visualised as a line which travels along a three-dimensional axis, with events appearing as points or spans along the line, although according to \citet[p. 14]{rosenberg2013cartographies}, the particular modern definition of a \textit{timeline} as a ``single axis and a regular, measured distribution of dates ... is not even 250 years old''.

Regardless of the spatial orientation or directionality, this perception of time as coming from somewhere and going to somewhere else maps well to a sequential representation. The core concept being that, if two events exist at different moments in time, then they can be put in some kind of spatial ordering corresponding to the temporal ordering. The strings described in \cref{sub:strings} are used to model sequences of events in such a way that they can be read in a manner similar to a series of snapshots or film reel, or like the panels of a comic: each `image' or moment of time features all of the events which are occurring at that time (relative to some fixed vocabulary of events).

An example of timelines in use is via Gantt charts, also known as harmonograms, which are a kind of bar chart that---like strings---display events over time. They are often used as a visual aid to show project schedules or similar temporal data \citep{kumar2005effective}. %\citep{gantt1910work,kumar2005effective}.
The vertical axis shows the events mentioned in the chart, and the horizontal axis represents time intervals, with the width of the bars showing the duration of each event, and the beginnings and endings also illustrated by the bars' horizontal placement. See, for example, \Cref{ex:gantt-chart} below.

\begin{center}
	\begin{figure}[h!]
		\centering
		\includegraphics[width=0.9\textwidth]{images/gantt-chart-example.png}
		\captionof{figure}{A Gantt chart featuring six events.}\label{ex:gantt-chart}
	\end{figure}
\end{center}
A chart like this is a useful visual display tool, though it can become a little unwieldy with large numbers of events, and there have been claims that they should be left behind in the field of project management \citep{maylor2001beyond}. Compare the same temporal data in \Cref{ex:gantt-chart} shown in a single string in \cref{ex:gantt-string}:
\begin{align}\label{ex:gantt-string}
	\EventString{{}|a|a,b|a,b,c|b,c|b|b,d|b,d,e|e|f|{}}
\end{align}
No matter how many events may be of interest, if all beginnings, endings, and durations are known---that is, the \textit{temporal closure} has been calculated---such as they could be displayed in a Gantt chart, then they can be represented in a single string, with reducts and projections available in order to focus an analysis on a subset of events if desired, demonstrating a compact method of representing the timeline. However, in general, natural language texts do not provide enough precise temporal detail to determine all of this data. Discourse is typically somewhat vague and relies heavily upon the use of context to determine and precisely temporally locate events. As a result, it's often not possible to immediately derive the temporal closure just from a text, even an annotated text such as one of the documents of the TimeBank \citep{pustejovsky2006timebank} corpus, which is marked up to show the events, times, and temporal relations found within---see also \cref{ssub:timeml}. With that said, even if a single, unified timeline cannot be constructed due to ambiguity, the information in one of these annotated texts may still be extracted and used to build strings, which can provide a visual picture of the content of the document, which may reveal insights not obvious when looking at the text alone.

A string, like a timeline, can be thought of as a simplistic narrative, depicting a world by the events which take place in it. A language of strings, then, is a set of alternate timelines, of parallel worlds. Each string in a language represents a different possible world, describing alternate sequences of the events which took place, and each of which might be considered equally probably to be \textit{veridical}---the actually true world, the correct sequence of events which occurred in reality---without further data and constraints. This allows strings as a framework to capture the ambiguity that very often arises from interpreting the temporal information that is derived from a natural language text. For example, given the information that some event $a$ meets some other event $b$, and $b$ is during some third event $c$, Allen's transitivities---see \Cref{tab:allen-trans-table}, p. \pageref{tab:allen-trans-table}---dictate that the relation between $a$ and $c$ can be one of overlaps, during, or starts. These first two relations $a$ meets $b$ and $b$ during $c$ may be represented by the strings \Meets{a}{b} and \During{b}{c}, respectively---see \Cref{tab:allen-rels-strings}, and in fact, the three possible relations between $a$ and $c$ are found by projection relative to $\{a,c\}$ over the result found from superposing thsee two strings, as in \cref{ex:parallel-worlds} below:
\begin{subequations}\label{ex:parallel-worlds}
	\begin{align}
		\Meets{a}{b} \spvc \During{b}{c} = &\{\EventString{{}|c|a,c|b,c|c|{}}, \EventString{{}|a|a,c|b,c|c|{}}, \EventString{{}|a,c|b,c|c|{}}\}\label{ex:parallel-worlds-a}\\
		\text{Under projection to }\{a,c\} = &\{\EventString{{}|c|a,c|c|{}}, \EventString{{}|a|a,c|c|{}}, \EventString{{}|a,c|c|{}}\}\label{ex:parallel-worlds-b}
	\end{align}
\end{subequations}
In \cref{ex:parallel-worlds-a}, the language result of the superposition contains three strings, or three parallel worlds depicting the relational and ordering information that is known about the events $a$, $b$, and $c$, with \cref{ex:parallel-worlds-b} showing the language under projection, which gives three strings depicting the three possible relations between $a$ and $c$---see \Cref{tab:allen-rels-strings}. Unless some further data becomes available constraining these possibilities, there is no way to tell which of the three results of \cref{ex:parallel-worlds-a} is the correct timeline, and so the language as a whole is considered valid. It is important to be wary, though, as too many of these parallel worlds can become unwieldy, just as Gantt charts can become difficult to keep track of if too large a number of events are depicted. One way to avoid an explosion in the number of timelines to keep track of is to avoid superpositions where the inputs do not share some vocabulary, or where the number of strings in the result would exceed some predetermined limit. Another possibility is to alter the granularity of the string, using semi-intervals \citep{Freksa1992} rather than plain intervals, although this does have its own tradeoffs---see \cref{ssub:incomplete}.

Conversely to considering a whole language as valid, occasionally an annotated text may feature inconsistencies, in that relations may be indicated which are impossible for one reason or another, whether that comes from human error on the part of a manual annotator, a poor machine annotation, or simply a text whose narrative is inconsistent. Whatever the source, it is important to be congnisant of these potential issues, as they are likely to lead to at least partially incorrect conclusions being drawn about the timeline of the narrative. Using vocabulary-constrained superposition, these inconsistencies can be discovered as the result of superposing a pair of strings which represent incompatible temporal data will be an empty set. For example:
\begin{align}
	\Before{a}{b} \spvc \During{a}{b} = \emptyset\label{ex:str-inconsistency}
\end{align}
This will always be the case, no matter how many events may appear in either string, and thus the relations between intervals appearing in a string effectively become constraints, which the string models: each string in \cref{ex:str-inconsistency} represents a different constraint, which are incompatible with each other---see also \cref{ssub:zebra}. When doing superposition of languages, if all of the strings from the first language are incompatible with all of the strings from the second, the result will also be an empty language---although, if some of the strings are compatible, then those superpositions are returned, as in \cref{ex:lang-inconsistencies} below, where the only strings which are mutually consistent are the second string of the first language and the first string of the second:
\begin{align}
	\{\EventString{{}|a|{}|b|{}|c|{}}, \EventString{{}|a|b|{}|c|{}}\} \spvc \{\EventString{{}|d|a|b|{}}, \EventString{{}|b,d|a|{}}\} = \{\EventString{{}|d|a|b|{}|c|{}}\}\label{ex:lang-inconsistencies}
\end{align}
A point of interest is that the thirteen interval relations given by the interval algebra of \citet{allen1983maintaining} fall out of the vocabulary-constrained superposition of a pair of strings which each feature a single and different finitely-bounded event\footnote{The superposition of more than two unconstrained intervals in this manner creates a rapidly expanding number of strings. Three intervals gives 409 strings, and by six intervals it has already exceeded three hundred million strings---see \citet[p. 129]{woods2017towards} and the full sequence in \url{https://oeis.org/A055203} \citep{oeisA055203}. This is obviously an excessive amount of information to process, and so generally superposition is to be avoided where there are no constraints between the intervals in one string and the other---that is, where there is no shared vocabulary between the strings to be superposed.}:
\begin{align}
	\mathcal{AR} := \{<,>,m,mi,o,oi,d,di,s,si,f,fi,=\}\label{def:allen-rel-set}\\
	\EventString{{}|v|{}} \spvc \EventString{{}|v'|{}} = \{\mathcal{S}_{\bigcdot}(v,v') ~|~ \bigcdot \in \mathcal{AR}\}
\end{align}
Each string $\mathcal{S}_{\bigcdot}(v,v')$ of the result set features one relation $v \bigcdot v'$, as shown in \Cref{tab:allen-rels-strings}, reproduced from \citet[p. 79, Table 1]{woods2018improving}.
\begin{center}
	\begin{tabular}[h!]{|c|c|c||c|c|c|}
		\hline
		${\bigcdot}$ & $v~{\bigcdot}~v'$ & $\mbox{$\cal{S}$}_{\bigcdot}(v,v')$ & ${\bigcdot}^{-1}$ & $v~{\bigcdot}^{-1}~v'$ & $\mbox{$\cal{S}$}_{{\bigcdot}^{-1}}(v,v')$ \\
		\hline
		$<$ & $v$ before $v'$ & \Before{v}{v'} & $>$ & $v$ after $v'$ & \After{v}{v'} \\
		m & $v$ meets $v'$ & \Meets{v}{v'} & mi & $v$ met by $v'$ & \iMeets{v}{v'} \\
		o & $v$ overlaps $v'$ & \Overlaps{v}{v'} & oi & $v$ overlapped by $v'$ & \iOverlaps{v}{v'} \\
		d & $v$ during $v'$ & \During{v}{v'} & di & $v$ contains $v'$ & \iDuring{v}{v'} \\
		s & $v$ starts $v'$ & \Starts{v}{v'} & si & $v$ started by $v'$ & \iStarts{v}{v'} \\
		f & $v$ finishes $v'$ & \Finishes{v}{v'} & fi & $v$ finished by $v'$ & \iFinishes{v}{v'} \\
		= & $v$ equals $v'$ & \Equals{v}{v'} & & &\\
		\hline
	\end{tabular}
	\captionof{table}{Allen interval relations in strings.}\label{tab:allen-rels-strings}
\end{center}
Using the notion of analogous strings (see \cref{para:analogous-strings}, p. \pageref{para:analogous-strings}), any block compressed string which has a vocabulary of cardinality 2 can be compared to the strings in \Cref{tab:allen-rels-strings}. If such a string $s \sim \mathcal{S}_{\bigcdot}(v,v')$ for some $\bigcdot \in \mathcal{AR}$, then the events appearing in $s$ can also be said to be related by $\bigcdot$. For example, $\iFinishes{c}{d} \sim \mathcal{S}_{\text{fi}}(v,v')$, and thus the relation between the events $c$ and $d$ is `$c$ finished by $d$'.

This is easily extended beyond strings featuring just two fluents. The relations between the events appearing in the string $s = \EventString{{}|a|{}|b|{}|c|{}}$ can be determined by taking its block compressed reduct relative to the subsets of the vocabulary which have cardinality 2---in this case, $a$ is before $b$, $b$ is before $c$, and $a$ is before $c$. Once these relations have been calculated, the relations between the events in another string $s' = \EventString{{}|d|{}|e|{}|f|{}}$ are immediately available on analogy $s \sim s'$.

While this is a relatively simple example, it can be extended for strings featuring any arbitrary number of events, and perhaps more usefully it can be used to shortcut superpositions and other string operations. For instance, given the pair of strings $s = \Overlaps{a}{b}$ and $s' = \During{b}{c}$, the vocabulary-constrained superposition is calculated as in \cref{ex:spvc-allen-transitivity}:
\begin{align}\label{ex:sp-analogy}
	s \spvc s' = \{&\EventString{{}|c|a,c|a,b,c|b,c|c|{}}, \EventString{{}|a|a,c|a,b,c|b,c|c|{}},\\
	&\EventString{{}|a,c|a,b,c|b,c|c|{}}\}\notag
\end{align}
Now, given two more strings $t = \Overlaps{x}{y}$ and $t' = \During{y}{z}$ such that $s \sim t$ and $s' \sim t'$, the generated results of superposing $t$ and $t'$ will also be analogous to the results in \cref{ex:sp-analogy}, and so there is no need to calculate $t \spvc t'$. Since the strings are analogous, there is a bijective mapping between the vocabularies $f: (\V_{s} \cup \V_{s'}) \leftrightarrow (\V_{t} \cup \V_{t'})$, and applying this function $f$ to the results in \cref{ex:sp-analogy} gives the same result as calculating the superposition $t \spvc t'$:
\begin{align}\label{ex:sp-analogy-equal}
	f(s \spvc s') = \{&\EventString{{}|z|x,z|x,y,z|y,z|z|{}}, \EventString{{}|x|x,z|x,y,z|y,z|z|{}},\\
	&\EventString{{}|x,z|x,y,z|y,z|z|{}}\}\notag\\
	=~&t \spvc t'
\end{align}
Again, this is a small example, but with larger numbers of events and more complex strings, leveraging the power of analogous strings has the potential to massively reduce the computational cost to calculate superpositions.

By extracting the relations from annotated text as strings and combining them using superposition, a timeline can be built up, which can assist an annotator or other reader in visualising the overall temporal structure of the text. It allows them to check for consistency and also provides a basis for interpreting the narrative in terms of the events that it describes.
% \subsubsection{Inferring New Information}\label{ssub:inferring}
% A text which has been manually annotated for temporal information will typically mark up the most important relations between events, at least as the annotator saw them. By extracting and making plain the timeline, any inconsistencies are immediately revealed, whether these are the fault of a problematic document or human error. Further, relations between times and events which were not previously obvious can be spelled out clearly.

% This can be extended over multiple documents by keeping track of the semantics of the elements in a string---being able to map between the symbols and their meanings---and linking documents which refer to the same events. This can provide a way for checking consistency between reports which originate from different sources, or a way for augmenting an existing timeline with new data.

% Additionally, conclusions can be drawn about whether certain inferences can be made based on existing data by determining the gap between premise strings and strings representing the question statement. That is, deriving the information that could be added to the premises to make the conclusion consistent with the known information (see \cref{ssub:residuals}).

\subsubsection{Constraints and Scheduling (Zebra Puzzle)}\label{ssub:zebra}
Scheduling as a general concept---whereby tasks or events are allocated an ordering according to some set of rules or constraints---is a multi-faceted problem that has been a subject of research for many years \citep{manne1960job,applegate1991computational,pinedo1992scheduling,gong2018memetic}. Available resources must be taken into account, and often it is desirable to find the most efficient way to order the events which are to appear in the schedule so as to minimise the amount of time required for all relevant events to finish. Here it will be shown how the strings described in \cref{sub:strings} can be applied to some scheduling and scheduling-like tasks.

The Zebra Puzzle, also sometimes known as Einstein's Riddle is a logic problem, involving solving a set of clues in order to assign a number of properties to a set of individuals. In the original puzzle, there are five houses in a street, each of a different colour, and in each lives a person of different nationality, who drinks a different beverage, smokes a different brand of cigarettes, and owns a different pet. The puzzle provides clues as to which house contains which set of all of the properties, except that the person who owns a zebra as a pet is not specified, and must be deduced by arranging all of the other elements into their correct houses so that the zebra's home can be determined by process of elimination. This may seem to be a very distinct problem from scheduling, but in fact, there are a number of parallels. Both problems can be viewed as constraint satisfaction problems, and although the Zebra Puzzle concerns spatial constraints, it is not a large stretch to model the street as a sequence of houses, and thus be able to use strings to represent the clues which can be superposed to solve the riddle. If a `house' is conceptualised as a set which contains as elements all of the properties associated with that house---for example, $\{red, english, zebra, coffee, kools\}$---and the street is a sequence of these sets, then the street can be depicted as a string. The left and right spatial relations are thought of in the same way as the previous and subsequent boxes in a string.

To make this connection clearer, below is presented a variant of the puzzle using clues pertaining to temporal relations instead of spatial ones. The clues to the puzzle are as follows in \Cref{tab:temporal-zebra-clues}:
\begin{center}
	\begin{tabular}[h!]{|l|}
		\hline
		There are five weekdays.\\
		The foggy day is mild.\\
		I am tired on the warm day.\\
		There is a traffic jam on the overcast day.\\
		It is cold on the day with little traffic.\\
		It is overcast the day after it snows.\\
		I'm sad the day that I'm reading.\\
		It rains the day I have printing to do.\\
		The traffic is average in the middle of the week.\\
		It's freezing at the beginning of the week.\\
		The stapling is done the day before or the day after I'm happy.\\
		Printing happens the day before or the day after I'm angry.\\
		There is a lot of traffic the day that filing happens.\\
		Shredding happens the day it's hot.\\
		It's freezing the day before or the day after the weather is clear.\\
		\hline
	\end{tabular}
	\captionof{table}{Temporal Zebra puzzle clues in English.}\label{tab:temporal-zebra-clues}
\end{center}
Using these clues, it should be possible to answer these questions:
\begin{itemize}
	\item What day is there no traffic?
	\item What day am I curious? 
\end{itemize}
The puzzle makes three assumptions: first, that the values as presented are \textit{discrete} rather than continuous---`freezing' and `cold', for example, are similar concepts (indeed, if the weather is freezing, then it is also cold), but for the purposes of this puzzle they are treated as being separate and unrelated; second, that each value lasts for only one day---if it rains on one of the days, it will not rain on any other; and third, each day only has one value per attribute---for example, only one task can be performed on any given day. Effectively, recalling that since the strings model inertial worlds, each day contains event-like statives which are treated as having a duration of the entire day. The particular values that appear may make it seem a little absurd---it is rather unlikely that a person would be doing something like printing for a full day, and then immediately begin stapling for another full day, and so on---however, they should suffice for the sake of the example.

For each of the given clues, a constraint can be constructed from one or more strings. Additional constraints can be formed to represent the external assumptions, and by superposing these constraints together, a solution to the puzzle can be found. The strings corresponding to each clue are given in \Cref{tab:zebra-clue-strings}.

For most of the clues, the constraint is formed as a set of a few strings: when event $a$ and $b$ occur on the same day, they will appear in the same box, but it's unknown whether they occur at the beginning of the week (\EventString{a,b|{}}), the end of the week (\EventString{{}|a,b}), or somewhere in the middle (\EventString{{}|a,b|{}}), and so all of the possibilities appear together. It's also worth noting that strings use the Allen relation `meets' where the clue states ``the day before'', rather than the `before' relation. This is due to the fact that one day meets the next, and there is no time between them. The clues which give an event specific position might also have been equivalently written by specifying that the particular value appeared in the same box as one of the day names---for instance \EventString{mon,freezing}. However, the five day names which appear in the first clue of \Cref{tab:zebra-clue-strings} are not actually required for the puzzle: a string of five empty boxes suffices. The names are included here purely for convenience of reading.
\begin{center}
	\onehalfspacing
	\begin{tabular}[h!]{|p{0.9\textwidth}|}
		\hline
		\begin{itemize}
			\item $\{\EventString{mon|tue|wed|thu|fri}\}$
			\item $\{ \EventString{{}|fog,mild},\EventString{{}|fog,mild|{}},\EventString{fog,mild|{}} \}$
			\item $\{ \EventString{{}|tired,warm},\EventString{{}|tired,warm|{}},\EventString{tired,warm|{}} \}$
			\item $\{ \EventString{{}|jammed,overcast},\EventString{{}|jammed,overcast|{}},\EventString{jammed,overcast|{}} \}$
			\item $\{ \EventString{{}|cold,little},\EventString{{}|cold,little|{}},\EventString{cold,little|{}} \}$
			\item $\{ \EventString{{}|snow|overcast},\EventString{{}|snow|overcast|{}},\EventString{snow|overcast|{}} \}$
			\item $\{ \EventString{{}|sad,reading},\EventString{{}|sad,reading|{}},\EventString{sad,reading|{}} \}$
			\item $\{ \EventString{{}|rain,printing},\EventString{{}|rain,printing|{}},\EventString{rain,printing|{}} \}$
			\item $\{\EventString{{}|{}|average|{}|{}}\}$
			\item $\{\EventString{freezing|{}|{}|{}|{}}\}$
			\item $\{ \EventString{{}|stapling|happy},\EventString{{}|stapling|happy|{}},\EventString{stapling|happy|{}},$
			\item[] $~~\EventString{{}|happy|stapling},\EventString{{}|happy|stapling|{}},\EventString{happy|stapling|{}} \}$
			\item $\{ \EventString{{}|printing|angry},\EventString{{}|printing|angry|{}},\EventString{printing|angry|{}},$
			\item[] $~~\EventString{{}|angry|printing},\EventString{{}|angry|printing|{}},\EventString{angry|printing|{}} \}$
			\item $\{ \EventString{{}|lots,filing},\EventString{{}|lots,filing|{}},\EventString{lots,filing|{}} \}$
			\item $\{ \EventString{{}|shredding,hot},\EventString{{}|shredding,hot|{}},\EventString{shredding,hot|{}} \}$
			\item $\{ \EventString{{}|freezing|clear},\EventString{{}|freezing|clear|{}},\EventString{freezing|clear|{}},$
			\item[] $~~\EventString{{}|clear|freezing},\EventString{{}|clear|freezing|{}},\EventString{clear|freezing|{}} \}$	
		\end{itemize}\\
		\hline
	\end{tabular}\captionof{table}{Temporal Zebra puzzle clues as strings.}\label{tab:zebra-clue-strings}
\end{center}
Formally, there are five Attributes (Weather, Temperature, Traffic, Tasks, Mood), each of which is a set of Values. The vocabulary $\V$ of the puzzle is the union of the Attributes:
\begin{itemize}
	\item Weather = $\{rain,clear,fog,snow,overcast\}$
	\item Temperature = $\{freezing,cold,mild,warm,hot\}$
	\item Traffic = $\{none,little,average,lots,jammed\}$
	\item Tasks = $\{printing,stapling,reading,filing,shredding\}$
	\item Mood = $\{happy,angry,sad,tired,curious\}$
\end{itemize}
The external constraints are formalised as follows:
\begin{itemize}
	\item[] ``Each Value only lasts for one day.''
	\item[\ipp\label{impl:zebra-constraintsA}] $\forall v \in \V~ (x \in \llbracket P_v \rrbracket \Longrightarrow \forall v' \in \V~ (v' \neq v \wedge x \notin \llbracket P_{v'} \rrbracket))$
	\item[] ``Each day only contains one Value for each Attribute.''
	\item[\ipp\label{impl:zebra-constraintsB}] $\forall A \in Attributes~ (x \in \llbracket P_v \rrbracket \wedge v \in A \Longrightarrow \forall v' \in A~ (v' \neq v \wedge x \notin \llbracket P_{v'} \rrbracket))$
\end{itemize}
One further constraint is needed due to the nature of superposition allowing for the result of superposing two strings to be longer than either of the input strings:
\begin{itemize}
	\item[]``There are only 5 days.''
	\item[\ipp\label{impl:zebra-constraintsC}] $\forall v \in \V~ (x \in \llbracket P_v \rrbracket \Longrightarrow 1 \leq x \leq 5)$
\end{itemize}
Now, superposing all of the languages in \Cref{tab:zebra-clue-strings} and taking into account the constraints in \cref{impl:zebra-constraintsA,impl:zebra-constraintsB,impl:zebra-constraintsC}, the (singular) result string is generated (each Value is displayed here abbreviated to its first two letters):
\begin{align}
	\EventString{ra,fr,pr,ha|cl,co,li,st,an|fo,mi,av,re,sa|sn,wa,lo,fi,ti|ov,ho,ja,sh}
\end{align}
Finally, superposing this string with languages representing the questions that were asked, a string containing the full ``week schedule" is obtained (\cref{ex:zebra-string-full-solution}), and by taking the reduct of this string relative to $\{none,curious\}$ the solution to the puzzle can be found (\cref{ex:zebra-string-solution-reduct}, and superposed with a string of weekday names for ease of reading in \cref{ex:zebra-string-solution-weekdays}):
\begin{itemize}
	\item $\{\EventString{{}|none}, \EventString{{}|none|{}}, \EventString{none|{}}\}$
	\item $\{\EventString{{}|curious}, \EventString{{}|curious|{}}, \EventString{curious|{}}\}$
\end{itemize}
\begin{subequations}
	\begin{gather}
		\footnotesize
		s = \EventString{ra,fr,pr,ha,\underline{no}|cl,co,li,st,an|fo,mi,av,re,sa|sn,wa,lo,fi,ti|ov,ho,ja,sh,\underline{cu}}\label{ex:zebra-string-full-solution}\\
		\footnotesize
		\rho_{\{none,curious\}}(s) = \EventString{\underline{none}|{}|{}|{}|\underline{curious}}\label{ex:zebra-string-solution-reduct}\\
		\footnotesize
		\rho_{\{none,curious\}}(s) \sp \EventString{mon|tue|wed|thu|fri} = \EventString{mon,\underline{none}|tue|wed|thu|fri,\underline{curious}}\label{ex:zebra-string-solution-weekdays}
	\end{gather}
\end{subequations}
Thus the answer is that ``there is no traffic on the first day of the week", and ``I am curious on the last day of the week". This result is reproduced in \Cref{tab:zebra-solution} below for the sake of completeness.
\begin{ctable}[\label{tab:zebra-solution}]{|r || c | c | c | c | c |}{Solution to Temporal Zebra puzzle as in \cref{ex:zebra-string-full-solution}.}
	\hline
	 & Mon & Tue & Wed & Thu & Fri\\\hline
	Weather & $rain$ & $clear$ & $fog$ & $snow$ & $overcast$\\
	Temperature & $freezing$ & $cold$ & $mild$ & $warm$ & $hot$\\
	Traffic & $\underline{none}$ & $little$ & $average$ & $lots$ & $jammed$\\
	Task & $printing$ & $stapling$ & $reading$ & $filing$ & $shredding$\\
	Mood & $happy$ & $angry$ & $sad$ & $tired$ & $\underline{curious}$\\
	\hline
\end{ctable}
While the Attributes and Values for this variation of the puzzle are relatively meaningless, the use of strings to represent the constraints given in the clues maps well to using strings to model constraints for scheduling problems such as the job-shop problem \citep{manne1960job,applegate1991computational}, wherein a finite number of resources (agents) are available to complete a set of tasks, and agent can only complete a single task at a time, and the goal is to create a schedule that minimises the maximum of their completion times given that each task also has a specified order that it must be completed in---for instance, task $A$ may only begin after task $B$ is completed, and must be completed before task $C$. This is very similar to the constraints described in the variant of the Zebra Puzzle above, with the additional factor of seeking a string which encapsulates all of the constraints and also has the shortest duration. Another similar problem is the trains example in \citet{durand2008tool}, reproduced below in \Cref{tab:train-problem-clues}, which features six trains arriving and leaving a busy station. The task is to determine the minimum number of platforms needed, given that a train is not allowed to arrive at the same platform while another train is still there.
\begin{center}
	\begin{tabular}[h!]{|l|}
		\hline
		There are 6 trains: $\{A,B,C,D,E,F\}$.\\
		A train may not arrive on a platform if another train has not left that platform.\\
		$\bullet$ $A$, $B$, and $E$ reach the station at the same time.\\
		$\bullet$ $A$ leaves before $B$.\\
		$\bullet$ $A$ leaves after or at the same time as $C$, but before the arrival of $D$.\\
		$\bullet$ $D$ and $F$ arrive at the same time as $B$ is leaving.\\
		$\bullet$ $E$ and $D$ leave at the same time.\\
		\hline
	\end{tabular}
	\captionof{table}{Constraints of the Trains example from \citet[p. 3283]{durand2008tool}.}\label{tab:train-problem-clues}
\end{center}
\citet[p. 3298]{durand2008tool} use a formalism they call S-languages developed based on the concept of S-arrangements \citep{schwer2002s}, which uses arrangements of subsets of elements with repetitions in a manner which is similar in some ways to the string framework described in \cref{sub:strings}, using sequences of elements to represent event-like entities, generally described as either instantaneous points, or in terms of their beginning and end points.

The constraints in \Cref{tab:train-problem-clues} can be represented through the languages in \Cref{tab:train-problem-strings} below, where each `fluent' symbol appearing in a string component represents that train being currently at the station. Therefore, if multiple symbols appear in a box, then each of those trains requires a separate platform at the same time.
\begin{center}
	\singlespacing
	\begin{tabular}[h!]{|p{0.95\textwidth}|}
		\hline
		\begin{itemize}
			\item $\{\EventString{{}|A,B,E|A,B|{}}, \EventString{{}|A,B,E|E|{}}, \EventString{{}|A,B,E|A,B|B|{}},$
			\item[] $~~\EventString{{}|A,B,E|A,E|A|{}}, \EventString{{}|A,B,E|A,E|E|{}}, \EventString{{}|A,B,E|A,B|A|{}},$
			\item[] $~~\EventString{{}|A,B,E|B|{}}, \EventString{{}|A,B,E|A,E|{}}, \EventString{{}|A,B,E|A|{}},$
			\item[] $~~\EventString{{}|A,B,E|B,E|B|{}}, \EventString{{}|A,B,E|{}}, \EventString{{}|A,B,E|B,E|{}}, \EventString{{}|A,B,E|B,E|E|{}}\}$
			\item $\{\EventString{{}|A|{}|B|{}}, \EventString{{}|A|B|{}}, \EventString{{}|A|A,B|B|{}}, \EventString{{}|A,B|B|{}}, \EventString{{}|B|A,B|B|{}}\}$
			\item $\{\EventString{{}|A,C|A|D|{}}, \EventString{{}|C|{}|A|D|{}}, \EventString{{}|C|A|{}|D|{}}, \EventString{{}|C|A,C|{}|D|{}}, \EventString{{}|C|A|D|{}},$
			\item[] $~~\EventString{{}|C|A,C|A|{}|D|{}}, \EventString{{}|A,C|{}|D|{}}, \EventString{{}|A,C|A|{}|D|{}}, \EventString{{}|A|A,C|A|{}|D|{}},$
			\item[] $~~\EventString{{}|C|{}|A|{}|D|{}}, \EventString{{}|A|A,C|{}|D|{}}, \EventString{{}|A,C|D|{}}, \EventString{{}|A|A,C|A|D|{}},$
			\item[] $~~\EventString{{}|C|A,C|A|D|{}}, \EventString{{}|A|A,C|D|{}}, \EventString{{}|C|A,C|D|{}}\}$
			\item $\{\EventString{{}|B|B,D,F|D,F|D|{}}, \EventString{{}|B|B,D,F|D,F|F|{}}, \EventString{{}|B|B,D,F|D,F|{}}\}$
			\item $\{\EventString{{}|E|D,E|{}}, \EventString{{}|D|E,D|{}}, \EventString{{}|D,E|{}}\}$
		\end{itemize}\\
		\hline
	\end{tabular}\captionof{table}{Train scheduling problem constraints as languages of strings.}\label{tab:train-problem-strings}
\end{center}
Superposing these constraints together produces a language of 48 strings, each of which represents a different possible way of satisfying all the scheduling constraints together. In order to determine the minimum number of resources required---in this problem, the number of platforms in the train station---it is simply a matter of finding the string(s) whose largest component is smaller than the largest component of all other strings. In this case, since $B$ and $E$ arrive together, and $D$ and $F$ arrive before either of them have finished leaving, all 48 strings feature a component \EventString{B,D,E,F}---for instance, \allowbreak{}\EventString{{}|C|A,B,E|B,D,E,F|D,E,F|{}}---and thus the minimum number of platforms is the cardinality of that component: four.

Additionally, despite strings not explicitly denoting the durations of events, it is possible to find the timeline which is the least durative by finding the shortest string(s) which uses the maximum possible number of resources. In the case of the trains example, this is \EventString{{}|A,B,C,E|B,D,E,F|D,E,F|{}}. To justify this, consider two events $a$ and $b$ which have durations $\tau_a$ and $\tau_b$, respectively, such that the $a$ lasts longer than $b$, $\tau_a > \tau_b$. In this scenario, four relations are impossible, namely $a$ equals $b$, $a$ during $b$, $a$ starts $b$, and $a$ finishes $b$---all of these imply that $b$ has the same or shorter duration than $a$. Of the remaining nine, the duration of the string representing the relations are as follows in \Cref{tab:rel-durations}\footnote{Disregarding the durations of the empty bounding boxes, which have effectively infinite duration.}:
\begin{center}
	\begin{tabular}[h!]{|c c c|}
		\hline
		\textbf{$\bigcdot$} & \textbf{$\mathcal{S}_{\bigcdot}$} & \textbf{$\tau_{\mathcal{S}_{\bigcdot}}$}\\
		\hline
		$a$ before $b$ & \Before{a}{b} & $\tau_{\mathcal{S}_{\bigcdot}} > \tau_a + \tau_b$\\
		$a$ after $b$ & \After{a}{b} & $\tau_{\mathcal{S}_{\bigcdot}} > \tau_a + \tau_b$\\
		$a$ meets $b$ & \Meets{a}{b} & $\tau_{\mathcal{S}_{\bigcdot}} = \tau_a + \tau_b$\\
		$a$ met by $b$ & \iMeets{a}{b} & $\tau_{\mathcal{S}_{\bigcdot}} = \tau_a + \tau_b$\\
		$a$ overlaps $b$ & \Overlaps{a}{b} & $\tau_a + \tau_b > \tau_{\mathcal{S}_{\bigcdot}} > \tau_a$\\
		$a$ overlapped by $b$ & \iOverlaps{a}{b} & $\tau_a + \tau_b > \tau_{\mathcal{S}_{\bigcdot}} > \tau_a$\\
		$a$ started by $b$ & \iStarts{a}{b} & $\tau_{\mathcal{S}_{\bigcdot}} = \tau_a$\\
		$a$ finished by $b$ & \iFinishes{a}{b} & $\tau_{\mathcal{S}_{\bigcdot}} = \tau_a$\\
		$a$ contains $b$ & \iDuring{a}{b} & $\tau_{\mathcal{S}_{\bigcdot}} = \tau_a$\\
		\hline
	\end{tabular}
	\captionof{table}{Durations of relations between $a$ and $b$.}\label{tab:rel-durations}
\end{center}
As can be seen, a string cannot have a duration shorter than it's longest lasting event, but it may have a duration longer than the sum of all its events. The strings which maximise for component size (overlaps, overlapped by, started by, finished by, contains) all have a duration shorter than those which do not, of which the two shorter strings (meets, met by) have durations shorter than the two longer (before, after). The shortest strings of the maximising five (started by, finished by) have durations equal to the duration of the longest event, $a$. The relations overlaps, overlapped by, and contains all contain a component of the same maximum size, and are the same length, so in the scenario where multiple strings are found to be equally the shortest maximising for component size, a further step would be required to determine the types of relations found in the string---a string with more contains relations should be chosen as being less durative than one with more overlaps or overlapped by relations. Finally, if instead $a$ and $b$ have equal durations, then only one relation is possible between them---$a$ equals $b$, \EventString{{}|a,b|{}}---which also maximises for component size, and has the shortest string length of all the Allen relations.

This chapter has described a framework which uses strings as sequences of sets of temporal entities---times and events---as part of an approach to temporal semantics known as finite-state temporality. Strings of this type are interpretable as finite models of Monadic Second-Order logic, and so due to an equivalence with regular languages, this allows for strings to be recognised by finite-state automata and systems which use them. Several operations are described which can be used to put these strings to work---in particular, the superposition operation allows for strings and sets of strings to be combined, such that all of the temporal data contained therein is incorporated into new strings, and the projection operation allows for determining the linear ordering and relation between a specified subset of times and events in a string.

These operations feed into applications such as deriving timelines from annotated narratives, such as are found in the documents of the TimeBank corpus, a dataset of newswire texts marked up with TimeML, which can be used to provide insight into the overall temporal structure of the text, as well as check for inconsistencies. Additionally, strings can be used as a tool for some kinds of temporal constraint satisfaction, such as in scheduling problems, as the superposition operation can be used to combine strings which model temporal constraints, producing sets of one or more strings in which all the constraints hold successfully, or an empty set if the constraints cannot hold simultaneously.

The next chapter will describe the procedures by which these strings can integrated with systems which produce semantic annotation for temporal data---primarily focusing on TimeML---and additionally how other resources can be used to augment the work that strings can do, such as being able to make inferences about what relations must be added to a knowledge base in order to be able to draw new conclusions.

\newpage
\section{Methods}\label{sec:methods}
This chapter details the approaches that were taken to produce the results.

\subsection{Extracting Strings from Annotated Text}\label{sub:extracting}
The primary source of data is the TimeBank corpus which is one of the largest available collections of documents which are annotated with TimeML. Events are marked up with an \verb|<EVENT>| tag, and times with the \verb|<TIMEX3>| tag. Within TimeML, they are treated as intervals (see \cref{ssub:granularity}) which becomes relevant to the present work in the way they are related.

\subsubsection{Linking the TLINKs}\label{ssub:tlinks}
TimeML primarily uses \verb|<TLINK>| tags to annotate relations between marked up events and times. These serve as the basis for initial string creation, as they provide the two intervals that are being related, as well as the relation between them. The relation set used is specific to TimeML, but has its roots in the set of Allen Relations---see table \refneeded{} below.

A translation is built from \verb|<TLINK>| to string, using the tag's attributes to provide the data, and the Allen Relations as an intermediary step, as per table \refneeded{}. For example: \nb{example here please}. Any events and times that feature in the document but are not mentioned in a \verb|<TLINK>| are kept aside for now, as there is not a given connection from these events to the rest of the timeline.

After creating strings from the \verb|<TLINK>|s, the next step is to start building the timeline, using the superposition technique described in \cref{ssub:operations}. This allows for connections to be built between events and times that were not explicitly related by the annotation. If every time and event in the text is related to every other time and event, there is \textit{temporal closure} in the text (see \cref{ssub:allen} \nb{See transfer p6---maybe include the graph of $n(n-1)/2, n \ge 2$?}). For there to be temporal closure in a TimeML document would be a large amount of work for an annotator, as for $N$ events/times there would be $N(N - 1) / 2$ relations between them, which increases quickly: at $N = 10$ there are 45 relations, at $N = 50$ there are 1225 relations. This becomes unwieldy for a human to annotate, especially given that it is generally understood by the human reader that if $A$ precedes $B$, and $B$ precedes $C$, then $A$ also precedes $C$, so it feels unnecessary to include this latter relation. However, an automated system does not have this inherent knowledge, and needs a way to calculate the implied relations.

The various relations can be combined according to the table in \cite{allen1983maintaining} \citeneeded[ -- get the page and figure number], reproduced in \cref{ssub:allen} \refneeded{}, and again below using strings in table \refneeded{}. The advantage to using strings is that the extension beyond three events is simple \nb{Defend this---I think there is something in ISA13 paper---or remove}. It should be noted that several combinations of relations produce a disjunction of relations, which follows from the fact that these combinations of $a \bigcdot b$ and $b \bigcdot c$ don't provide enough information to determine the exact relation $a \bigcdot c$. It may be possible to narrow this down, though, with the addition of further relations connecting other intervals ($d,e,f,...$) with some or all of $a,b,c$.



\begin{center}
	\begin{tabular}[]{| l c c |}
		\hline
		\textbf{Relation} & \textbf{Count} & \textbf{Proportion}\\
		\hline
		BEFORE & 1408 & 21.94\%\\
		AFTER & 897 & 13.98\%\\
		IBEFORE & 34 & 0.53\%\\
		IAFTER & 39 & 0.61\%\\
		DURING & 302 & 4.71\%\\
		DURING\_INV & 1 & 0.02\%\\
		IS\_INCLUDED & 1357 & 21.14\%\\
		INCLUDES & 582 & 9.07\%\\
		BEGINS & 61 & 0.95\%\\
		BEGUN\_BY & 70 & 1.09\%\\
		ENDS & 76 & 1.18\%\\
		ENDED\_BY & 177 & 2.76\%\\
		SIMULTANEOUS & 671 & 10.45\%\\
		IDENTITY & 743 & 11.58\%\\
		\hline
		Total & 6418 & 100\%\\
		\hline
	\end{tabular}
\end{center}


\subsubsection{Handling Incomplete Data}\label{ssub:incomplete}
When dealing with any temporal information, there is often a lack of specificity that means temporal closure cannot be calculated. For example, knowing that event $a$ occurred before event $c$, as in \Before{a}{c}, and that event $b$ also occurred before $c$, as in \Before{b}{c}, does not impart any information on the relation between $a$ and $b$. If this is all of the available data, then it is only possible to choose this relation $a \bigcdot b, ~\bigcdot \in \mathcal{AR}$ with a precision of 1 in 13 chance. However, further information which includes other relations for each of $a$ and $b$ individually may eventually reveal their connection.

See also, using semi-intervals to simplify disjunctions of Allen Relations. Minimal sets of maximal strings.

Freksa relations using semi-intervals: the appearance of $\alpha(a)$ within a box represents a negation of the fluent $a$ conjoined with a formula stating that $a$ will be true in a subsequent box; similarly, $\omega(a)$ represents a negation of the fluent $a$ conjoined with a formula stating that $a$ was true in a previous box.

\begin{align}
%\alpha_a(x) := x \notin P_a \land \exists y(x < y \land y \in P_a \land \forall z(z < y \rightarrow z \notin P_a))\\
%\omega_a(x) := x \notin P_a \land \exists y(y < x \land y \in P_a \land \forall z(y < z \rightarrow z \notin P_a))
\alpha_a(x) ~:=~ \exists y ~(x < y ~\land~ y \in \llbracket P_a \rrbracket ~\land~ \forall z ~(z < y \Longrightarrow z \notin \llbracket P_a \rrbracket))\\
\omega_a(x) ~:=~ \exists y ~(y < x ~\land~ y \in \llbracket P_a \rrbracket ~\land~ \forall z ~(y < z \Longrightarrow z \notin \llbracket P_a \rrbracket))
\end{align}
Note that it is convenient to write $\alpha(a)$ for $\alpha_a(x)$ when using the box-notation.

A string may be translated to one using semi-intervals by placing $\alpha(f)$ in every box preceding one in which a fluent $f$ appears, and $\omega(f)$ in every box succeeding it, repeating this for each $f \in A$.\par
Thus a string $s = \EventString{{}|a|b|{}}$ becomes \nb{decide on semin-interval translation fn symbol}
\begin{align}
	\left\lceil s\right\rceil = \EventString{\alpha(a),\alpha(b)|\alpha(b)|\omega(a)|\omega(a),\omega(b)}
\end{align}
in which each of the fluents originally appearing in the string $s$ have also been removed in . It should be noticed that strings which use semi-intervals do not (necessarily) feature empty boxes at each end. This is due to the fact that a semi-interval need not be finitely bounded. In fact

This mechanism allows for partially known information to be represented using strings. For example, the string \EventString{\alpha(a), \alpha(b)|{}} represents the knowledge that the events labelled $a$ and $b$ both begin at the same moment, without stating anything about when they each finish---they may end simultaneously, $a$ may finish before $b$, or $b$ may finish before $a$. Which of these states is true is unknown without further data.

Semi-intervals allow for representing a greater range of event relations than the Allen relations do alone. Freksa \citeneeded{} describes 18 new relations which are made up of disjunctions of Allen's relations, based on the concept of cognitive neighbourhoods. That is, groupings of relations that make intuitive sense and can be derived from compositions of Allen relations \nb{expand, clarify `intuitive', decide between Allen/Allen's, try to use less `relations'?} \nb{include table or graphic of Freksa's relations, eg p18 or p21}.

Many of the Freksa relations can be captured using the semi-intervals \nb{duh, he designed them that way} in the box-notation.

For example, the relation \textit{older} corresponds with a disjunction of five Allen relations: \textit{before}, \textit{meets}, \textit{overlaps}, \textit{contains}, and \textit{finished by}.
\begin{center}
	\begin{tabular}[h!]{|r | l|}
	\hline
	\textbf{Relation} & \textbf{String}\\
	\hline
	before &\Before{a}{b}\\
	meets &\Meets{a}{b}\\
	overlaps &\Overlaps{a}{b}\\
	contains &\iDuring{a}{b}\\
	finished by&\iFinishes{a}{b}\\
	\hline
	\end{tabular}
	\captionof{table}{The Freksa relation \textit{older}.}
\end{center}
The similarities between these five relations becomes more apparent when they are translated to use semi-intervals instead:
\begin{center}
	\begin{tabular}[h!]{|r | l|}
	\hline
	\textbf{Relation} & \textbf{String}\\
	\hline
	before &\siBefore{a}{b}\\
	meets &\siMeets{a}{b}\\
	overlaps &\siOverlaps{a}{b}\\
	contains &\siiDuring{a}{b}\\
	finished by&\siiFinishes{a}{b}\\
	\hline
	\end{tabular}
	\captionof{table}{The Freksa relation \textit{older} using semi-intervals, before block compressed reduct.}
\end{center}

Each of these five strings projects (using a block compressed reduct) to the string \EventString{\alpha(a),\alpha(b)|\alpha(b)|{}}, meaning the relation $a$ \textit{older} $b$ can be represented using just this one string, instead of five. This does raise the question, however, of whether the tradeoffs are worth it: a great reduction in the cardinality of the timeline set can be achieved, but at the cost of using a more complex vocabulary and reducing the precision of the known information.

Unfortunately, only 10 of the 18 Freksa relations can be described using a single string without further complicating the vocabulary which may appear within a box in a string. Since all of these new relations are disjunctions of Allen relations, it follows that it may be acceptable to include disjunctions of semi-intervals inside a box, such as \ebox{\alpha(a) \lor \alpha(b)}, which is interpreted as one might expect: either $\alpha(a)$ appears in the box, or $\alpha(b)$ does, or they both do. This allowance admits a further 5 Freksa relations to be described in single strings. Of the remaining 3 relations, the \textit{unknown} relation may also be described using a single string, but it is trivial, since it encompasses a disjunction of all 13 Allen relations, and is formed by a simple disjunction of all possible semi-inteval symbols: \ebox{\alpha(a) \lor \alpha(b) \lor \omega(a) \lor \omega(b) \lor \epsilon}, or even more simply, with a string consisting of just a single empty box: \ebox{}.

Below is a table of the 18 Freksa relations, the Allen relations they comprise, and the string which they will project to. \nb{tidy up ordering of allens}

\begin{center}
	\footnotesize
	\begin{tabular}[h!]{|c | c | c|}
		\hline
		Freksa & Allen & string\\
		\hline
		unknown & b, bi, m, mi, s, si, f, fi, d, di, o, oi, e & \ebox{}\\
		older & b, m, o, di, fi & \EventString{\alpha(a),\alpha(b)|\alpha(b)|{}}\\
		younger & bi, mi, oi, d, f & \EventString{\alpha(a),\alpha(b)|\alpha(a)|{}}\\
		head to head & s, si, e & \EventString{\alpha(a),\alpha(b)|{}}\\
		tail to tail & f, fi, e & \EventString{{}|\omega(a),\omega(b)}\\
		survived by & b, m, s, d, o & \EventString{{}|\omega(a)|\omega(a),\omega(b)}\\
		survives & bi, mi, si, di, oi & \EventString{{}|\omega(b)|\omega(a),\omega(b)}\\
		born before death & b, m, s, si, f, fi, d, di, o, oi, e & \EventString{\alpha(a)|{}|\omega(b)}\\
		died after birth & bi, mi, s, si, f, fi, d, di, o, oi, e & \EventString{\alpha(b)|{}|\omega(a)}\\
		precedes & b, m & \EventString{\alpha(b) \lor \omega(a)}\\
		succeeds & bi, mi & \EventString{\alpha(a) \lor \omega(b)}\\
		contemporary & s, si, f, fi, d, di, o, oi, e & \EventString{\alpha(a) \lor \alpha(b)|{}|\omega(a) \lor \omega(b)}\\
		older contemporary & o, fi, di & \EventString{\alpha(a),\alpha(b)|\alpha(b)|{}|\omega(a) \lor \omega(b)}\\
		younger contemporary & oi, f, d & \EventString{\alpha(a),\alpha(b)|\alpha(a)|{}|\omega(a) \lor \omega(b)}\\
		surviving contemporary & di, si, oi & \EventString{\alpha(a)|{}|\omega(b)|\omega(a),\omega(b)}\\
		survived by contemporary & d, s, o & \EventString{\alpha(b)|{}|\omega(a)|\omega(a),\omega(b)}\\
		older and survived by & b, m, o & ...\\
		younger and survives & bi, mi, oi & ...\\
		\hline
	\end{tabular}
	\captionof{table}{The Freksa relations and the strings they project to.}
\end{center}
The last two relations on this list cannot be represented using a single string, and are instead must use conjunctions of pairs of strings: 
\begin{center}
	\footnotesize
	\begin{tabular}[h!]{|c | c | c|}
		\hline
		Freksa & Allen & strings\\
		\hline
		older and survived by & b, m, o & \EventString{\alpha(a),\alpha(b)|\alpha(b)|{}} $\land$ \EventString{{}|\omega(a)|\omega(a),\omega(b)}\\
		younger and survives & bi, mi, oi & \EventString{\alpha(a),\alpha(b)|\alpha(a)|{}} $\land$ \EventString{{}|\omega(b)|\omega(a),\omega(b)}\\
		\hline
	\end{tabular}
	\captionof{table}{The remaining Freksa relations and the strings they project to.}
\end{center}

While having to use a conjunction of strings may seem like a confounding issue which prevents all the relations from being representable by a single string, it is in fact less problematic than allowing disjunction inside a box, in some ways.
\subsection{Enhancing Strings using DRT}\label{sub:enhancing}
Automatic DRS-parsing of plain text is a task which has recently seen new approaches \citeneeded[ -- shared task, see fsmnlp paper sec 1]. It is possible to leverage the temporal information that features in a DRS to create strings. While the relations generally found using this approach are less specific than those found in TimeML \nb{compare the relations}, the advantage of using DRSs is that elements such as event participants are described.

Boxer \citep{Bos2008} is one available tool for parsing a text into one or more DRSs, though the temporal information in this version struggles a bit. A newer version of the tool is used as part of the Parallel Meaning Bank \citeneeded toolchain, although it has not been made publicly available at present.

\subsubsection{Parallel Meaning Bank}\label{ssub:pmb}
Under the assumption that the version of Boxer that is used in PMB would become available at some point, here is a description of PMB, and how the present work utilises the data within it as a demonstration of using text that has been parsed into DRSs can be transformed into strings for temporal reasoning.

Discourse relations are also a thing here.

\subsubsection{VerbNet and WordNet}\label{ssub:verbwordnet}
Two resources which are used in PMB and that can be leveraged are VerbNet, which supplies the semantic roles in a DRS, and WordNet, which is used to specify the particular sense of a verb (or other word). The version of Boxer used in the PMB includes this information in its output.

This data can be useful when trying to make links between events which have not been given a specific relation. For example, finding the closest hypernym between two verbs, or checking verbs which share participants may help to inform the building of relations (see \cref{ssub:ontology}).

\subsection{Reasoning with Strings}\label{sub:reasoning}
Strings of events are used to reason about the temporal information they contain: to infer a linear ordering and new relations between the events that were not previously stated. There are several methods which can be employed depending on the specific results that are desired.

Superposition of strings creates new strings which feature all of the constraints of their `parent' strings, and new relations can be derived by taking the projections of these strings in relation to an intersection of the parent vocabularies. Using resources such as VerbNet and WordNet (see \cref{ssub:verbwordnet}), the lexical semantics can be leveraged in order to suggest links between events which were not previously explicitly related. Additionally, residuals are employed to determine what relations may need to hold in order for other inferences to be made.

\subsubsection{Superposition and Projection}\label{ssub:superposition}
Having extracted strings from an annotated document, such as one of those in the TimeBank \citeneeded{} corpus, superposition can be used to work out the relations which are not yet made explicit. Strings model sets of constraints between the events they mention, and when two strings are superposed, all of these constraints also hold in the resulting language's strings. This can be shown through an example:
\begin{align}
\Before{a}{b} \spvc \Before{b}{c} = \EventString{{}|a|{}|b|{}|c|{}}
\end{align}
The constraint that event $a$ occurs before event $b$ holds in $s = \Before{a}{b}$, and also in the resulting string\footnote{Note that the result of a superposition is a language, but if its cardinality is 1, it can be conflated with its sole member string.} $s'' = \EventString{{}|a|{}|b|{}|c|{}}$. In order to prove this, take the \textit{projection} \refneeded[-- ref section 3.1]{} of $s''$ with respect to the vocabulary of $s$, and see that it is equal to $s$:
\begin{align}
\pi_{voc(s)}(s'') &= \bc(\rho_{voc(s)}(\EventString{{}|a|{}|b|{}|c|{}}))\\ \notag
				  &= \bc(\EventString{{}|a|{}|b|{}|{}|{}})\\ \notag
				  &= \Before{a}{b} = s
\end{align}
The string $s''$ thus \textit{projects} to the string $s$, which says that the information in $s$ is contained within $s''$. Similarly, the constraint that event $b$ occurs before event $c$ holds in both $s' = \Before{b}{c}$ and in $s''$; $s''$ projects to $s'$.

Since $s''$ projects to both to both $s$ and $s'$, $s''$ also models the constraints that are represented by both $s$ and $s'$, using a single string rather than two strings. However, $s''$ also projects to the string \Before{a}{c}, which represents the constraint that event $a$ occurs before event $c$. This constraint was not represented in either $s$ or $s'$, but is represented in $s''$. Thus, a new relation between events has been discovered and made explicit. 

\subsubsection{Event Ontology}\label{ssub:ontology}
Event hierarchy and using verb/word nets to figure out how events might relate to each other, e.g. using verb roles/participants to forge connections
\subsubsection{Residuals and Gaps}\label{ssub:residuals}
Finding out what additional information would be required on top of some set of premises in order to make some other conclusion(s) true.

\newpage
\section[String Temporal Annotation and Relation Tool (START)]{String Temporal Annotation and Relation Tool\\(START)}\label{sec:implementation}
Various approaches were trialled using different programming languages, including Prolog, JavaScript (Node.js), and Python. Ultimately, Python was chosen for its speed, wide cross-platform availability, and availability of compatible tooling.

Ultimately a widely-useable package was created that could be used either on the web or as a standalone application, as this gave the broadest opportunity to demonstrate the use of the developed technology.

It combined data and technologies from a number of sources to create a package that could be used to reason about temporal information, and augment annotation-based data in a way that is both efficient and simple-to-use \nb{hard claim to make without a study backing it up...}

\subsection{Back-end Pipeline}\label{sub:backend} % ?NLTK
This is the main stuff. Lots of python code all interacting with the data and PMB and Boxer (and the failure of old boxer and why the output is basically theoretical).

Describe all the moving parts, but mostly go into detail about the different API layers (i.e. multiple languages, tooling exposing only certain parts, NLTK having only some parts up to scratch), and the code that I personally wrote to stitch them all together, since that's basically the only sort-of valuable thing in this entire project.

pseudocode of superpose all langs sensible:

Superposes all langs sensible takes a list of languages of strings and attempts to superpose them all together, while avoiding superpositions of languages with no shared vocabulary and which will produce a result no longer than some user-specified limit (default is 12, since 13 usually is the result from no relation being found). When a superposition is avoided, the constituent languages are returned along with any successful superpositions.

\begin{verbatim}
superpose_sensible(a, b, lim):
  if a = b:
    # same string
    return {a}
  if Va = Vb:
    # inconsistent
    return {}
  if (Va intersect Vb) = 0 or len(superpose(a, b)) > lim:
    # disjoint vocab
    return {a, b}
  else:
    return superpose(a, b)

superpose_languages(A, B, lim):
  return superpose_sensible(a, b, lim) for a in A, for b in B

superpose_all_languages(list, lim):
  for i=0, i++, i < len(list):
    for j=i+1, j++, j < len(list)
      sp = superpose_lanugages(list[i], list[j], lim)
      if sp = {}:
        # inconsistent
      elif sp = {list[i], list[j]}:
        # superposition avoided
      else:
        newList = sp + (list - {list[i], list[j]})
        return superposition_all_languages(newList)
  return list
\end{verbatim}

Also taking advantage of Python's built-in caching, the most recent 100 (string) superpositions are stored, which can vastly improve the speed of the calculation. For example, using Python's timeit function to test how much time repetitions of a function takes, with the input $\{\EventString{{}|b,e|{}}, \EventString{{}|c|{}|d|{}}, \EventString{{}|f|{}|d|{}}, \EventString{{}|e|{}|d|{}}, \EventString{{}|a|a,b|a|{}}\}$, we find the following times:
\begin{center}
	\footnotesize
	\begin{tabular}[h!]{|r|c  c  c|}
		\hline
		Iterations & No caching (s) & Caching (s) & $\frac{caching}{no caching}$ (\%)\\
		\hline
		1 & 0.04596 & 0.00018 & 0.39\%\\
		100 & 1.60466 & 0.02367 & 1.48\%\\% 32 vs 100
		10000 & 156.47007 & 0.27928 & 0.18\%\\% 98 vs 14
		\hline
	\end{tabular}
	\captionof{table}{Caching superpositions improving efficiency.}
\end{center}
\subsection{Front-end Interface}\label{sub:frontend}
% Brython application with some JS to grease the wheels---why not TKinter? because HTML and CSS are convenient and widely used tools and allow for a web-based interface as well as an offline tool (electron for packaging? maybe?) why not just use js frontend and python in the backend? python generators are hard, and don't play nice when you want to use `next()' to use an effectively `pausable' function and get a new value.



\newpage
\section{Evaluation}\label{sec:evaluation}
Evaluation was performed on the basis of three main components: asserting that the results produced were not invalid (i.e. did not cause contradictions internally within a string/language or externally within a knowledge-base), performing inferences using the FRACAS semantic test suite, and verifying that all implementation code was tested so as to produce exactly and only the expected results. \nb{how and why does this add \textit{value}---data is compressed? human readable/intuitive?}
\subsection{Timeline Validity}\label{sub:validity}
If data is superposed or otherwise manipulated, no information is lost or falsified in the process.
\subsection{FRACAS Semantic Test Suite}\label{sub:fracas}
A de facto standard for testing semantic inferencing.
\subsection{Correctness of Code}\label{sub:correct}
Python test suites.
\newpage
\section{Conclusion}\label{sec:conclusion}
What have I spent the last four and a half years of my life doing?
%%%%%%%%%%%%%%%%%%%%%%%%%%%%%%%%%%%%%%%%%%%%%%%%%%%%%%%%%%%%%%%%%%%%%%%%%%%%%%%%
%                                                                              %
%                             Actual Document Ends                             %
%                                                                              %
%%%%%%%%%%%%%%%%%%%%%%%%%%%%%%%%%%%%%%%%%%%%%%%%%%%%%%%%%%%%%%%%%%%%%%%%%%%%%%%%

\newpage
\pagestyle{empty}
\onehalfspacing
\bibliographystyle{apa}
\bibliography{refs}
\newpage
%\appendix
\section*{Appendices}
\addcontentsline{toc}{section}{Appendices}
\subsection*{Python Code}
\addcontentsline{toc}{subsection}{Python Code}
% \lstinputlisting[language=Python,breaklines=true]{filename.py}
\end{document}
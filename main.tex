\documentclass[a4paper,12pt,leqno]{article}

%%%%%%%%%%%%%%%%%%%%%%%%%%%%%%%%%%%%%%%%%%%%%%%%%%%%%%%%%%%%%%%%%%%%%%%%%%%%%%%%
%                                                                              %
%                                     TODO                                     %
%                                                                              %
%%%%%%%%%%%%%%%%%%%%%%%%%%%%%%%%%%%%%%%%%%%%%%%%%%%%%%%%%%%%%%%%%%%%%%%%%%%%%%%%
%                                                                              %
% - Double check gutter margin is included                                     %
% - Make sure title page wording and formatting is acceptable                  %
%   - Should supervisors be included on title page?                            %
% - Does GDPR need to be included in Declaration?                              %
% - Make sure ADAPT acknowledgement is correct                                 %
% - Check all references are accurate and not abbreviated                      %
% - Find out if "Summary" should be included versus Abstract                   %
% - Include lists of tables/figures(?) and double-check for correct labels     %
% - Ensure 100k word limit is not reached... lol                               %
% - Check that appendices and other sections not automatically included in the %
%   TOC are accurate and up to date.                                           %
% - Should Related Publications be included?                                   %
% - Is Times a preferred font/does it matter?                                  %
% - Don't have solitary sections (e.g. 4.1.1 without 4.1.2)                    %
% - Vinny Wade and Dave Lewis may be internal examiners, so try to address the %
%   issues they brought up during the transfer                                 %
%                                                                              % 
%%%%%%%%%%%%%%%%%%%%%%%%%%%%%%%%%%%%%%%%%%%%%%%%%%%%%%%%%%%%%%%%%%%%%%%%%%%%%%%%


% Use I_A as correction to below for intervals, not nominals
% Use N_A for first order nominals, as first order ensures each element occurs exactly once, rather than L_A which allows infinite strings for the gap calculation thing

\usepackage{natbib}
%\usepackage{times}
\usepackage{url}
\usepackage{array}
\usepackage{latexsym}
\usepackage{caption}
\usepackage{amssymb,amsmath,amscd}
\usepackage{xcolor}
\usepackage[hang,flushmargin]{footmisc}
\usepackage{graphicx}  %%% for including graphics
%\usepackage[bindingoffset=10mm,margin=25mm]{geometry} % I think this is the required "gutter margin"
\usepackage[margin=25mm]{geometry}
\usepackage{setspace}
\usepackage{diagbox}
\usepackage[nodayofweek]{datetime}

\def\drs#1#2{
\begin{tabular}[c]{| c |}
	\hline #1 \\
	\hline #2 \\
	\hline
\end{tabular}
}

% Tim's custom commands
\newcommand{\bc}{{\rm b\!c}}
\newcommand{\unpad}{\mbox{{\rm unpad}}}
\newcommand{\vph}[1]{\vphantom{#1}}
\newcommand{\sta}[2]{\stackrel{#1}{#2}}


% David's custom commands
\newcommand{\ebox}[1]{\fbox{$\vph{'(),}#1$}}
\newcommand{\eboxl}[1]{\fbox{$\vph{'}#1$}}
\newcommand{\eboxh}[1]{\fbox{$\vph{,}#1$}}
\newcommand{\eboxb}[1]{\fbox{$\vph{@}#1$}}

\newcommand{\nbBefore}[2]{\ebox{#1}\ebox{}\ebox{#2}}
\newcommand{\nbMeets}[2]{\ebox{#1}\ebox{#2}}
\newcommand{\nbOverlaps}[2]{\ebox{#1}\ebox{#1,#2}\ebox{#2}}
\newcommand{\nbDuring}[2]{\ebox{#2}\ebox{#1,#2}\ebox{#2}}
\newcommand{\nbStarts}[2]{\ebox{#1,#2}\ebox{#2}}
\newcommand{\nbFinishes}[2]{\ebox{#2}\ebox{#1,#2}}
\newcommand{\nbEquals}[2]{\ebox{#1,#2}}

\newcommand{\nbAfter}[2]{\nbBefore{#2}{#1}}
\newcommand{\nbiMeets}[2]{\nbMeets{#2}{#1}}
\newcommand{\nbiOverlaps}[2]{\nbOverlaps{#2}{#1}}
\newcommand{\nbiDuring}[2]{\nbDuring{#2}{#1}}
\newcommand{\nbiStarts}[2]{\nbStarts{#2}{#1}}
\newcommand{\nbiFinishes}[2]{\nbFinishes{#2}{#1}}

\newcommand{\Before}[2]{\ebox{}\nbBefore{#1}{#2}\ebox{}}
\newcommand{\Meets}[2]{\ebox{}\nbMeets{#1}{#2}\ebox{}}
\newcommand{\Overlaps}[2]{\ebox{}\nbOverlaps{#1}{#2}\ebox{}}
\newcommand{\During}[2]{\ebox{}\nbDuring{#1}{#2}\ebox{}}
\newcommand{\Starts}[2]{\ebox{}\nbStarts{#1}{#2}\ebox{}}
\newcommand{\Finishes}[2]{\ebox{}\nbFinishes{#1}{#2}\ebox{}}
\newcommand{\Equals}[2]{\ebox{}\nbEquals{#1}{#2}\ebox{}}
\newcommand{\After}[2]{\ebox{}\nbAfter{#1}{#2}\ebox{}}
\newcommand{\iMeets}[2]{\ebox{}\nbiMeets{#1}{#2}\ebox{}}
\newcommand{\iOverlaps}[2]{\ebox{}\nbiOverlaps{#1}{#2}\ebox{}}
\newcommand{\iDuring}[2]{\ebox{}\nbiDuring{#1}{#2}\ebox{}}
\newcommand{\iStarts}[2]{\ebox{}\nbiStarts{#1}{#2}\ebox{}}
\newcommand{\iFinishes}[2]{\ebox{}\nbiFinishes{#1}{#2}\ebox{}}

\newcommand{\cBefore}[2]{``$#1$  before $#2$'' -- \Before{#1}{#2}}
\newcommand{\cMeets}[2]{``$#1$ meets $#2$'' -- \Meets{#1}{#2}}
\newcommand{\cOverlaps}[2]{``$#1$ overlaps $#2$'' -- \Overlaps{#1}{#2}}
\newcommand{\cDuring}[2]{``$#1$ during $#2$'' -- \During{#1}{#2}}
\newcommand{\cStarts}[2]{``$#1$ starts $#2$'' -- \Starts{#1}{#2}}
\newcommand{\cFinishes}[2]{``$#1$ finishes $#2$'' -- \Finishes{#1}{#2}}
\newcommand{\cEquals}[2]{``$#1$ equals $#2$'' -- \Equals{#1}{#2}}
\newcommand{\cAfter}[2]{``$#1$ after $#2$'' -- \After{#1}{#2}}
\newcommand{\ciMeets}[2]{``$#1$ imet by $#2$'' -- \iMeets{#1}{#2}}
\newcommand{\ciOverlaps}[2]{``$#1$ overlapped by $#2$'' -- \iOverlaps{#1}{#2}}
\newcommand{\ciDuring}[2]{``$#1$ contains $#2$'' -- \iDuring{#1}{#2}}
\newcommand{\ciStarts}[2]{``$#1$ started by $#2$'' -- \iStarts{#1}{#2}}
\newcommand{\ciFinishes}[2]{``$#1$ finished by $#2$'' -- \iFinishes{#1}{#2}}

\newcommand{\projects}[3]{\bc(\rho_{#3}(#1)) = #2}
\newcommand{\projectsVoc}[2]{\projects{#1}{#2}{voc(#2)}}

\renewcommand{\sp}{~\&~}
\newcommand{\spasync}{~\&_*~}
\newcommand{\spsigma}[1][\Sigma, \Sigma']{~\&_{#1}~}
\newcommand{\spvc}{~\&_{v\!c}~}

\renewcommand{\emptyset}{\varnothing}
\renewcommand{\phi}{\varphi}

% Allows entry of EventStrings as |a|{}|b,c|d|
% Use {} for empty box
\usepackage{etoolbox}
\DeclareListParser{\PipeParser}{|}
\newcommand{\EventString}[1]{
	\renewcommand*{\do}[1]{\ebox{##1}}%
	\PipeParser{#1}
}


%\usepackage[nottoc,numbib]{tocbibind} %This includes the bibliography as a numbered section in the TOC
\usepackage[nottoc]{tocbibind}

\doublespacing
\linespread{2} %Not sure about this, may need to reset to 1(?)
\newdateformat{monthyeardate}{\monthname[\THEMONTH] \THEYEAR}

\title{\textbf{Strings for Temporal Annotation and\\Semantic Representation of Events}}

\author{by\\{\textbf{David Woods}}\bigskip\bigskip}

\date{\parbox{\linewidth}{\centering%
		{\large A dissertation submitted\\in fulfillment of the requirements\\for the Degree of\\\textbf{Doctor of Philosophy}}\\		
		\bigskip\bigskip\bigskip
		{\Large \textbf{University of Dublin, Trinity College}}\\\endgraf \monthyeardate\today}}

\begin{document}
\maketitle
\thispagestyle{empty}
%\linespread{1}

\newpage
\pagenumbering{roman}
\section*{Declaration}
\addcontentsline{toc}{section}{Declaration}
\noindent
I, the undersigned, declare that this thesis has not been submitted as an exercise for a degree at this or any other university and it is entirely my own work.\\

\noindent
I, the undersigned, agree to deposit this thesis in the University's open access institutional repository or allow the Library to do so on my behalf, subject to Irish Copyright Legislation and Trinity College Library conditions of use and acknowledgement.

%\noindent
%I, the undersigned, consent / do not consent to the examiner retaining a copy of the thesis beyond the examining period, should they so wish (EU GDPR May 2018).

\vspace{\fill}

\begin{table*}[!htbp]
	\flushright
	\begin{tabular}{l}
		\makebox[10cm]{\hrulefill}\\[0.5cm]
		David Woods\\[0.25cm]
		{\monthyeardate\today}
	\end{tabular}
\end{table*}

\vspace{5em}

\newpage
\begin{abstract}
\addcontentsline{toc}{section}{Abstract}
\noindent
%This work describes the use of strings as models for the representation of temporal data -- i.e. events and times -- to form the basis of a framework for reasoning about that data. Some of the relevant motivating literature is examined, and a breakdown is given of the work done to develop and flesh out the framework so far, including discussion on superposition for collation of information into single, timeline-like strings, and projection which allows for the identification of temporal relations between arbitrary events and times from the strings. Possible ways of treating incomplete information are also looked at, including moving from intervals as primitives to semi-intervals. Some work done to implement this framework in code is described, with a discussion of potential applications in modern intelligent systems, including tooling for annotation software.
\end{abstract}

\newpage
\section*{Acknowledgements}
\addcontentsline{toc}{section}{Acknowledgements}
%My thanks to Tim for his patience and understanding, to Carl for guiding and reassuring comments, and to my friends and family for their continuous encouragement and support. In particular, Brian, who had the misfortune to be staying with me while I was working on this report, and the members of DU Trampoline club, who have had a more bad-tempered coach of late.

This research is supported by Science Foundation Ireland (SFI) through the CNGL 
Programme (Grant 12/CE/I2267) in the ADAPT Centre 
(\url{https://www.adaptcentre.ie}) at Trinity College Dublin. The
ADAPT Centre for Digital Content Technology is funded under the SFI Research 
Centres Programme (Grant 13/RC/2106) and is co-funded under the European 
Regional Development Fund.


\newpage
\section*{Related Publications}
\addcontentsline{toc}{section}{Related Publications}
% what they are and how they figure in, but don't go into it


\newpage
\tableofcontents
\newpage
\pagenumbering{arabic}
%%%%%%%%%%%%%%%%%%%%%%%%%%%%%%%%%%%%%%%%%%%%%%%%%%%%%%%%%%%%%%%%%%%%%%%%%%%%%%%%
%                                                                              %
%                            Actual Document Begins                            %
%                                                                              %
%%%%%%%%%%%%%%%%%%%%%%%%%%%%%%%%%%%%%%%%%%%%%%%%%%%%%%%%%%%%%%%%%%%%%%%%%%%%%%%%
\section{Introduction}\label{sec:intro}
\newpage
\section{Relevant Literature}\label{sec:litreview}
\subsection{Times and Events}
\subsubsection{Allen Relations}
\subsubsection{Tense and Aspect} % Reichenbach, Vendler, TEA
% Aspect hypothesis of dowty - build from statives - this is behind block-compression/destuttering
% as difference for Schwer s-words
\subsection{Annotation}
% Derczinski
\subsubsection{ISO-TimeML}
\subsubsection{TimeBank}
\subsection{Semantics}
\subsubsection{Discourse Representation Theory}
% Bunt's use for semantic annotation - can we get his slides from the DCLRS?
\subsubsection{Boxer}
\newpage
\section{Finite State Temporality}
\subsection{Strings for Times and Events}
\subsubsection{Creating Strings}
\subsubsection{Granularity: Points vs Intervals vs Semi-intervals}
\subsubsection{String Operations}
\subsection{Applications}
\subsubsection{Timelines from Texts}
\subsubsection{Scheduling (Zebra Puzzle)}
\newpage
\section{Methods}\label{sec:methods}
\subsection{Extracting Strings from Annotated Text}\label{sec:annotation}
\subsubsection{TLINKs}
\subsection{Enhancing Strings using DRT}\label{sec:semantics}
\subsubsection{Parallel Meaning Bank}
\subsubsection{VerbNet and WordNet}
\subsection{Inference via Residuals}
\newpage
\section{Implementation}\label{sec:implementation}
\subsection{Computational Pipeline} % ?NLTK

\newpage
\section{Evaluation}\label{sec:evaluation}
\subsection{Timeline Validity}\label{sub:strcreation}
\subsection{FRACAS Semantic Test Suite}
\subsection{Correctness of Code}\label{sub:correct}
\newpage
\section{Conclusion}\label{sec:conclusion}

%%%%%%%%%%%%%%%%%%%%%%%%%%%%%%%%%%%%%%%%%%%%%%%%%%%%%%%%%%%%%%%%%%%%%%%%%%%%%%%%
%                                                                              %
%                             Actual Document Ends                             %
%                                                                              %
%%%%%%%%%%%%%%%%%%%%%%%%%%%%%%%%%%%%%%%%%%%%%%%%%%%%%%%%%%%%%%%%%%%%%%%%%%%%%%%%

\newpage
\pagestyle{empty}
\onehalfspacing
\bibliographystyle{apa}
\bibliography{refs}
\newpage
%\appendix
\section*{Appendices}
\addcontentsline{toc}{section}{Appendices}
\subsection*{Python Code}
\addcontentsline{toc}{subsection}{Python Code}
\end{document}